% \iffalse meta-comment
% Copyright (C) 2006-2009 by Martin Scharrer <martin@scharrer-online.de>
% http://www.scharrer-online.de/latex/
% -----------------------------------------------------------------
%
% This work may be distributed and/or modified under the
% conditions of the LaTeX Project Public License, either version 1.3
% of this license or (at your option) any later version.
% The latest version of this license is in
%   http://www.latex-project.org/lppl.txt
% and version 1.3 or later is part of all distributions of LaTeX
% version 2005/12/01 or later.
%
% This work has the LPPL maintenance status `maintained'.
%
% The Current Maintainer of this work is Martin Scharrer.
%
% This work consists of the files svn-multi.dtx and svn-multi.ins
% and the derived files svn-multi.sty and svnkw.sty.
% $Id$
% \fi
% \iffalse
%<*package|driver|wrapper>
\def\filedate$#1: #2 #3 #4-#5-#6 #7 ${%
 \year#4\month#5\day#6\relax
 \def\filedate{#4/#5/#6}%
 \def\filerev{#3}%
}
\filedate$Id$
\def\fileversion{v2.0 beta 1}
%</package|driver|wrapper>
%<*driver>
\ProvidesFile{svn-multi.dtx}
 [\filedate\space\fileversion\space SVN Keywords for multi-file LaTeX documents]

\documentclass{ltxdoc}
\usepackage{svn-multi}
\usepackage{ifpdf}
\ifpdf
  % use hypdoc if you have it, hyperref else
  %\usepackage{hyperref}
  \usepackage{hypdoc}
  \urlstyle{tt}
\else\let\url=\texttt\fi
\usepackage{xspace}
\newcommand{\ie}{i.e.\@\xspace}
\newcommand{\eg}{e.g.\@\xspace}

\iffalse
\let\css=\cs
\let\op=\cs
\let\DescribeOption\DescribeMacro
\let\DescribeScript\DescribeOption
\else % crossreference of macros in documentation
\makeatletter

\usepackage{xspace}
\@namedef{seen@package@latex}{1} %^^A avoid footnotes for 'latex'
\newcommand*{\pkg}[1]{%
  \href{http://tug.ctan.org/pkg/#1}{\texttt{#1}}%
  % URL footnote (for print-out) on first appearance:
  \@ifundefined{seen@package@#1}{%
    \footnote{CTAN: \url{http://tug.ctan.org/pkg/#1}}%
    \@namedef{seen@package@#1}{1}%
  }{}%
  \xspace
}
\newcommand*{\svnmulti}{%
  \texttt{svn-multi}\xspace%
}

% link \cs to macro definitions
\let\origmacro\macro
\let\origendmacro\endmacro
\let\origStopEventually\StopEventually
\let\origPrintDescribeMacro\PrintDescribeMacro
\usepackage{xcolor}
\definecolor{darkred}{rgb}{0.333.0.0,0.0}
\hypersetup{colorlinks=true,linkcolor=darkred,urlcolor=darkred}
\definecolor{macrodesccolor}{rgb}{0.0,0.0,0.8}
\definecolor{macroimplcolor}{rgb}{0.0,0.0,0.4}
\definecolor{metacolor}{rgb}{0.0,0.4,0.4}
\definecolor{scriptcolor}{rgb}{0.2,0.6,0.2}
\definecolor{optioncolor}{rgb}{0.3.0.2,0}

\let\macroline\\
\newlength{\macrosep}
\setlength{\macrosep}{-3em}
\renewcommand{\meta@font@select}{\color{metacolor}\itshape}
\newcommand{\macroformat}[1]{\textbf{\ttfamily #1}}
\newcommand{\optionformat}[1]{\textbf{\sffamily #1}}
\newcommand{\scriptformat}[1]{\textbf{\ttfamily #1}}
\newcommand{\macroargformat}[1]{\texttt{#1}}
\newcommand{\scriptargformat}[1]{\textbf{#1}}
\newcommand{\macrohlinkprefix}{desc}
\newcommand{\macrolink}{}

\usepackage[T1]{fontenc}
\usepackage{lmodern}

\def\DescribeMacro{\@ifnextchar*{\DescribeMacroS}{\DescribeMacroN}}
\def\DescribeMacroN{%
  \bigskip\pagebreak[3]\par\noindent\DescribeMacroS*%
}
\def\DescribeMacroS*#1#2{%
  \begingroup
  \g@namedef{href@desc@#1}{}%
  \immediate\write\@mainaux{%
    \noexpand\g@namedef{href@desc@#1}{}%
  }%
  \@ifundefined{href@impl@#1}%
    {\let\macrolink\relax}%
    {\def\macrolink{\hyperlink{impl@#1}}}%
  \hypersetup{linkcolor=macrodesccolor}%
  \hspace*{\macrosep}%
  \raisebox{\baselineskip}[\baselineskip]{\hypertarget{desc@#1}{}}%
  \macrolink{\macroformat{\textcolor{macrodesccolor}{\textbackslash #1}}}%
  \noindent\mbox{}\macroargformat{#2}\nopagebreak
  \macroline*[0.2\baselineskip]%
  \endgroup
  \nopagebreak
  \ignorespaces
}
\def\DescribeScript{\@ifnextchar*{\DescribeScriptS}{\DescribeScriptN}}
\def\DescribeScriptN{%
  \bigskip\par\pagebreak[2]\noindent\DescribeScriptS*%
}
\def\DescribeScriptS*#1#2{%
  \hspace*{\macrosep}%
  \raisebox{\baselineskip}[\baselineskip]{\hypertarget{script@#1}{}}%
  \scriptformat{\textcolor{scriptcolor}{#1}}%
  \noindent\mbox{}\scriptargformat{\ {#2}}\macroline*[0.2\baselineskip]%
  \nopagebreak
}
\def\DescribeOption{\@ifnextchar*{\DescribeOptionS}{\DescribeOptionN}}
\def\DescribeOptionN{%
  \bigskip\par\noindent\DescribeOptionS*%
}
\def\DescribeOptionS*#1{%
  \hspace*{\macrosep}%
  \raisebox{\baselineskip}[\baselineskip]{\hypertarget{option@#1}{}}%
  \optionformat{\textcolor{optioncolor}{#1}}%
  \noindent\mbox{}\macroline*[0.2\baselineskip]%
  \nopagebreak
}
\newcounter{macrolevel}
\renewenvironment{macro}[1]{%
  \addtocounter{macrolevel}{1}%
  \expandafter\macroX\expandafter{\expandafter\@gobble\string#1}%
}{%
  \addtocounter{macrolevel}{-1}%
}
\providecommand*{\g@namedef}[1]{%
  \expandafter\gdef\csname #1\endcsname
}
\newcommand*{\macroX}[1]{%
  \ifnum\c@macrolevel<2
    \smallskip
  \fi
  \par\noindent
  \g@namedef{href@impl@#1}{}%
  \immediate\write\@mainaux{%
    \noexpand\g@namedef{href@impl@#1}{}%
  }%
  \@ifundefined{href@desc@#1}%
    {\let\macrolink\relax}%
    {\def\macrolink{\hyperlink{desc@#1}}}%
  \hspace*{\macrosep}%
  \raisebox{\baselineskip}[\baselineskip]{\hypertarget{impl@#1}{}}%
  \macrolink{\macroformat{%
    \textcolor{macroimplcolor}{\textbackslash #1}}}%
  \\*[\smallskipamount]%
  \@ifnextchar\begin{\vspace*{-\baselineskip}}{\imacroarg}%
}

\newcounter{macroargs}
\newcounter{nmacroarg}

\newcommand*{\imacroarg}[1][0]{%
  \setcounter{macroargs}{#1}%
  \setcounter{nmacroarg}{1}%
  \ifnum\c@macroargs>0
    \expandafter\imacroargX
  \fi
}
\newcommand*{\aftermacroargs}{%
  \@ifnextchar\begin
    {\\*[-2ex]\ignorespaces}%
    {\\*[\smallskipamount]\ignorespaces}%
}
\newcommand*{\imacroargX}[1]{%
  \hspace*{-1em}\texttt{\#\thenmacroarg:} #1\relax
  \ifnum\c@macroargs>1
    \newline
  \fi
  \addtocounter{nmacroarg}{1}%
  \addtocounter{macroargs}{-1}%
  \ifnum\c@macroargs>0
    \expandafter\imacroargX
  \else
    \expandafter\aftermacroargs
  \fi
}


\def\karg#1{\{\$\textcolor{metacolor}{#1}\$\}}
\def\kmarg#1{\{\$\meta{#1}\$\}}

\DeclareRobustCommand{\csi}[1]{%
  \begingroup
  \hypersetup{linkcolor=macroimplcolor}%
  \renewcommand{\macrohlinkprefix}{impl}%
  \@ifundefined{href@impl@#1}%
    {\let\macrolink\relax}%
    {\def\macrolink{\hyperlink{impl@#1}}}%
  \csX{#1}%
  \endgroup
}
\DeclareRobustCommand{\csd}[1]{%
  \begingroup
  \hypersetup{linkcolor=macrodesccolor}%
  \renewcommand{\macrohlinkprefix}{macro}%
  \@ifundefined{href@desc@#1}%
    {\let\macrolink\relax}%
    {\def\macrolink{\hyperlink{desc@#1}}}%
  \csX{#1}%
  \endgroup
}
\DeclareRobustCommand{\csX}[1]{%
  \begingroup
  \macrolink{\texttt{\textbackslash#1}}%
  \endgroup
}
\let\cs\csd
\DeclareRobustCommand{\css}[1]{\texttt{\textbackslash#1}}
\DeclareRobustCommand{\op}[1]{%
  \begingroup
  \hypersetup{linkcolor=optioncolor}%
  \hyperlink{option@#1}{\textbf{\sffamily #1}}%
  \endgroup
}
\DeclareRobustCommand{\scr}[1]{%
  \begingroup
  \hypersetup{linkcolor=scriptcolor}%
  \hyperlink{script@#1}{\scriptformat{#1}}%
  \endgroup
}

\def\StopEventually#1{\origStopEventually{#1}%
\let\cs\csi
}

\fi

\EnableCrossrefs
%\DisableCrossrefs
\CodelineIndex
%\PageIndex
\RecordChanges
%\OnlyDescription
\widowpenalty=500
\clubpenalty=500
\begin{document}
  \DocInput{svn-multi.dtx}%
  \PrintChanges
  \clearpage
  \PrintIndex
\end{document}
%</driver>
%<*package>
% \fi
%
% \CheckSum{0}
%
% {\makeatother
% \CharacterTable
%  {Upper-case    \A\B\C\D\E\F\G\H\I\J\K\L\M\N\O\P\Q\R\S\T\U\V\W\X\Y\Z
%   Lower-case    \a\b\c\d\e\f\g\h\i\j\k\l\m\n\o\p\q\r\s\t\u\v\w\x\y\z
%   Digits        \0\1\2\3\4\5\6\7\8\9
%   Exclamation   \!     Double quote  \"     Hash (number) \#
%   Dollar        \$     Percent       \%     Ampersand     \&
%   Acute accent  \'     Left paren    \(     Right paren   \)
%   Asterisk      \*     Plus          \+     Comma         \,
%   Minus         \-     Point         \.     Solidus       \/
%   Colon         \:     Semicolon     \;     Less than     \<
%   Equals        \=     Greater than  \>     Question mark \?
%   Commercial at \@     Left bracket  \[     Backslash     \\
%   Right bracket \]     Circumflex    \^     Underscore    \_
%   Grave accent  \`     Left brace    \{     Vertical bar  \|
%   Right brace   \}     Tilde         \~}
% }
% \changes{v1.0}{2006/05/27}{Initial version}
% \changes{v1.1}{2006/06/08}{Added macros to extract and typeset date/time
% information. Added macros to set and typeset main URL or filename.}
% \changes{v1.2}{2007/06/22}{Renamed package from \texttt{svnkw} to
% \texttt{svn-multi} to match CTAN directory. Wrapper file
% \texttt{svnkw.sty} is provided for backward compatibility.}
% \changes{v1.3}{2007/07/01}{Added verbatim support. Keywords can now contain
% special character like \texttt{\_ \^{} \$ \% \& \textbackslash}. Rewrote
% keyword check macros to work with verbatim code. \texttt{\textbackslash
% nofiles} is now obeyed.}
% \changes{v1.3a}{2007/07/10}{Fixed issue with unwanted spaces generated by
% \cs{svnid}, \cs{svnidlong} and \cs{svnkwsave}, \eg when used in a file which
% is included with \css{input}}
% \changes{v1.3b}{2008/12/03}{Changed the way catcodes are modified to be
% compatible with the french option of the babel package or other packages which
% modify the list of special characters.}
% \changes{v1.4}{2009/02/27}{Added support for timezones with non-zero minute
% part, \eg +0530.}
% \changes{v1.5}{2009/02/28}{Added \css{today}-style macros \cs{svntoday} and
% \cs{svnfiletoday}.}
% \changes{v2.0}{2009/03/12}{New features: keyword groups, external file
% support, auto-include of images, files-as-groups and table of revisions.}
%
% ^^A \GetFileInfo{svn-multi.dtx}
%
% \DoNotIndex{\newcommand,\newenvironment,\AtBeginDocument,\AtEndDocument}
% \DoNotIndex{\def,\let,\edef,\xdef,\item,\space,\write,\jobname,\relax,\!}
% \DoNotIndex{\closeout,\csname,\DeclareRobustCommand,\else,\empty,\newwrite}
% \DoNotIndex{\endcsname,\expandafter,\fi,\Hurl,\hyper@normalise,\@ifnextchar}
% \DoNotIndex{\ifnum,\@ifundefined,\ifx,\immediate,\InputIfFileExists,\ }
% \DoNotIndex{\newcount,\noexpand,\openout,\PackageWarning,\@percentchar}
% \DoNotIndex{\@sanitize,\@makeother,\@iwsvn,\%,\_,\&,\^,\$,\#,\ ,\\,\if@filesw}
% \DoNotIndex{\gdef,\begingroup,\endgroup,\catcode}
% \DoNotIndex{\^,\ ,\_,\(,\),\$,\&,\#,\@ampersamchar,\AtEndOfPackage}
% \DoNotIndex{\@backslashchar,\begin,\bgroup,\chapter,\day}
% \DoNotIndex{\DeclareOption,\do,\dospecials,\@dottedtocline,\egroup}
% \DoNotIndex{\end,\ExecuteOptions,\filedate,\fileversion,\@for}
% \DoNotIndex{\futurelet,\g@addto@macro,\global,\@gobbletwo,\hline}
% \DoNotIndex{\hspace,\if@restonecol,\if@twocolumn,\ignorespaces}
% \DoNotIndex{\makeatletter,\MakeUppercase,\@mkboth,\month,\@namedef}
% \DoNotIndex{\NeedsTeXFormat,\newif,\onecolumn,\orig@fink@prepare}
% \DoNotIndex{\orig@fink@restore,\PackageError,\ProcessOptions}
% \DoNotIndex{\ProvidesPackage,\renewcommand,\RequirePackage}
% \DoNotIndex{\@restonecolfalse,\@restonecoltrue,\section,\strut}
% \DoNotIndex{\tableofcontents,\tableofrevisions,\texttt,\today}
% \DoNotIndex{\twocolumn,\@undefined,\url,\year}
%
% \title{The \textsf{svn-multi} package
%   \\formerly known as \textsf{svnkw}}
% \author{Martin Scharrer \\ \url{martin@scharrer-online.de} \\
% \url{http://www.scharrer-online.de/latex/svn-multi}\\
% CTAN: \url{http://tug.ctan.org/pkg/svn-multi}}
% \date{Version \expandafter\@gobble\fileversion\\[0.5ex]\today}
%
% \ifpdf
% \hypersetup{
%   pdfauthor={Martin Scharrer <martin@scharrer-online.de>},
%   pdftitle={The svn-multi package, \fileversion, r\filerev},
%   pdfsubject={Documentation of LaTeX package svn-multi which allows the
%   typesetting of Subversion keywords in multi-file LaTeX documents},
%   pdfkeywords={svn-multi, svnkw, LaTeX, Subversion, keywords, Version
%   Control, Id, Table Of Revisions}
% }
% \fi
% \maketitle
%
% \vspace{3\baselineskip}
% \section{Package naming}
% The authors first choice for this package was |svnkw| but the CTAN
% maintainer suggested a more descriptive name and put the package in the
% \svnmulti\footnote{CTAN:
% \url{http://www.ctan.org/tex-archive/macros/latex/contrib/svn-multi/}}
% directory. Therefore the style file got renamed to \svnmulti, but a
% |svnkw| dummy style file which loads the new package is still provided
% for backward compatibility.
%
% \section{Introduction}
% This package lets you typeset keywords of the version control system
% Subversion\footnote {Subversion homepage: \url{http://subversion.tigris.org/}}
% (svn), which is the successor of the popular CVS, inside your \LaTeX{} files
% anywhere you like.  Unlike the package \pkg{svn} the usage of
% multiple files for one \LaTeX{} document is well supported. The package
% acquires the keywords of the last changed file and provides them to the user
% through macros. The package has to read all keywords of all files first and
% writes the most recent values in an auxiliary file with a `|.svn|' extension.
% This file is read back at the next \LaTeX{} run which introduces a delay like
% by the table of contents. The standard \LaTeX{} switch |\nofiles| can be used
% to suppress the file generation. Macros to typeset the keywords of the current
% |\include|-d or |\input|-ed \LaTeX{} file are also provided.
%
% \newpage
% \subsection{Scope of Keywords}
% This package provides the Subversion keyword data in several different scopes:
% document-global, file-local and, new with v2.0, by group.
%
% \subsubsection*{Document global}
% The normal macros, \eg \cs{svnrev}, return the latest version control
% information (keyword data) for the whole multi-file document, \ie the
% information of the latest changed file of the document. To collect, sort and
% provide this information is the main functionality of this package.
%
% \subsubsection*{Local to current file}
% There are also other macros, \eg \cs{svnfilerev}, which return the version
% control information of the current file, \ie the file they are used in. It is
% assumed here that every file using this macros calls first either a \cs{svnid}
% or \cs{svnidlong} macro or both.  See section~\ref{sec:usage:id} for more
% details about the id macros.  Please note that the file-local macros
% technically actually return the \emph{last registered} information from the
% last \cs{svnid} or \cs{svnidlong} even when these were in a previous file. 
% This will cause wrong results if they are used in a file before or without any
% id macros.
%
% \subsubsection*{Groups}
% Version 2.0 introduces the concept of groups. Several files\footnote{actually
% several id macros} of a multi-file \LaTeX\ document can be grouped together
% and the latest version control information of all files of a group is
% provided by macros. This works in the same way as the global macros mentioned
% above but only with the files in the group. It can also be seen from the other
% side: the macros are local like the file-local macros mentioned above but for
% all files of the group, not only the current one.
%
% This groups could also be called \textit{file groups}, \textit{keyword groups}
% or, like in programming languages, \textit{namespaces}. In this manual they
% will be reference as simple \textit{groups} most the time. In places where
% they could be confused with \TeX\ groups (|{ }|, |\begingroup| |\endgroup|),
% \eg ``in the current group'' or ``group local'', they will be called
% \textit{keyword groups}.
%
% There is no limitation (besides internal \LaTeX\ resource limits) for the
% number of different groups. The files of one group do not have to be included
% in a row but can be included everywhere in the document. The version control
% information of the current group can be typeset with macros like
% \cs{svncgrev} (|cg| for \emph{current group}).  Also, a general but less
% robust macro \cs{svng}\marg{group name}\marg{key} is provided to access others
% groups by name everywhere in the document. To avoid some macro robustness
% problems the current group can be changed locally for the output macros using
% \cs{svnsetcg}\marg{group name}.
%
% See section~\ref{sec:group} for further details and usage instructions on
% group macros.
%
% \section{Usage}
% The version control information are provided by Subversion keywords which
% first need to be read in by dedicated macros and can then be typeset using
% different macros.
%
% \subsection{Package Options} Since v2.0 this package provides options to
% enable only a needed features, \eg to avoid problems with other packages or
% save \TeX\ memory. For backwards compatibility to pre-2.0 package versions all
% old features are enabled by default and all new features are disabled to
% save a little of \TeX\ memory.
%
% All options except the first two are boolean key=value options (so far) and
% await either `|true|' or `|false|' as value. A missing value means `|true|'.
% So \eg |[groups=true,verbatim=false,external]|, enables the \op{external} and
% \op{groups} options but disables the \op{verbatim} option.
%
% The available options are:\par
%
% \DescribeOption{old}
% Only pre-v2.0 features are active. This enables \op{verbatim} and disables all
% other options below. This is the default for reasons mentioned above.
%
% \DescribeOption{all}
% Activates all features of the package.
%
% \DescribeOption{verbatim}
% Controls the verbatim mode of the keyword parser
% macros. Normally verbatim mode is very much wanted to support strange
% characters in URLs and file names, but this options gives the user a
% possibility to disable verbatim, \eg for trouble shooting.  Please note that
% verbatim mode is needed in order to make \svnmulti work with some packages,
% like \pkg{babel} with the |french| option.
%
% \DescribeOption{external}
% Controls the support for keywords from external files
% described in section~\ref{sec:external}. An old |.svn| file should be removed
% when this feature is deactivated to avoid \textit{undefined macro} errors
% caused by macros placed there by this feature.\par
% If the \op{groups} option is enabled the macro
% \csi{svnexternalgroup}\marg{group name} can be used to declare a own group
% which is used for all the external files.  Otherwise they are placed in the
% currently active group.  This macro can be used several times during the
% document where an empty argument means \emph{no group} and a `|*|' means
% \emph{current group}.\par
%
% \DescribeOption{groups}
% Controls the keyword groups feature described in
% section~\ref{sec:group}.
%
% \DescribeOption{filesasgroups}
% Controls the automatic declaration of all input files as
% independent groups so that there keyword information can be typeset inside
% other files. The group name is the file path (`|subdir/file.tex|'). The files
% are still added to the current group defined by \cs{svngroup}.\par This can be
% disabled and re-enabled using \cs{svnfilesasgroupfalse} and
% \cs{svnfilesasgrouptrue} inside the document. See
% section~\ref{sec:filesasgroups} for additional information.
%
% \DescribeOption{graphics}
% This option allows to automatically declare all images included using the
% macro |\includegraphics| from the \pkg{graphics}/\pkg{graphicx} package as external
% files (see section~\ref{sec:external}). The options \op{external} and
% \op{autoload} are activated by this option so that the produced
% |.svx| files are loaded automatically.  An |autoload=false| option after
% \pkg{graphics} will deactivate this, but then an \cs{svnexternal} macro must be
% included in all \LaTeX\ files which should take the image revisions into
% account.\par The \pkg{graphics} package is loaded if this option is active. If
% this package is needed with some special options it should be loaded by the
% \LaTeX\ document before \svnmulti.\par  Please note that this feature needs
% to tie itself into the \pkg{graphics} package and might fail if the internal
% structure of this package changes in future versions.\par
% If the \op{groups} option is enabled the macro
% \csi{svngraphicsgroup}\marg{group name} can be used to declare a own group
% which is used for all the graphic files (also for pgf images, see below).
% Otherwise they are placed in the group specified by \cs{svnexternalgroup}
% which defaults to the currently active group.  This macro can be used several
% times during the document where an empty argument means \emph{no group} and a
% `|*|' means \emph{current group}.\par Some graphics like logos can appear
% frequently in a document. Do not count them as part of each chapter they can
% be ignored using \csi{svnignoregraphic}\marg{file path}. The macro
% \csi{svnconsidergraphic}\marg{file path} disables this again. Such graphics
% can be then included manually using an explicit \cs{svnexternal} macro.
%
% \DescribeOption{pgfimages}
% Identical to like the \op{graphics} option but for the \pkg{pgf} package
% (implemented against the version from 2008/01/15) with the |\pgfuseimage| and
% |\pgfimage| macros.  Please also see the notes about package loading and ties
% mentioned above.
%
% \DescribeOption{autoload}
% Controls automatic loading of corresponding |.svx|
% files at the begin of files included using |\input| or |\import|. This avoids
% the need of putting an \cs{svnexternal} macro in every file just to load the
% |.svx| files created automatically by the \op{graphics} option. The option
% \op{external} is activated by \op{autoload}.
%
% \DescribeOption{table}
% Controls the generation of a table of revisions which can be included using
% the \cs{tableofrevisions} macro. This table shows the
% revisions of all files and groups. This needs \op{groups} to work which is
% activated with \op{table}. Enable \op{filesasgroup} to include a list of all
% files per group. See the section~\ref{sec:table} for more information.
%
% \DescribeOption{<<inputfilehooks>>}
% This is not really an option but an internal switch which loads the \pkg{fink}
% package and installs at-begin-input-file and at-end-input-file hooks which are
% needed for many of the options above. These options enable this switch
% automatically.
%
% \subsection{Including of the Subversion keywords}\label{sec:usage:id}
% To include your Subversion Id keywords use \cs{svnid} or \cs{svnidlong}.
% These macros should be written very early in each file, \ie in the preamble of
% the main document soon after |\documentclass| and |\usepackage{svn-multi}| and
% as first in \emph{every} |\include|d subfile before the |\chapter| macro. They
% do not create any output.  See section~\ref{sec:kwaccess} to learn how to
% typeset the keyword values.
%
% \DescribeMacro{svnid}{\karg{Id}}
% Macro for the svn Id keyword.  Write the macro as |\svnid{$||Id$}| into your
% \LaTeX{} files. A trailing colon with spaces after the |Id| is also valid but
% \textbf{everything else} except a valid Subversion string will cause a \TeX{}
% parse error.  Don't forget to set the subversion property |svn:keywords| of
% the files to at least `|Id|'. Subversion will expand it at the next commit.
% Please note that because the value is read verbatim the macro should exactly
% be written like above. Spaces, newlines or comments between |\svnid| and the
% \{ will lead to \TeX{} parse errors.
%
% \DescribeMacro{svnidlong}{\karg{HeadURL}\karg{LastChangedDate}\ignorespaces
% \karg{LastChangedRevision}\karg{LastChangedBy}}
% Macro for a ``long Id''.  Saves similar values like in `|Id|' but from the
% above four keywords. The usage of \cs{svnid} or \cs{svnidlong} is a matter
% of taste. The second is more readable inside the code and results in a nicer
% date and a full URL, not only the filename. Both can also be used together.
% In this case the \cs{svnid} macro should be come last. Because its revision is
% not higher (but identical) than the revision of the \cs{svnidlong} macro it
% does not override its values. This way both the full time zone from the long
% and the file name from the short id macro can be accessed. Please note that
% all features from the 2.0 version load the \pkg{fink} package which lets you
% typeset the current file name anyway using |\finkbase.\finkext|\footnote{The
% file name in \texttt{\${}Id\$} is always the original Subversion file name
% while the one given by the \pkg{fink} package is the current file name.
% Both could differ if the file got renamed.}.
%
% Write this macro like this (order of arguments not meaningfull)\\[2ex]
% |      \svnidlong|\\
% |      {$||HeadURL$}|\\
% |      {$||LastChangedDate$}|\\
% |      {$||LastChangedRevision$}|\\
% |      {$||LastChangedBy$}|\\[2ex]
% in your files and set the subversion property |svn:keywords| of them
% to\\`|HeadURL LastChangedDate LastChangedRevision LastChangedBy|'.
%
% Please note that the arguments are read verbatim. Special precaution are taken
% to allow spaces, newlines and comments direct after the |\svnidlong| and after
% each of the four arguments, just in case someone need this.  In fact
% everything not inside braces \{ \} is ignored.
%
% \DescribeMacro{svn}{\kmarg{keyword}}
% \DescribeMacro*{svn*}{\kmarg{keyword}}
% This macro let you typeset svn keywords directly. The dollars will be stripped
% and the rest is typeset as normal text. The star version strips also the space
% before the last dollar.  This macro alone was the very first version of
% |svnkw| and is still included for fast and simple keyword typesetting.
%
% \DescribeMacro{svnkwsave}{\kmarg{keyword}}
% This macro lets you include and save any keyword you like. The keyword can be
% already expanded or not (no value and only ``|:|'' or nothing after the key
% name). This macro is also used internally and does not create any output.
% Please note that the argument is read verbatim and that there should be no
% space between the macro and the argument's left brace.
%
% \subsection{Typesetting the keyword values}\label{sec:kwaccess}
% The following macros can be used to typeset the keyword values anywhere in the
% document. Please note that not all \LaTeX{} fonts have all special
% characters, \eg `\_' is not provided in the standard roman font. To proper
% typeset file names and URLs containing these letters you can use either
% teletype font (|\texttt|) or use |{\urlstyle{rm}\svnnolinkurl{...}}| which
% requires the \pkg{hyperref} package.
%
% Like already mentioned \svnmulti knows three scopes of keywords. The first
% contains of the keywords for the complete document which hold the values of
% the most recent committed file and the second contains of the \emph{current}
% or \emph{file local} keywords, \eg the keywords of the current file. Only this
% two are described here while the third scope is described in
% section~\ref{sec:group}.
%
% \DescribeMacro{svnrev}{}
% \DescribeMacro*{svndate}{}
% \DescribeMacro*{svnauthor}{}
% These macros hold the keyword values of the whole document, \ie of the most
% recent revision. They can be used everywhere in every file of the \LaTeX{}
% document, after |\usepackage{svn}| of course. Please see
% section~\ref{sec:date} how to typeset parts of the date.
%
% \DescribeMacro{svnfilerev}{}
% \DescribeMacro*{svnfiledate}{}
% \DescribeMacro*{svnfileauthor}{}
% These macros hold the keyword values of the current \LaTeX{} file, but only if
% it contains a \cs{svnid} or \cs{svnidlong} macro. Otherwise the macros hold
% either zero values or the values of the last file dependent on whether an
% option is enabled which enabled the \pkg{fink} package. Please see
% section~\ref{sec:date} how to typeset parts of the date. See \cs{svnkw} below
% for all other keywords.
%
% \DescribeMacro{svnmainurl}{}
% \DescribeMacro*{svnmainfilename}{}
% The macro \cs{svnmainurl} and \cs{svnmainfilename} hold the URL and the
% filename of the main \LaTeX{file} as long the keywords |HeadURL| or |Id| were
% used in it, respectively.  These can be used to typeset this information
% anywhere in the document which might be more descriptive as the name of the
% current file (which can be typeset with \cs{svnkw}|{HeadURL}| or
% \cs{svnkw}|{Filename}| after \cs{svnid} or \cs{svnidlong}, respectively).
%
% \DescribeMacro{svnsetmainfile}{}
% This will declare the current file as the main LaTeX file by defining the
% above macros. It will automatically be called at the end of the preamble so
% the user normally doesn't have to use it by him- or herself as long it isn't
% needed in the preamble.\par Please note that this macro changes the
% definition of \cs{svnmainurl} and \cs{svnmainfilename} directly without going
% over the auxiliary file. Calling it in several files will make this two macros
% inconsistent.
%
% \DescribeMacro{svnkw}{\marg{keyword name}}
% All keywords saved with \cs{svnid}, \cs{svnidlong} or \cs{svnkwsave} can be
% typeset by this macro which is a holdover from a very early version of this
% package when multiple files where not supported.  It takes one argument which
% must be a subversion keyword name. It then returns the current value of this
% keyword or nothing (|\relax|) when the keyword was not set yet.
% Examples:\\
% \indent\indent |\textsl{Revision: \svnkw{Revision}}|\\
% \indent\indent |URL: \url{\svnkw{HeadURL}}|\\
% In the second example |\url| (\pkg{hyperref} package) is used to add a hyperlink
% and to avoid problems with underscores (|_|) inside the URL.  \svnmulti is
% also providing a macro \cs{svnnolinkurl} which works like |\url| but doesn't
% adds an hyperlink. See the description of this macro for more details.
%
% If the given keyword doesn't exists a package warning is given to allow
% spelling errors to be tracked down. This doesn't work well when \cs{svnkw} is
% used inside |\url|. In this case the warning code will be typeset(!) verbatim
% into the document by |\url|.
%
% \DescribeMacro{svnkwdef}{\marg{keyword name}\marg{value}}
% This macro is used to define the keyword values. This is normally only called
% internally but could be used by the user to override single keywords.  The
% values can then be typeset by \cs{svnkw}.  Note that this macro has no
% influence on the calculation of the latest revision.
%
% Note that for \cs{svnkw} and \cs{svnkwdef} all different names for one keyword
% are valid and result in the access of the same variable. So \eg subversion
% treats |Rev|, |Revision| and |LastChangedRev| the same way and so does this
% macros. You can \eg say |\svnkwdef{Rev}{123}| and then typeset it with
% |\svnkw{Revision}| or |\svnkw{LastChangedRev}| if you like.
%
% \subsubsection{Accessing Date Values}\label{sec:date}
% \begin{tabular}{@{}l@{\hspace{-2\macrosep}}l@{}}\\
% \DescribeMacro*{svnfileyear}{}          &\DescribeMacro*{svnyear}{}\\
% \DescribeMacro*{svnfilemonth}{}         &\DescribeMacro*{svnmonth}{}\\
% \DescribeMacro*{svnfileday}{}           &\DescribeMacro*{svnday}{}\\
% \DescribeMacro*{svnfilehour}{}          &\DescribeMacro*{svnhour}{}\\
% \DescribeMacro*{svnfileminute}{}        &\DescribeMacro*{svnminute}{}\\
% \DescribeMacro*{svnfilesecond}{}        &\DescribeMacro*{svnsecond}{}\\
% \DescribeMacro*{svnfiletimezone}{}      &\DescribeMacro*{svntimezone}{}\\
% \DescribeMacro*{svnfiletimezonehour}{}  &\DescribeMacro*{svntimezonehour}{}\\
% \DescribeMacro*{svnfiletimezoneminute}{}&\DescribeMacro*{svntimezoneminute}{}
% \\
% \end{tabular}
% \\*[\medskipamount]
% Whenever the date information is read, \ie by
% \cs{svnkwsave}|{LastChangedDate}| \cs{svnkwsave}|{Date}|, \cs{svnidlong} or
% \cs{svnid}, the following macros are set to the appropriate date parts for the
% current file (the |\svnfile...| versions) and for the whole document.
%
% Please note that the hour and timezone are dependend on the keyword which
% defines the date information. The hour will be in UTC aka Zulu-time, \ie
% timezone +0000, when the date comes from the |Id| keyword.
% Otherwise the hour and timezone will be in local time.
% To avoid confusion the |Id| and |Date|/|LastChangedDate| keywords, \eg
% \cs{svnid} and \cs{svnidlong}, should not be intermixed and/or the timezone
% should always be typeset together with the time.
%
% Starting with v1.4 of \svnmulti the timezone macros return the full
% timezone, \ie sign, hour and minute part, \eg |+0100|, not only the sign and
% hour. The new macros % \cs{svntimezonehour}/\cs{svnfiletimezonehour} and
% \cs{svntimezoneminute}/\linebreak[3]\cs{svnfiletimezoneminute} can be used to
% access only the hour including sign or the minute part, respectively.
%
% Older versions of this manual assumed the minute part as always |00| and
% suggested to add it manually if needed: |\svnfiletimezone00| or
% |\svntimezone00|.  In order not to ``break'' documents which followed this
% suggestion this two macros now remove a trailing |00| if present.  However,
% this can be a problem when they are used inside an argument of another macro.
% One solution for this is to redefine them without the |00| removal part:\\
% \begingroup\small
% |\renewcommand{\svntimezone}{\svntimezonehour\svntimezoneminute}|\\
% |\renewcommand{\svnfiletimezone}{\svnfiletimezonehour\svnfiletimezoneminute}|
% \endgroup\par
% To revert to the old (pre-v1.4) definition use:\\
% \begingroup\small
% |\renewcommand{\svntimezone}{\svntimezonehour}|\\
% |\renewcommand{\svnfiletimezone}{\svnfiletimezonehour}|
% \endgroup
% \vspace{1ex}
%
% \DescribeMacro{svntime}{}
% \DescribeMacro*{svnfiletime}{}
% \DescribeMacro*{svncgtime}{}
% This macros return the time part of the date only and simply return the
% corresponding hour, minute and second macros with a colon as separator.
% \vspace{2\baselineskip}
%
% \DescribeMacro{svnpdfdate}{}
% Returns the last changed date of the whole document in a format needed for
% |\pdfinfo|. Can be used like this:\\
% \hbox{}\hfill|\pdfinfo{ /CreationDate (D:\svnpdfdate) }|\hfill\hbox{}\\
% to set the PDF creation date to the last changed date if you use |pdflatex| to
% compile your \LaTeX{} document.
%
% \DescribeMacro{svntoday}{}
% \DescribeMacro*{svnfiletoday}{}
% \DescribeMacro*{svncgtoday}{}
% These macros typeset the document-global, current-file or current-group date,
% respectively, using the format of |\today| which depends on the used language.
% To adjust the language of your document use the \pkg{babel} package.
%
% \subsection{Using full author names}
% If you like to have the full author\footnote{This means subversion authors,
% \eg the persons who commit changes into the svn repository.} names, not only
% the usernames, in your document you can use the following macros. First you
% have to register all authors of the document with \cs{svnRegisterAuthor} and
% then you can write \eg |\svnFullAuthor{\svnauthor}| or
% |\svnFullAuthor{\svnfileauthor}|.
%
% \DescribeMacro{svnRegisterAuthor}{\marg{author}\marg{full name}}
% This macro registers \meta{full name} as full name for \meta{author} (a
% subversion username) for later use with \cs{svnFullAuthor}.
%
% \DescribeMacro{svnFullAuthor}{\marg{author name or macro}}
% \DescribeMacro*{svnFullAuthor*}{\marg{author name or macro}}
% Takes the username as argument and returns the full name if it was registered
% first with \cs{svnRegisterAuthor}, otherwise it returns the given username.
% The star version returns the username in parentheses after the full name.
% This is normally used in one of the following forms:\\
% \hspace*{3em}\cs{svnFullAuthor}\{\cs{svnauthor}\}\\
% \hspace*{3em}\cs{svnFullAuthor}\{\cs{svnfileauthor}\}\\
% \hspace*{3em}\cs{svnFullAuthor}\{\cs{svncgauthor}\}
%
% \subsection{Using full revision names}
% Like the author's also revision names/tags can be registered and used later.
% These macros were implemented on user request and have the drawback that you
% have to guess the next revision number of your document in order to get
% correct results when you like to tag the to-be-checked-in revision.  Please
% note that this has nothing to do with the normal subversion tagging.
%
% \DescribeMacro{svnRegisterRevision}{\marg{revision number}\marg{tag name}}
% This registers \meta{tag name} as tag name for \meta{revision number} for
% later use with \cs{svnFullRevision}.
%
% \DescribeMacro{svnFullRevision}{\marg{revision number or macro}}
% \DescribeMacro*{svnFullRevision*}{\marg{revision number or macro}}
% Takes a revision number coming from a macro like \cs{svnrev}, \cs{svnfilerev}
% or a number as argument and returns the full name if it was registered first
% with \cs{svnRegisterRevision}, otherwise it returns ``Revision \meta{revision
% number}''.  The star version returns also the revision number leaded by `r' in
% parentheses after the tag name, \eg |Name (r123)|.
%
% \subsection{Verbatim URLs with and without hyperlinks}
% \vspace{-\baselineskip}
% \DescribeMacro{svnnolinkurl}{\marg{macro with returns special text}}
% This macro allows you to write |\svnnolinkurl{\svnkw{HeadURL}}| and get the
% Head URL typeset verbatim. However |\url{|\cs{svnkw}|{HeadURL}}|
% (\pkg{hyperref} package) gives you the same result with hyperlinked. Both
% macros require the \pkg{hyperref} package which is not automatically loaded by
% \svnmulti.  Please load it manually when you like to use \cs{svnnolinkurl}.
%
% Since v1.3 all keywords are read and typeset verbatim so this macro isn't this
% important anymore. However together with \pkg{hyperref}'s |\urlstyle| macro it
% can be used to have keyword values with special characters in roman font,
% which normally doesn't hold letters like `\_'.
%
% Please note that you can't use \pkg{hyperref}s |\nolinkurl| because it won't
% expand \cs{svnkw}.
%
% \subsection{Groups}\label{sec:group}
% Starting with v2.0 files can be grouped together and the keyword values of the
% latest revision of a group can be accessed. Use the \op{groups} option to
% activate these macros.
%
% \DescribeMacro{svngroup}{\marg{group name}}
% This macro declares all following files (actually only following \cs{svnid},
% \cs{svnidlong} and \cs{svnkwsave} macros) until the next \cs{svngroup} as part
% of the given keyword group. It can be placed inside the main file before some
% |\include|/|\input| macros or inside sub-files before the id macros, \ie
% direct at the start of the file. Please note that the group will then swap
% over to the following file. However, the group can be closed manually using an
% `|*|' as argument (|\svngroup{*}|), \eg at the end of a file.\par The changes
% done by this macro are \TeX\ global, \ie there can't be caught using \TeX\
% groups (|{ }|).
%
% \smallskip
% \DescribeMacro{thesvngroup}{}
% Returns the name of the current keyword group.
%
% \medskip\noindent
% \begin{tabular}{@{}l@{\hspace{-2\macrosep}}l@{}}\\
% \DescribeMacro*{svncgrev}{}    & \DescribeMacro*{svncghour}{}           \\
% \DescribeMacro*{svncgauthor}{} & \DescribeMacro*{svncgminute}{}         \\
% \DescribeMacro*{svncgdate}{}   & \DescribeMacro*{svncgsecond}{}         \\
% \DescribeMacro*{svncgyear}{}   & \DescribeMacro*{svncgtimezone}{}       \\
% \DescribeMacro*{svncgmonth}{}  & \DescribeMacro*{svncgtimezonehour}{}   \\
% \DescribeMacro*{svncgday}{}    & \DescribeMacro*{svncgtimezoneminute}{} \\
% \end{tabular}
% \\*[\medskipamount]
% These macros return keyword values of the currently selected keyword group,
% like the \cs{svnrev}, \cs{svnfilerev}, etc., macros described in
% sections~\ref{sec:kwaccess} do for the whole document and the current file,
% respectively.
% In order to hold them robust, which is important to use them in macros like
% \cs{svnFullAuthor} they do not provide any arguments to select other groups
% than the current one. To access keyword values of other groups use the general
% macro \cs{svng}\marg{group name}\marg{key} or change the locally selected
% keyword group using the \cs{svnsetcg}.\par
%
% \smallskip\noindent
% \DescribeMacro{svnsetcg}{\marg{group name}}
% Normally the |\svncgXXX| macros mentioned above use the last keyword
% group defined by \cs{svngroup}, but this can be changed using the
% \cs{svnsetcg} macro.\par The idea is that the currently selected group can be
% changed locally to the current \TeX\ group for the keyword output macros
% |\svncgXXX| only while the group for the keyword input macros like
% \cs{svnid} is unaffected.\par
% To reset the used group to the last one defined by \cs{svngroup} simply
% use \cs{svnsetcg} with an `|*|' as argument.\par
% \paragraph*{Example 1:} |{\svnsetcg{abc}\svnFullAuthor{\svncgauthor}}|\\ would
% output the full author's name of group \textit{abc}.\par
% \paragraph*{Example 2:} To typeset the three keyword values of group
% \textit{abc} somewhere outside this group use:\\
% |{\svnsetcg{abc}Rev: \svncgrev\\||Date: \svncgdate\\|\\{}
% |Author: \svncgauthor\\}|
% \paragraph*{Example 3:} To typeset the date of group \textit{abc} outside of
% this group in the format of |\today| use: |{\svnsetcg{abc}\svncgtoday}|
% \smallskip
%
% \DescribeMacro{thesvncg}{}
% Returns the name of the current group selected by \cs{svnsetcg}.
%
% \DescribeMacro{svng}{\marg{group name}\marg{key}}
% This macro is a general form of the |\svncgXXX| macro mentioned above.
% The first argument is the requested keyword group, the second one the
% requested keyword in the form of |rev|, |date|, |author|, |year|, etc.. Please
% note that this macro can not be used inside macros like \cs{svnFullAuthor}.
%
% \subsubsection{Files as Own Groups}\label{sec:filesasgroups}
% The group feature could be used to access the version control information of
% single files anywhere in the document when these are defined as own groups for
% themselves. Because a file can only be in one group this would not be compatible
% with the normal usage of the group feature. Therefore a special feature was
% introduced to automatically or manually define a file as \emph{extra} group
% for itself which doesn't influence its membership in a real group. This means
% that a set of (internal) group macros is produced for the file like for a
% normal group but neither is the current group changed nor is it included in
% the (internal) list of groups. The file will still be part of the current
% group.\par
% \paragraph{Declaration:}
% This feature is enabled by the \op{filesasgroups} option. All files of the
% document are then automatically declared as extra groups. This can be disabled
% for parts of the document using \csi{svnfilesasgroupsfalse} and re-enabled
% using \cs{svnfilesasgroupstrue} macros. The current file can be manually
% declared as extra group with the \cs{svnfileasgroup} macro.\par
% \paragraph{Exclude/Consider files extensions:}
% The above mentioned automatically group declaration uses an hook which is
% triggered every time another file is read by the document. This unfortunately
% includes other packages, some auxiliary files and font, config and other files
% read in by this packages. An internal filter is in place to ignore this files
% by their file extension. This filter can by modified by the two following
% macros.
% \DescribeMacro{svnignoreextensions}{\marg{comma separated list of extension without
% leading dot}}
% Tells \svnmulti to ignore the following file extension and never declare
% files with them as extra groups.
% \DescribeMacro{svnconsiderextensions}{\marg{comma separated list of extension without
% leading dot}}
% Tells \svnmulti to (re-)consider the following file extension and declare
% files with them as extra groups if read in.
%
% \paragraph{Typesetting:}
% The keyword information of the files can be typeset with the normal group
% typeset macro where the filename (including extension) is the group name.\\
% \emph{Example:} |\svnsetcg{some_file.tex}\svncgauthor| would typeset the
% author of the file |some_file.tex|
%
% \subsection{Table of Revisions}\label{sec:table}
% Version 2.0 introduces this new feature which is not yet fully documented and
% might change a little in future versions.
%
% \DescribeMacro{tableofrevisions}{}
% If the \op{table} is enabled a table of revision is written into a file called
% \meta{main \LaTeX\ file}|.svt| (|t| for table) which can be included using the
% \cs{tableofrevisions} macro. The table contains the revision information of
% the complete document and of all defined groups.
% If the \op{filesasgroups} option is set the table will also include the single
% files sorted by group.\par
% The |.svt| file is in an general format which uses a lot of macros to format
% the table and the different cells. This macros are called |\svntab...| or
% similar and can be redefined to produce a different look.  Please see the
% macro implementation in section~\ref{sec:impl:table}, the package example
% files and/or the |.svt| file itself to learn which macros exists and how to
% redefine them.\par
% \textbf{Note:} The generic table is wrapped in
% \csi{svntable}/\csi{endsvntable} macros which are by default
% defined to use a |tabular| table environment. This macros can (in theory) be
% redefined to use other table environments. However, \TeX\ processes the tables
% in quite a special way (which is the reason why an auxiliary file is used in
% the first place), so most alternative table packages like \pkg{tabularx} seem to
% read the content in an verbatim like way which doesn't work (yet) with the
% current format of the |.svt| file. Future versions of the package might produce
% a pre-processed version of the |.svt| file on user request containing \eg
% a \pkg{tabularx} table which simply has to be read using |\input| (this is done
% inside the \cs{tableofrevisions} macro).
%
% \subsection{Including Keywords of External Files}\label{sec:external}
% Version 2.0 now supports the inclusion of keywords of external files using the
% following macros and the Perl script \scr{svn-multi.pl}.
%
% \DescribeMacro{svnexternal}{\oarg{group name}\{\{\meta{filea}\}\ignorespaces
% \{\meta{fileb}\}\ldots\{\meta{filex}\}\}}
% Subversion keywords of external files (\eg non-\LaTeX\ files like images or
% even directories) can be included using this macro which awaits a list of
% files, each with the full path relative to the main \LaTeX\ file and each
% enclosed by |{ }|. The files must be under version control by Subversion, of
% course. Use \cs{svnexternalpath} to specify paths to be scanned for this
% files if they are not located relative to the main file.  The requested
% filenames are written into the |.svn| auxiliary file and then processed by the
% external script \scr{svn-multi.pl} which must be executed like described
% below. The appropriate keywords are then written in \meta{source file}|.svx|
% files (|x| like eXternal) which are read in by the same \cs{svnexternal} macro
% at the next \LaTeX\ run. If a group name of `|*|' for the external files is
% specified using the optional argument the keywords will be placed in the same
% keyword group as this macro. The file local macros like \cs{svnfilerev} which
% appear in a source file after \cs{svnexternal} are affected, \ie updated if
% one of the external revision is higher than the one of the source file. This
% makes sense if \eg the included graphics are taken as logical part of a source
% file.
%
% \DescribeMacro{svnexternalpath}{\{\{\meta{patha}\}\ignorespaces
% \{\meta{fileb}\}\ldots\{\meta{pathx}\}\}}
% This macro can be used in the document preamble to declare a set of paths to
% be scanned for files specified with \cs{svnexternal}. This avoids the need to
% provide the path again and again for every file.
% The paths need to be enclosed in |{ }| and must be in Unix style, \ie with
% `|/|' as directory separator and should end with a `|/|'. Windows users should
% just replace all `|\|' with `|/|', \eg `|C:\My dir|' gets `|{C:/My dir/}|'.
%
% \subsection*{Script \hypertarget{script@svn-multi.pl}{\texttt{svn-multi.pl}}}
% The file \scr{svn-multi.pl} which comes with the \svnmulti package is an
% external Perl script which has to be run in the command line or by a \LaTeX\
% development environment/editor like other tools like |Bib|\TeX\ or
% |Makeindex|. A Perl interpreter and a Subversion command line client (|svn|)
% must be installed to execute this script. Both are available for free for all
% major operating systems.\par
% The script should be run inside the document folder in the following order:
% \begin{enumerate}
%  \item Compile \LaTeX\ document, |.svn| file is generated.
%  \item Run \scr{svn-multi.pl} script, |.svx| files are generated.
%  \item Compile \LaTeX\ document, |.svx| files are read in.
% \end{enumerate}
%
% The script can be used with three different sets of arguments and with
% any combination of them. Please note that the word |jobname| stands for
% the main \LaTeX\ file name without the |.tex| extension.
%
% \DescribeScript{svn-multi.pl}{\meta{jobname}}
% As already mentioned in the \cs{svnexternal} description above this script
% reads the requested external filename from the \meta{jobname}|.svn| file. The
% Subversion command line client |svn| is then used to fetch the needed keywords
% which are placed in a \cs{svnidlong} macro inside a \meta{source file}|.svx|
% file. Every single source file which uses \cs{svnexternal} will become its own
% |.svx| file which allows to attach specific external files to one (or more)
% specific source files.
%
% \DescribeScript{svn-multi.pl}{\meta{jobname} [-{}-group \meta{group name}]
% \meta{file(s)} \ldots}
% The second way to use \scr{svn-multi.pl} is to call it with a list of external
% files. A keyword group can be specified using the |--group| \meta{group name}
% option which can placed any number of times between the file names. The group
% is used for all external files listed after the option until the next group is
% specified. All keywords of these files are written in the \meta{jobname}|.svx|
% file and read in by the main \LaTeX\ file if a, possible empty,
% \cs{svnexternal} macro is included. This allows for easy including of many
% external files without specifying them all inside the source file. For example
% \scr{svn-multi.pl}| */*.jpg| (under Linux/Unix) will include the keywords of
% all JPG files in all subdirectories.\par
% It is also possible to do this with a sub-(\LaTeX)-file by calling the script
% on it: \scr{svn-multi.pl}~\mbox{\meta{sub file} \meta{external files for sub
% file}}, which will create/overwrite the \meta{sub file}|.svx| file.  However
% the files given by \cs{svnexternal} in this sub-file will not be honoured in
% this case.
%
% \DescribeScript{svn-multi.pl}{\meta{jobname} [-{}-group \meta{group name}]
% -{}-fls}
% Instead of providing a list of all non-\LaTeX/external files the |--fls|
% option can be used to read this list from the \meta{jobname}|.fls| file. This
% file is produced by the \LaTeX\ compiler when run with the |--recorder| option
% and contains a list of all input and output files. Only input files with a
% relative path are used. A corresponding keyword group can also be specified.
%
% \section{Known Issues}\label{sec:issues}
% This section lists some known issues of the \svnmulti package and tries to
% provide some workaround. Please feel free to write \svnmulti author if you
% detect any side effects or other issues causes by this package.
%
% \subsection{Packet \texttt{listings} uses \textbackslash input}
% If a file \meta{basename}.\meta{extension} is typeset verbatim using
% |\lstinputlisting|, which uses |\input| to read the file, an existing
% \meta{basename}|.svx| file is also included as part of the listing. This can
% be avoided by code like this:
% \begin{verbatim}
%  {\makeatletter\let\input\@input
%  \lstinputlisting[options]{filename}
%  }
% \end{verbatim}
%
% \section{Package Dependencies and Acknowledgements}\label{sec:depack}
% This package uses some features from other packages and/or patches some macros
% of them to provide additional related features. This section is used to list
% this packages, their internal macro which got used and acknowledge the
% authors/maintainers of them. Please send error reports to the author of
% \svnmulti and not to the people listed below.\par
% All packages (including \svnmulti) stand under the \LaTeX\ Project Public
% Licence (LPPL) which can be found at
% \hbox{\url{http://www.latex-project.org/lppl/}} and can be freely downloaded
% from the Comprehensive TeX Archive Network (CTAN) at
% \hbox{\url{http://www.ctan.org/}}.
%
% \newenvironment*{DepPackage}[1]{\ignorespaces
% \subsection*{\pkg{#1}}%
% \def\thedeppackage{#1}%
% \def\infoline##1{\par\smallskip\textbf{##1}\hspace{1em}}%
% }
% {\infoline{Location:} CTAN: \url%
% {http://tug.ctan.org/pkg/\thedeppackage}}
%
% \begin{DepPackage}{hyperref}
% The macro \cs{svnnolinkurl} is resembling the \pkg{hyperref} macro
% |\nolinkurl| and uses some its internal macros from the |\url| macro
% definition.
% \infoline{Used internal macros:} |\hyper@normalise|, |\Hurl|
% \infoline{Version used:} 2008/11/18 v6.78m
% \infoline{Licence:} LPPL, any version
% \infoline{Authors/Maintainers:} Sebastian Rahtz, Heiko Oberdiek
% \end{DepPackage}
%
% \begin{DepPackage}{fink}
% The FiNK (File Name Keeper) package is used to get the file name and path of
% the input files. Two macros are patched to install hooks to execute own macros
% before and after each input file. This is only done if needed for an enabled
% option and therefore can be disabled using the option \op{old}.
% \infoline{Used internal macros:} |\fink@file|, |\fink@nextdir|,
% |\fink@nextext|, |\fink@nextbase|
% \infoline{Patched internal macros:} |\fink@prepare|, |\fink@restore|
% \infoline{Version used:} 2008/02/27 v2.1.1
% \infoline{Licence:} LPPL, any version
% \infoline{Author/Maintainer:} Didier Verna
% \end{DepPackage}
%
% \begin{DepPackage}{graphics}
% If the \op{graphics} option is enabled the following macro is patched to
% record the file name and path of the included graphic.
% \infoline{Patched internal macros:} |\Gin@setfile|
% \infoline{Version used:} 2006/02/20 v1.0o
% \infoline{Licence:} LPPL, any version
% \infoline{Author/Maintainer:} David Carlisle, \LaTeX3 Project
% \end{DepPackage}
%
% \begin{DepPackage}{pgf}
% Like the \pkg{graphics} package above a macro of this package is pathed to
% record the file names and paths of included images when the option
% \op{pgfimages} is enabled. Because this images pre-declares images for later
% use the internal declared `image macros' are patched as well.
% \infoline{Patched internal macros:} |\pgf@declareimage|,
% \hbox{|\pgf@image@|\meta{image name}|!|}
% \infoline{Used internal macros:} |\pgf@filename|, |\pgf@image|
% \infoline{Version used:} 2008/01/15 v2.00
% \infoline{Licence:} LPPL v1.3c and GPL v2
% \infoline{Author\&Maintainer:} Till Tantau
% \end{DepPackage}
%
% \begin{DepPackage}{latex}
% Parts of the macro definitions of the |\tableofcontents| macros from the
% |article| and |book| class of standard \LaTeX\ were used to define a similar
% \csi{tableofrevisions} macro for both this classes and other similar classes.
% \infoline{Version used:} 2005/09/16 v1.4f
% \infoline{Licence:} LPPL v1.3c
% \infoline{Authors/Maintainers:} \LaTeX3 Project
% \end{DepPackage}
%
%
% \section{Further Reading}
% The \textsf{svn-multi} package (in version 1.3) and its usage got discussed in
% the following articles:
%
% \begin{itemize}
%  \item[{[1]}] Martin Scharrer, ``Version Control of LaTeX Documents with
%  svn-multi'', The Prac\TeX\ Journal, (3), 2007.
%  URL: \url{http://www.tug.org/pracjourn/2007-3/scharrer/}
%  \item[{[2]}] Mark Eli Kalderon, ``LaTeX and Subversion'',
%  The Prac\TeX\ Journal, (3), 2007.
%  URL: \url{http://www.tug.org/pracjourn/2007-3/kalderon-svnmulti/}
%  \item[{[3]}] Uwe Ziegenhagen , ``LaTeX Document Management with Subversion'',
%  The Prac\TeX\ Journal, (3), 2007.
%  URL: \url{http://www.tug.org/pracjourn/2007-3/ziegenhagen/}
% \end{itemize}
%
% \StopEventually{}
% %%%%%%%%%%%%%%%%%%%%%%%%%%%%%%%%%%%%%%%%%%%%%%%%%%%%%%%%%%%%%%%%%%%%%%%%%%%%
% \section{Implementation}
% \subsection{Package Header}
% \subsubsection*{Package Identification}
%    \begin{macrocode}
\NeedsTeXFormat{LaTeX2e}[1999/12/01]
\ProvidesPackage{svn-multi}
 [\filedate\space\fileversion\space SVN Keywords for multi-file LaTeX documents]
%    \end{macrocode}

% \subsubsection*{Options}
% Declaration of options and internal switches.
%    \begin{macrocode}
\RequirePackage{kvoptions}

\SetupKeyvalOptions{%
  family = svn-multi,
  prefix = @svnmulti@
}
\newif\if@svnmulti@needinputfilename
\newif\if@svnmulti@anygraphic
\newif\if@svnmulti@autoload

\DeclareVoidOption{old}{%
  \@svnmulti@verbatimtrue
  \@svnmulti@groupsfalse
  \@svnmulti@externalfalse
  \@svnmulti@graphicsfalse
  \@svnmulti@pgfimagesfalse
  \@svnmulti@autoloadfalse
  \@svnmulti@tablefalse
  \@svnmulti@needinputfilenamefalse
  \@svnmulti@filesasgroupsfalse
}
\DeclareVoidOption{all}{%
  \@svnmulti@verbatimtrue
  \@svnmulti@groupstrue
  \@svnmulti@externaltrue
  \@svnmulti@graphicstrue
  \@svnmulti@pgfimagestrue
  \@svnmulti@autoloadtrue
  \@svnmulti@tabletrue
  \@svnmulti@needinputfilenametrue
  \@svnmulti@filesasgroupstrue
}
\DeclareBoolOption[true]{verbatim}
\DeclareBoolOption[false]{groups}
\DeclareBoolOption[false]{external}
\DeclareBoolOption[false]{filesasgroups}
\DeclareBoolOption[false]{graphics}
\DeclareBoolOption[false]{pgfimages}
\DeclareStringOption{autoload}[true]
\DeclareBoolOption[false]{table}

\ExecuteOptions{old}
\ProcessKeyvalOptions{svn-multi}
%    \end{macrocode}
%
% Enable dependent options:
%    \begin{macrocode}
\def\svn@depoption#1{%
  \csname if@svnmulti@#1\endcsname\else
  \message{svn-multi: Required option '#1' enabled.}%
  \csname @svnmulti@#1true\endcsname
  \fi
}

\if@svnmulti@groups
  \@svnmulti@needinputfilenametrue
\fi
\if@svnmulti@external
  \@svnmulti@needinputfilenametrue
\fi
\if@svnmulti@filesasgroups
  \svn@depoption{groups}
  \@svnmulti@needinputfilenametrue
\fi
\if@svnmulti@graphics
  \svn@depoption{external}
  \svn@depoption{autoload}
  \@svnmulti@needinputfilenametrue
\fi
\if@svnmulti@pgfimages
  \svn@depoption{external}
  \svn@depoption{autoload}
  \@svnmulti@needinputfilenametrue
\fi
\if@svnmulti@autoload
  \svn@depoption{external}
  \@svnmulti@needinputfilenametrue
\fi
\if@svnmulti@table
  \svn@depoption{groups}
  \@svnmulti@needinputfilenametrue
\fi
%    \end{macrocode}
%
% Check if \op{autoload} was set explicitly and obey the value.
%    \begin{macrocode}
\ifx\@svnmulti@autoload\@undefined
\else
\ifx\@svnmulti@autoload\empty
\else
\def\svn@temp{true}
\ifx\@svnmulti@autoload\svn@temp
  \@svnmulti@autoloadtrue
  \svn@depoption{external}
  \@svnmulti@needinputfilenametrue
\else
\def\svn@temp{false}
\ifx\@svnmulti@autoload\svn@temp
  \if@svnmulti@autoload
  \PackageWarning{svn-multi}{Option 'autoload' disabled.}
  \fi
  \@svnmulti@autoloadfalse
\else
  \PackageError{svn-multi}%
    {Invalid value for 'autoload' option: '\@svnmulti@autoload'^^J%
     ! Only 'true','false' or empty (='true') are allowed!}
\fi\fi\fi\fi
%    \end{macrocode}

% General switch if any graphic option is enabled:
%    \begin{macrocode}
\if@svnmulti@graphics
  \@svnmulti@anygraphictrue
\fi
\if@svnmulti@pgfimages
  \@svnmulti@anygraphictrue
\fi
%    \end{macrocode}
%

% \subsection{General Internal Macros}
% Some internal used macro which don't fit in any other section.
%
% \begin{macro}{\svn@ifempty}[1]{string}
% Tests if the given argument is empty. If so the first of the next two token
% will be expanded, the second one otherwise.
%    \begin{macrocode}
\def\svn@ifempty#1{%
  \begingroup
  \def\svn@temp{#1}%
  \ifx\svn@temp\empty
    \endgroup
    \expandafter
    \@firstoftwo
  \else
    \endgroup
    \expandafter
    \@secondoftwo
  \fi
}
%    \end{macrocode}
% \end{macro}
%
% \begin{macro}{\svn@ifequal}[2]{string a}{string b}
% Tests if the given arguments are identical, \eg same strings. If so the first
% of the next two token will be expanded, the second one otherwise.
%    \begin{macrocode}
\def\svn@ifequal#1#2{%
  \begingroup
  \def\svn@stringa{#1}%
  \def\svn@stringb{#2}%
  \ifx\svn@stringa\svn@stringb
    \endgroup
    \expandafter
    \@firstoftwo
  \else
    \endgroup
    \expandafter
    \@secondoftwo
  \fi
}
%    \end{macrocode}
% \end{macro}
%
% \begin{macro}{\svn@input}[1]{file name/path}
% Macro to load |.svx| and |.svt| files.  The current keyword group is saved
% away and restored after the |.svx| file is loaded.  The macros |\IfFileExists|
% with |\@@input| are used because |\InputIfFileExists| got redefined by the
% \pkg{fink} package and there is no need to use \pkg{fink} for this files.
%    \begin{macrocode}
\def\svn@input#1{%
  \begingroup
    \let\svn@temp\svn@g
    \IfFileExists{#1}{\@@input #1\relax}{}
    \global\let\svn@g\svn@temp
  \endgroup
}
%    \end{macrocode}
% \end{macro}


% \subsection{Definition of init values}
% Initialisation of at least the revision to a numeric value is necessary to not
% break the |\ifnum| tests later in this package. The revision is initialised to
% -1, but will be set to 0 if an unexpanded |$||Rev:$| keyword is read. This way
% it can be tested if a file had any keyword macros or not.\par
% Note that there a two different macros for the document global keywords:\par
% The user level |\svn|\meta{kw} macros hold the global value and are only valid
% after a \LaTeX\ run. They are initialised here and defined in the |.svn|
% file which is read at the end of the package if it exists and written at the
% end of the document.\par
% The internal macros |\@svn@|\meta{kw} store the oldest (i.e. highest revision)
% keywords read so far from the \cs{svnid} and \cs{svnidlong} macros. They
% change during the document and are used to produce the values of the
% |\svn|\meta{kw} macros when the |.svn| file is written.\par
% Group wide macros are initialised when the group is first defined and have
% three different macros: |\svng@|\meta{group}|@|\meta{kw} (defined in |.svn|),
% |\@svng@|\meta{group}|@|\meta{kw} (accumulator) and also an access
% macro |\svncg|\meta{group} which uses
% |\svn@g|\meta{current group}|@|\meta{kw}.
%    \begin{macrocode}
% Init values
\def\svnrev{-1}             \def\@svn@rev{-1}
\def\svndate{}              \def\@svn@date{}
\def\svnauthor{}            \def\@svn@author{}
\def\svnyear{0000}          \def\@svn@year{0000}
\def\svnmonth{00}           \def\@svn@month{00}
\def\svnday{00}             \def\@svn@day{00}
\def\svnhour{00}            \def\@svn@hour{00}
\def\svnminute{00}          \def\@svn@minute{00}
\def\svnsecond{00}          \def\@svn@second{00}
\def\svntimezonehour{+00}   \def\@svn@timezonehour{+00}
\def\svntimezoneminute{00}  \def\@svn@timezoneminute{00}
\def\svnmainurl{NOT SET}    \def\svnmainfilename{NOT SET}
\def\svnurl{} \def\svnfname{}
\def\svn@temp{}

\if@svnmulti@groups
\def\svn@g{} \def\svn@cg{\svn@g}
\def\@svng@@files{}
\fi

\def\svn@initfile{%
  \gdef\svnfilerev{-1}%
  \gdef\svnfiledate{}%
  \gdef\svnfileauthor{}%
  \gdef\svnfileyear{0000}%
  \gdef\svnfilemonth{00}%
  \gdef\svnfileday{00}%
  \gdef\svnfilehour{00}%
  \gdef\svnfileminute{00}%
  \gdef\svnfilesecond{00}%
  \gdef\svnfiletimezonehour{+00}%
  \gdef\svnfiletimezoneminute{00}%
  \gdef\svnfileurl{}%
  \gdef\svnfilefname{}%
}
\svn@initfile
%    \end{macrocode}
%
% \subsection{Timezone macros}
% \begin{macro}{\svntimezone}
% \begin{macro}{\svnfiletimezone}
% \begin{macro}{\svncgtimezone}
% These macros return the global, file-local and current group time zones,
% respectively. Since v1.4 the minute part is returned as well and the macro
% removes manually added |00| after it to support older documents.
% \changes{v1.4}{2009/02/27}{Return now full timezone (hour + minute part).
% Manually added 00 minutes are removed.}
%    \begin{macrocode}
\def\svntimezone{\svntimezonehour\svntimezoneminute\svn@gobblezeros}
\def\svnfiletimezone{\svnfiletimezonehour\svnfiletimezoneminute\svn@gobblezeros}
\def\svncgtimezone{\svncgtimezonehour\svncgtimezoneminute}
%    \end{macrocode}
% \end{macro}
% \end{macro}
% \end{macro}

% \begin{macro}{\svn@gobblezeros}
% \begin{macro}{\svn@gobblezeros@}
% This two cascaded macros remove a trailing |00| and are used by
% \cs{svnfiletimezone} and \cs{svntimezone}.
%    \begin{macrocode}
\def\svn@gobblezeros{%
  \futurelet\svn@nextchar\svn@gobblezeros@
}
\def\svn@gobblezeros@{%
  \let\@tempa=\relax
  \def\@tempb{0}%
  \ifx0\svn@nextchar
    \let\@tempa=\@gobbletwo
  \fi
  \@tempa
}
%    \end{macrocode}
% \end{macro}
% \end{macro}

% \begin{macro}{\svntime}
% \begin{macro}{\svnfiletime}
% \begin{macro}{\svncgtime}
% This macros simple use the hour, minute and second macros.
%    \begin{macrocode}
\def\svntime{\svnhour:\svnminute:\svnsecond}
\def\svnfiletime{\svnfilehour:\svnfileminute:\svnfilesecond}
\def\svncgtime{\svncghour:\svncgminute:\svncgsecond}
%    \end{macrocode}
% \end{macro}
% \end{macro}
% \end{macro}

% \subsection{\textit{Today} macros}
% These macros use the |\today| macro to typeset the current date using the
% local language settings. Thanks and credit goes to Manuel P\'egouri\'e-Gonnard
% for suggesting this feature and for providing the code.
% \begin{macro}{\svntoday}
%    \begin{macrocode}
\newcommand*{\svntoday}{%
  \begingroup
    \year\svnyear \month\svnmonth \day\svnday
    \relax \today
  \endgroup
}
%    \end{macrocode}
% \end{macro}
%
% \begin{macro}{\svnfiletoday}
%    \begin{macrocode}
\newcommand*{\svnfiletoday}{%
  \begingroup
    \year\svnfileyear \month\svnfilemonth \day\svnfileday
    \relax \today
  \endgroup
}
%    \end{macrocode}
% \end{macro}
%
% \begin{macro}{\svncgtoday}
%    \begin{macrocode}
\newcommand*{\svncgtoday}{%
  \@ifundefined{svng@\svn@cg @year}{??}{%
    \begingroup
      \year\svncgyear \month\svncgmonth \day\svncgday
      \relax \today
    \endgroup
  }%
}%
%    \end{macrocode}
% \end{macro}

% \subsection{Id macros}
% \subsubsection{Normal Id}
% \begin{macro}{\svnid}
% Calls \cs{svnkwsave} with |\@svnidswtrue| so that the Id keyword will be
% parsed at the end of \cs{svnkwsave}.
%    \begin{macrocode}
\newcommand*{\svnid}{%
  \@svnidswtrue
  \svnkwsave
}
\newif\if@svnidsw
\@svnidswfalse
%    \end{macrocode}
% \end{macro}
%

% \begin{macro}{\svn@scanId}[5]{file name}{revision}{date (YYYY-MM-DD)}{time
% (HH:MM:SSZ)}{author (username)}
% Scans svn Id (after it got parsed by \cs{svnkwsave}).  Awaits only Id value
% without leading `|Id:|' and a trailing |\relax| as end marker.  It calls
% \cs{@svn@scandate} to extract the date information and \cs{@svn@updateid} to
% update global Id values and also sets the appropriate keywords.
%    \begin{macrocode}
\def\svn@scanId#1 #2 #3 #4 #5\relax{%
  \@svn@scandate{#3 #4}%
  \@svn@updateid{#2}{#3 #4}{#5}{#1}%
  \svnkwdef{Filename}{#1}%
  \svnkwdef{Date}{#3 #4}%
  \svnkwdef{Revision}{#2}%
  \svnkwdef{Author}{#5}%
}
%    \end{macrocode}
% \end{macro}
%

% \begin{macro}{\@svn@updateid}[4]{rev}{date}{author (username)}{url}
% We first define the expanded arguments to variables for the user.  The
% expansion is needed because the arguments content is mostly generic like
% |\svn@value| which can change very soon after this macro.
%    \begin{macrocode}
\def\@svn@updateid#1#2#3#4{%
  \xdef\svnfilerev{#1}%
  \xdef\svnfiledate{#2}%
  \xdef\svnfileauthor{#3}%
  \xdef\svnfileurl{#4}%
  \svn@getfilename\svnfileurl%
%    \end{macrocode}
% Then we check if the revision is non-empty (not yet expanded by subversion?)
% and larger then the current maximum value |\@svn@rev|.  If yes we save all
% value to save them in the .svn-file later.
%    \begin{macrocode}
  \ifx\svnfilerev\empty\else
    \ifnum\@svn@rev<\svnfilerev
      \xdef\@svn@rev{\svnfilerev}%
      \xdef\@svn@date{\svnfiledate}%
      \xdef\@svn@author{\svnfileauthor}%
      \xdef\@svn@year{\svnfileyear}%
      \xdef\@svn@month{\svnfilemonth}%
      \xdef\@svn@day{\svnfileday}%
      \xdef\@svn@hour{\svnfilehour}%
      \xdef\@svn@minute{\svnfileminute}%
      \xdef\@svn@second{\svnfilesecond}%
      \xdef\@svn@timezonehour{\svnfiletimezonehour}%
      \xdef\@svn@timezoneminute{\svnfiletimezoneminute}%
      \xdef\@svn@url{\svnfileurl}%
      \xdef\@svn@fname{\svnfilefname}%
    \fi

    \if@svnmulti@groups
      \ifx\svn@g\empty\else
        \expandafter
        \ifnum\csname @svng@\svn@g @rev\endcsname<\svnfilerev
          \@svncg@save{rev}{\svnfilerev}%
          \@svncg@save{date}{\svnfiledate}%
          \@svncg@save{author}{\svnfileauthor}%
          \@svncg@save{year}{\svnfileyear}%
          \@svncg@save{month}{\svnfilemonth}%
          \@svncg@save{day}{\svnfileday}%
          \@svncg@save{hour}{\svnfilehour}%
          \@svncg@save{minute}{\svnfileminute}%
          \@svncg@save{second}{\svnfilesecond}%
          \@svncg@save{timezonehour}{\svnfiletimezonehour}%
          \@svncg@save{timezoneminute}{\svnfiletimezoneminute}%
          \@svncg@save{url}{\svnfileurl}%
          \@svncg@save{fname}{\svnfilefname}%
        \fi
      \fi
    \fi
  \fi
}

\def\@svncg@save#1#2{%
  \expandafter\xdef\csname @svng@\svn@g @#1\endcsname{#2}%
}

%    \end{macrocode}
% \end{macro}
%

% \subsubsection{Long Id}
% \begin{macro}{\svnidlong}
% We clear the keyword value first to reduce the risk though bad user input.
%    \begin{macrocode}
\newcommand{\svnidlong}{%
  \svnkwdef{HeadURL}{}%
  \svnkwdef{LastChangedDate}{}%
  \svnkwdef{LastChangedRevision}{0}%
  \svnkwdef{LastChangedBy}{}%
%    \end{macrocode}
% The catcodes are changed by \cs{svn@catcodes} to allow \TeX-special characters
% inside the keywords.  The braces \{ \} are changed to allow comments between
% the arguments.  \cs{svnidlong@readargsfull} is called to read the arguments.
%    \begin{macrocode}
  \begingroup
    \if@svnmulti@verbatim
      \svn@catcodes
      \catcode`\{=12
      \catcode`\}=12
    \else
      \def\svnidlong@readargsfull{\svnidlong@readargs}%
    \fi
    \svnidlong@readargsfull
}
%    \end{macrocode}
% \end{macro}

% \begin{macro}{\svn@catcodes}
% Changes all \TeX-special character to category ``other''. The newline aka
% return is changed to category ``ignore'' so line breaks are not taken as part
% of the verbatim arguments.
%    \begin{macrocode}
\if@svnmulti@verbatim
\def\svn@catcodes{%
  \let\do\@makeother
  \dospecials
  \catcode`\^^M9
  \catcode`\ 10
  \catcode`\{1
  \catcode`\}2
}
\else
  \def\svn@catcodes{}
\fi
%    \end{macrocode}
% \end{macro}

% \begin{macro}{\svnidlong@readargsfull}[8]
% {some text, ignored}{Keyword 1}
% {some text, ignored}{Keyword 2}
% {some text, ignored}{Keyword 3}
% {some text, ignored}{Keyword 4}
% Reads all four arguments of \cs{svnidlong} and passes them to
% \cs{svnidlong@readargs}. The normal argument braces are changed to category
% ``other'' and put into the macros parameter text to remove all code between
% them. This is done to avoid problems with comments direct after one of the
% arguments. Because the braces are now non-special the parentheses are made to
% a local replacement.
%
%    \begin{macrocode}
\if@svnmulti@verbatim
\begingroup
\catcode`\{=12\catcode`\}=12
\catcode`\(=1\catcode`\)=2
\gdef\svnidlong@readargsfull#1{#2}#3{#4}#5{#6}#7{#8}(%
 \svnidlong@readargs(#2)(#4)(#6)(#8)%
)
\endgroup
\fi
%    \end{macrocode}
% \end{macro}

% \begin{macro}{\svnidlong@readargs}[4]{Keyword 1}{Keyword 2}{Keyword 3}
% {Keyword 4}
% Calls sub macro for all four arguments and ends the catcode changes made
% by \cs{svnidlong}.
%    \begin{macrocode}
\def\svnidlong@readargs#1#2#3#4{%
    \svnkwsave@read #1\relax
    \svnkwsave@read #2\relax
    \svnkwsave@read #3\relax
    \svnkwsave@read #4\relax
  \endgroup
%    \end{macrocode}
% Now the update macros for date and id are called.
%    \begin{macrocode}
  \ifx\svnkwLastChangedDate\empty\else
    \@svn@scanlongdate{\svnkwLastChangedDate}%
  \fi
  \@svn@updateid{\svnkw{LastChangedRevision}}{\svnkw{LastChangedDate}}%
  {\svnkw{LastChangedBy}}{\svnkw{HeadURL}}%
  \ignorespaces
}%
%    \end{macrocode}
% \end{macro}

% \subsection{KeyWord Macros}
% \begin{macro}{\svnkwsave}
% Enabled verbatim mode and uses a sub macro to read the arguments afterwards.
%    \begin{macrocode}
\def\svnkwsave{%
  \begingroup
    \svn@catcodes
    \svnkwsave@readargs
}
%    \end{macrocode}
% \end{macro}

% \begin{macro}{\svnkwsave@readargs}[1]{\$kw: value\$}
% Reads full argument, calls parse submacro and ends catcode changes.
% If \cs{svnkwsave} was called by \cs{svnid} scans the id keyword by calling the
% scan macro.
%    \begin{macrocode}
\gdef\svnkwsave@readargs#1{%
    \svnkwsave@read#1\relax
  \endgroup
  \if@svnidsw
    \ifx\svnkwId\empty\else
      \expandafter
      \svn@scanId\svnkwId\relax
      \@svnidswfalse
    \fi
  \fi
  \ignorespaces
}
%    \end{macrocode}
% \end{macro}

% \begin{macro}{\svnkwsave@read}[1]{keyword line without surrounding \$ \$}
% Reads the full keyword and strips the dollars.
%    \begin{macrocode}
\begingroup
\if@svnmulti@verbatim
\catcode`\$=12
\fi
\gdef\svnkwsave@read $#1$\relax{%
  \svn@checkcolon#1:\relax
}
\endgroup
%    \end{macrocode}
% \end{macro}

% \begin{macro}{\svnkwsave@parse}[2]{key}{value}
% Parse the keyword and save it away.
%    \begin{macrocode}
\begingroup
\catcode`\$=11
\gdef\svnkwsave@parse$#1:#2${%
  \expandafter\xdef\csname svnkw#1\endcsname{#2}%
}%
\endgroup
%    \end{macrocode}
% \end{macro}

% \begin{macro}{\svnkwdef}[2]{key}{value}
% First we check if there is a `setter'-macro for the keyword called
% \cs{svnkwdef@}\meta{keyword}.
%    \begin{macrocode}
\newcommand{\svnkwdef}[2]{%
  \@ifundefined{svnkwdef@#1}%
%    \end{macrocode}
% If not we call the general macro \cs{svnkwdef@}.
%    \begin{macrocode}
    {\svnkwdef@{#1}{#2}}%
%    \end{macrocode}
% If yes we just call it with the value as argument.
%    \begin{macrocode}
    {\csname svnkwdef@#1\endcsname{#2}}%
}
%    \end{macrocode}
% \end{macro}

% \begin{macro}{\svnkwdef@}[2]{key}{value}
% This macro defines the second argument under \cs{svnkw}\meta{1st argument}.
% The |\xdef| is used to expand the content first (needed for internal use) and
% make the definition globally.
%    \begin{macrocode}
\newcommand{\svnkwdef@}[2]{%
  \expandafter\xdef\csname svnkw#1\endcsname{#2}%
}
%    \end{macrocode}
% Example: |\svnkwdef{Revision}{23}| will define |\svnkwRevision| as 23.
% \end{macro}

% \begin{macro}{\svnkwdef@Rev}
% \begin{macro}{\svnkwdef@Author}
% \begin{macro}{\svnkwdef@Date}
% \begin{macro}{\svnkwdef@URL}[1]{value}
% `Setter'-macros for single keywords, used by \cs{svnkwdef}.\\ These are needed
% to have have a common value for all alternative keyword names ala |Rev|,
% |Revision|, |LastChangedRevision|.
%
% The keywords |Author| and |Date| are just calling \cs{svnkwdef@} with a fixed
% first argument.  For the revision the value is checked if empty and then a 0
% is substituted.
%    \begin{macrocode}
\def\svnkwdef@Rev#1{%
  \svn@ifempty{#1}%
    {\svnkwdef@{Rev}{0}}%
    {\svnkwdef@{Rev}{#1}}%
}
\def\svnkwdef@Author#1{\svnkwdef@{Author}{#1}}
\def\svnkwdef@Date#1{\svnkwdef@{Date}{#1}}
\def\svnkwdef@URL#1{\svnkwdef@{HeadURL}{#1}}
%    \end{macrocode}
% The long keywords are defined then as aliases of the short,\\
% first for writing
%    \begin{macrocode}
\let\svnkwdef@Revision=\svnkwdef@Rev
\let\svnkwdef@LastChangedRevision=\svnkwdef@Rev
\let\svnkwdef@LastChangedBy=\svnkwdef@Author
\let\svnkwdef@LastChangedAt=\svnkwdef@Date
%    \end{macrocode}
% and then for reading.
%    \begin{macrocode}
\def\svnkwRevision{\svnkwRev}
\def\svnkwLastChangedRevision{\svnkwRev}
\def\svnkwLastChangedBy{\svnkwAuthor}
\def\svnkwLastChangedAt{\svnkwDate}
\def\svnkwURL{\svnkwHeadURL}
%    \end{macrocode}
% So \eg |\svnkw{LastChangedRevision}| is always be the
% same as |\svnkw{Rev}|.
% \end{macro}
% \end{macro}
% \end{macro}
% \end{macro}

% We define default values for normal keywords. Keyword |Filename| is the name
% given by |Id| and not a real keyword.
%    \begin{macrocode}
\svnkwdef{Rev}{0}
\svnkwdef{Date}{}
\svnkwdef{Author}{}
\svnkwdef{Filename}{}
\svnkwdef{HeadURL}{}
%    \end{macrocode}

% \begin{macro}{\svnkw}[1]{keyword name}
% Macro to get keyword value. Just calls \cs{svnkw}\meta{ARGUMENT} where
% the argument interpreted as text. So \eg |\svnkw{Date}| is the same as
% |svnkwDate| but this could be changed later so always use this interface
% to get the keyword values.
%
% \changes{v1.2}{2007/06/22}{Added warning when a wrong, maybe
% misspelled, keyword is given.}
%    \begin{macrocode}
\newcommand{\svnkw}[1]{%
  \@ifundefined{svnkw#1}%
    {\PackageWarning{svn-multi}{SVN keyword '#1' not defined (typo?)}}%
    {\csname svnkw#1\endcsname}%
}%
%    \end{macrocode}
% \end{macro}
%

% \subsection{Keyword check and strip macros}
% The following macros are used to test whether the given keywords are fully
% expanded or not.
% Subversion supports unexpanded keywords as input with or without colon and
% with or without trailing space(s), \ie a:~|$KW$|, b:~|$KW:$| or c:~|$KW: $|.
% To avoid \LaTeX{} syntax errors in this pre-commit state the keyword is
% checked by the following macros. Unexpanded keywords result in an empty value.
% Also leading and trailing spaces are removed.
%
% \begin{macro}{\svn@checkcolon}[2]{key}{potential value, might be empty}
% Checks if the keyword contains a colon. It is called by \cs{svnkwsave@read}
% with a trailing |:\relax| so that \#2 will be empty if there is no earlier
% colon or will hold the value with this trailing colon otherwise.
% The first case means that the keyword is unexpanded without colon (case a)
% which leads to an empty value. In the second case \cs{svn@stripcolon} is
% called to strip the colon and surrounding spaces. The final value is
% returned by |\svn@value|.
%    \begin{macrocode}
\def\svn@checkcolon#1:#2\relax{%
  \svn@ifempty{#2}%
    {\svnkwdef{#1}{}}%
    {\svn@stripcolon#2\relax\svnkwdef{#1}{\svn@value}}%
}
%    \end{macrocode}
% \end{macro}

% \begin{macro}{\svn@stripcolon}[1]{potential value}
% Strips the previous added colon (for \cs{svn@checkcolon}).
% The remaining argument is checked if it's empty (case b) or only a space
% (case c). Otherwise the keyword is expanded and \cs{svn@stripspace} is
% called to strip the spaces.
%    \begin{macrocode}
\def\svn@stripcolon#1:\relax{%
  \svn@ifempty{#1}%
    {\gdef\svn@value{}}%
    {\svn@ifequal{#1}{ }%
      {\gdef\svn@value{}}%
      {\svn@stripspace#1\relax\relax}%
    }%
}
%    \end{macrocode}
% \end{macro}

% \begin{macro}{\svn@stripspace}[2]{first character}{rest of string}
% Strips leading space if present and calls \cs{svn@striptrailingspace} to
% strip the trailing space.
%    \begin{macrocode}
\def\svn@stripspace#1#2\relax{%
  \svn@ifequal{#1}{ }%
    {\gdef\svn@value{#2}}%
    {\svn@striptrailingspace#1#2\relax}%
}
%    \end{macrocode}
% \end{macro}

% \begin{macro}{\svn@striptrailingspace}[1]{string}
% Strips trailing space using the macros parameter text. Must be called with
% |\relax| as end marker.
%    \begin{macrocode}
\def\svn@striptrailingspace#1 \relax{%
  \gdef\svn@value{#1}%
}
%    \end{macrocode}
% \end{macro}

% \begin{macro}{\svn@gdefverb}[1]{macro}
%    \begin{macrocode}
\def\svn@gdefverb#1{%
  \begingroup
    \def\svn@temp{#1}%
    \begingroup
      \if@svnmulti@verbatim
        \svn@catcodes
      \fi
      \svn@gdefverb@
}
%    \end{macrocode}
% \end{macro}

% \begin{macro}{\svn@defverb@}[1]{verbatim stuff}
%    \begin{macrocode}
\def\svn@gdefverb@#1{%
    \endgroup
    \expandafter\gdef\svn@temp{#1}%
  \endgroup
}
%    \end{macrocode}
% \end{macro}

% \begin{macro}{\svn@namegdefverb}[1]{macro name}
%    \begin{macrocode}
\def\svn@namegdefverb#1{%
  \begingroup
    \expandafter\def
    \expandafter\svn@temp
    \expandafter{\csname #1\endcsname}%
    \begingroup
      \if@svnmulti@verbatim
        \svn@catcodes
      \fi
      \svn@gdefverb@
}
%    \end{macrocode}
% \end{macro}


% \subsection{Date Macros}
% \begin{macro}{\@svn@scandate}[1]{date}
% Scans data information in Id keyword and saves them in macros.
%    \begin{macrocode}
\def\@svn@scandate#1{\@svn@scandate@#1\relax}

\def\@svn@scandate@#1-#2-#3 #4:#5:#6#7#8\relax{%
  \gdef\svnfileyear{#1}%
  \gdef\svnfilemonth{#2}%
  \gdef\svnfileday{#3}%
  \gdef\svnfilehour{#4}%
  \gdef\svnfileminute{#5}%
  \gdef\svnfilesecond{#6#7}%
  \gdef\svnfiletimezonehour{+00}%
  \gdef\svnfiletimezoneminute{00}% #8 always 'Z' for Zulu-time (UTC)
}
%    \end{macrocode}
% \end{macro}

% \begin{macro}{\@svn@scanlongdate}[8]{Year}{Month}{Day}{Hour}{Minute}{Second}
% {Timezone}{Date description string (ignored)}
% Scans date information in Date keyword and saves them in macros.
%    \begin{macrocode}
\def\@svn@scanlongdate#1{\expandafter\@svn@scanlongdate@#1\relax}
%
\def\@svn@scanlongdate@#1-#2-#3 #4:#5:#6 #7 #8\relax{%
  \gdef\svnfileyear{#1}%
  \gdef\svnfilemonth{#2}%
  \gdef\svnfileday{#3}%
  \gdef\svnfilehour{#4}%
  \gdef\svnfileminute{#5}%
  \gdef\svnfilesecond{#6}%
  \@svn@parsetimezone#7\relax%
}
%    \end{macrocode}
% \end{macro}

% \begin{macro}{\@svn@parsetimezone}[5]{sign (+/-)}{hour first digit}{hour
% second digit}{minute first digit}{minute second digit}
% Scans timezone and splits hour and minute part.
%    \begin{macrocode}
\def\@svn@parsetimezone#1#2#3#4#5\relax{%
  \gdef\svnfiletimezonehour{#1#2#3}%
  \gdef\svnfiletimezoneminute{#4#5}%
}
%    \end{macrocode}
% \end{macro}

% \begin{macro}{\svnpdfdate}
% Returns date in a format needed for |\pdfinfo|.
%    \begin{macrocode}
\def\svnpdfdate{%
  \svnyear\svnmonth\svnday
  \svnhour\svnminute\svnsecond\svntimezonehour'\svntimezoneminute'%
}
%    \end{macrocode}
% \end{macro}

% \subsection{Mainfile Makros}
% \begin{macro}{\svnsetmainfile}
% Saves the current |HeadURL| and |Filename| keywords to macros.
% Will be called automatically in the preamble.
% \changes{v1.2}{2007/06/22}{New macro}
%    \begin{macrocode}
\newcommand{\svnsetmainfile}{%
  \xdef\svnmainurl{\svnfileurl}%
  \xdef\svnmainfilename{\svnfilefname}%
}
\AtBeginDocument{\svnsetmainfile}
%    \end{macrocode}
% \end{macro}

% \subsection{Register and FullName Macros}
% \begin{macro}{\svnRegisterAuthor}[2]{author username}{Full Name}
%    \begin{macrocode}
% Saves the author's name by defining
% |svn@author@|\meta{username} to it.
%    \begin{macrocode}
\newcommand{\svnRegisterAuthor}[2]{%
  \expandafter\def\csname svn@author@#1\endcsname{#2}%
}
%    \end{macrocode}
% \end{macro}

% \begin{macro}{\svnFullAuthor}
% \begin{macro}{\svnFullAuthor*}
% We test if the starred or the normal version is used and call the
% appropriate submacro |svnFullAuthor@star| or |svnFullAuthor@normal|.
% \changes{v1.2}{2007/06/22}{Macro now returns the username if the full name
% was not registered.}
%    \begin{macrocode}
\newcommand{\svnFullAuthor}{%
  \@ifnextchar{*}%
    {\svnFullAuthor@star}%
    {\svnFullAuthor@normal}%
}%
%    \end{macrocode}
% \end{macro}
% \end{macro}
% \begin{macro}{\svnFullAuthor@star}[1]{username}
% Both submacros are calling |svnFullAuthor@| but with different arguments.
% The star macro also removes the star of course.
%    \begin{macrocode}
\def\svnFullAuthor@star*#1{%
  \edef\svn@temp{#1}%
  \svnFullAuthor@{\svn@temp}{~(\svn@temp)}%
}%
%    \end{macrocode}
% \end{macro}
% \begin{macro}{\svnFullAuthor@normal}[1]{username}
%    \begin{macrocode}
\def\svnFullAuthor@normal#1{%
  \edef\svn@temp{#1}%
  \svnFullAuthor@{\svn@temp}{}%
}%
%    \end{macrocode}
% \end{macro}
% \begin{macro}{\svnFullAuthor@}[2]{username}{previous defined trailing string}
% |svnFullAuthor@| now sets the author's full name. Note that |#2| is empty
% when the normal version is called.
%    \begin{macrocode}
\def\svnFullAuthor@#1#2{%
  \@ifundefined{svn@author@#1}%
    {#1}%
    {\csname svn@author@#1\endcsname #2}%
}
%    \end{macrocode}
% \end{macro}

% \begin{macro}{\svnRegisterRevision}[2]{revision number}{tag name}
% Saves the revision's name or tag by defining
% |svn@revision@|\meta{revisionnumber} to it.
% \changes{v1.2}{2007/06/22}{New macro}
%    \begin{macrocode}
\newcommand{\svnRegisterRevision}[2]{%
  \expandafter\def\csname svn@revision@#1\endcsname{#2}%
}
%    \end{macrocode}
% \end{macro}

% \begin{macro}{\svnFullRevision}
% \begin{macro}{\svnFullRevision*}
% We test if the starred or the normal version is used and call the
% appropriate submacro |svnFullRevision@star| or |svnFullRevision@normal|.
% \changes{v1.2}{2007/06/22}{New macro}
%    \begin{macrocode}
\newcommand{\svnFullRevision}{%
  \@ifnextchar{*}%
    {\svnFullRevision@star}%
    {\svnFullRevision@normal}%
}
%    \end{macrocode}
% \end{macro}
% \end{macro}
%
% \begin{macro}{\svnFullRevision@star}[1]{revision number}
% Both submacros are calling |svnFullRevision@| but with different arguments.
% The star macro also removes the star of course.
%    \begin{macrocode}
\def\svnFullRevision@star*#1{%
  \edef\svn@temp{#1}%
  \svnFullRevision@{\svn@temp}{~(r\svn@temp)}%
}
%    \end{macrocode}
% \end{macro}
% \begin{macro}{\svnFullRevision@normal}[1]{revision number}
%    \begin{macrocode}
\def\svnFullRevision@normal#1{%
  \edef\svn@temp{#1}%
  \svnFullRevision@{\svn@temp}{}%
}
%    \end{macrocode}
% \end{macro}
% \begin{macro}{\svnFullRevision@}[2]{revision number}{previous defined trailing
% string}
% |svnFullRevision@| now sets the revision name. Note that |#2| is empty
% when the normal version is called.
%    \begin{macrocode}
\def\svnFullRevision@#1#2{%
  \@ifundefined{svn@revision@#1}%
    {Revision #1}%
    {\csname svn@revision@#1\endcsname #2}%
}
%    \end{macrocode}
% \end{macro}

% \subsection{Input File Name}
% The FiNK package is used to get the input file names. AtBegin/AtEnd hooks are
% installed which will be used later.
%    \begin{macrocode}
\if@svnmulti@needinputfilename
%    \end{macrocode}

% Load \pkg{fink} package and check if all needed macros are provided.
%    \begin{macrocode}
\RequirePackage{fink}[2008/02/27]
\begingroup
\def\svn@finkerror{%
\PackageError{svn-multi}{Your installed version of the 'fink' package does not
provide the needed macros. It is either too old or too new.
Try a different version, e.g. v2.1.1 from 2008/02/27}{}%
\let\svn@finkerror\relax
}
\@ifundefined{finkpath}{\svn@finkerror}{}%
\@ifundefined{finkdir}{\svn@finkerror}{}%
\@ifundefined{finkbase}{\svn@finkerror}{}%
\@ifundefined{fink@prepare}{\svn@finkerror}{}%
\@ifundefined{fink@restore}{\svn@finkerror}{}%
\endgroup
%    \end{macrocode}

% \begin{macro}{\svnmulti@begininputfilehook}
% This hook is installed in the |\fink@prepare| macro from the \pkg{fink}
% package which will be executed at the begin of a input file. The file name and
% path are not yet in |\finkpath| etc. but in |\fink@nextpath|.
%    \begin{macrocode}
\message{Package svn-multi: patching macro '\string\fink@prepare' from the
'fink' package!}%
\def\svnmulti@begininputfilehook{}
\let\svnmulti@fink@prepare\fink@prepare
\renewcommand*{\fink@prepare}[1]{%
  \svnmulti@fink@prepare{#1}%
  \svnmulti@begininputfilehook
}
%    \end{macrocode}
% \end{macro}

% \begin{macro}{\svnmulti@endinputfilehook}
% This hook is installed in the |\fink@restore| macro from the \pkg{fink}
% package which will be executed at the end of a input file. The file path
% |\finkpath| etc. is still valid.
%    \begin{macrocode}
\message{Package svn-multi: patching macro '\string\fink@restore' from the
'fink' package!}%
\def\svnmulti@endinputfilehook{}
\let\svnmulti@fink@restore\fink@restore
\def\fink@restore{%
  \svnmulti@endinputfilehook
  \svnmulti@fink@restore
}
%    \end{macrocode}
% \end{macro}

% \begin{macro}{\svnmulti@atbegininputfile}
% This macro adds the argument to the end of the \cs{svnmulti@begininputfilehook}.
%    \begin{macrocode}
\def\svnmulti@atbegininputfile{%
  \g@addto@macro\svnmulti@begininputfilehook
}
%    \end{macrocode}
% \end{macro}

% \begin{macro}{\svnmulti@atendinputfile}
% This macro adds the argument to the \emph{begin} of the
% \cs{svnmulti@endinputfilehook}. This ensures that code added first is more at
% the end than code added later.
% The code below was adapted from the definition of the \LaTeX2e macro
% |\g@addto@macro| which was used above.
%    \begin{macrocode}
\long\def\svnmulti@atendinputfile#1{%
  \begingroup
    \@temptokena\expandafter{\svnmulti@endinputfilehook}%
    \toks@{#1}%
    \xdef\svnmulti@endinputfilehook{\the\toks@\the\@temptokena}%
  \endgroup
}
%    \end{macrocode}
% \end{macro}

%    \begin{macrocode}
\def\svn@filestack{{}}
\def\svn@indent{}

\def\svn@pushfilestack{%
  \edef\svn@indent{.\svn@indent}%
  \xdef\svn@filestack{{%
    {\svnfilerev}%
    {\svnfiledate}%
    {\svnfileauthor}%
    {\svnfileyear}%
    {\svnfilemonth}%
    {\svnfileday}%
    {\svnfilehour}%
    {\svnfileminute}%
    {\svnfilesecond}%
    {\svnfiletimezonehour}%
    {\svnfiletimezoneminute}%
    {\svnfileurl}%
    {\svnfilefname}%
  }\svn@filestack}%
}

\def\svn@indentit#1#2\empty{%
  \xdef\svn@indent{#2}%
}

\def\svn@restorefilekws#1#2\relax{%
  \svn@restorefilekws@#1\empty
  \empty \empty \empty \empty
  \empty \empty \empty \empty
  \empty \empty \empty \empty
  \svn@ifempty{#2}%
    {\gdef\svn@filestack{{}}}%
    {\gdef\svn@filestack{#2}}%
}
\def\svn@restorefilekws@#1#2#3#4#5#6#7#8#9{%
  \gdef\svnfilerev{#1}%
  \gdef\svnfiledate{#2}%
  \gdef\svnfileauthor{#3}%
  \gdef\svnfileyear{#4}%
  \gdef\svnfilemonth{#5}%
  \gdef\svnfileday{#6}%
  \gdef\svnfilehour{#7}%
  \gdef\svnfileminute{#8}%
  \gdef\svnfilesecond{#9}%
  \svn@restorefilekws@@
}

\def\svn@restorefilekws@@#1#2#3#4{%
  \gdef\svnfiletimezonehour{#1}%
  \gdef\svnfiletimezoneminute{#2}%
  \gdef\svnfileurl{#3}%
  \gdef\svnfilefname{#4}%
}

\def\svn@popfilestack{%
  \ifx\svn@filestack\empty
    \PackageWarning{svn-multi}{Underflow of file keyword stack!}%
  \else
    \svn@ifequal{\svn@filestack}{{}}%
      {\PackageWarning{svn-multi}{Underflow of file keyword stack!}}%
      {\expandafter\svn@restorefilekws\svn@filestack\relax}%
  \fi
}

\svnmulti@atbegininputfile{%
  \svn@pushfilestack
}

\svnmulti@atendinputfile{%
  \svn@popfilestack
}%
%
%    \end{macrocode}

% \begin{macro}{\svn@removedotslash}[1]{string (\eg file path) which might start
% with \texttt{./}}
% Removes leading './' from given macro (holding a directory path). Awaits a
% macro as argument which is redefined inside the current group!
%    \begin{macrocode}
\def\svn@removedotslash#1{%
  \def\svn@removedotslash@##1##2##3\relax{%
    \svn@ifequal{./}{##1##2}%
      {\edef#1{##3}\def\next{\svn@removedotslash@##3\empty\empty\empty\relax}}%
      {\edef#1{##1##2##3}\let\next\relax}%
    \next
  }%
  \expandafter\svn@removedotslash@#1\empty\empty\empty\relax
}
%    \end{macrocode}
% \end{macro}

%    \begin{macrocode}
\fi
%    \end{macrocode}

% \subsection{Keyword Group Macros}
% These macros implement the user interface for the keyword group functionality
% introduced with v2.0.
%
% The list of keyword groups |\svn@glist| is initial set empty and will be
% filled by \cs{svngroup}.
%    \begin{macrocode}
\if@svnmulti@groups
\let\svn@glist=\empty
%    \end{macrocode}

% \begin{macro}{\svngroup}[1]{group name}
% Saves the group to |\svn@g| and initiates |\svn@g@|\meta{group name}|@rev|
% and |\@svn@g@|\meta{group name}|@rev| if this is the first time the group
% got used.\par
% The current group symbol `|*|' is invalid here because there is no way to
% change to a current group.
%    \begin{macrocode}
\def\svngroup#1{%
  \svn@ifequal{#1}{*}%
    {\PackageError{svn-multi}%
      {The group name '*' is invalid for '\string\svngroup'}{}{}%
    }{}%
  \xdef\svn@g{#1}%
  \ifx\svn@g\empty\else%
%    \end{macrocode}
% Only initialise the group at first usage:
%    \begin{macrocode}
    \expandafter
    \ifx\csname svn@g@#1\endcsname\relax%
      \expandafter\gdef\csname svn@g@#1\endcsname{1}%
%    \end{macrocode}
% If first use, init revision numbers to avoid not-a-number errors:
%    \begin{macrocode}
      \expandafter\gdef\csname @svng@#1@rev\endcsname{-1}%
      \expandafter
      \ifx\csname svng@#1@rev\endcsname\relax
        \expandafter\gdef\csname svng@#1@rev\endcsname{-1}%
      \fi
%    \end{macrocode}
% Now save new group to list. The list is checked if its empty to avoid an
% unwanted leading comma.
%    \begin{macrocode}
      \ifx\svn@glist\empty
        \xdef\svn@glist{#1}%
      \else
        \xdef\svn@glist{\svn@glist,#1}%
      \fi
    \fi
  \fi
}
%    \end{macrocode}
% \end{macro}

% \begin{macro}{\thesvngroup}
% Returns the current group name to the user.
%    \begin{macrocode}
\def\thesvngroup{\svn@g}
%    \end{macrocode}
% \end{macro}

% \begin{macro}{\svnsetcg}[1]{group name}
% Defines |\svn@cg| to the given argument or to |\svn@g| if the argument was
% `|*|'.
%    \begin{macrocode}
\def\svnsetcg#1{%
  \svn@ifequal{#1}{*}%
    {\def\svn@cg{\svn@g}}%
    {\def\svn@cg{#1}}%
}
%    \end{macrocode}
% \end{macro}

% \begin{macro}{\svncg@def}[1]{key name, \eg `rev', `date'}
% Defines a |\svncgXXX| macro, \eg |svncgrev|, which returns the
% requested keyword values of the current keyword group.
%    \begin{macrocode}
\def\svncg@def#1{%
  \expandafter
  \def\csname svncg#1\endcsname{%
    \@ifundefined{svng@\svn@cg @#1}{??}{%
    \csname svng@\svn@cg @#1\endcsname}%
  }%
}
%    \end{macrocode}
% \end{macro}

% \begin{macro}{\svncgrev}
% \begin{macro}{\svncgdate}
% \begin{macro}{\svncgauthor}
% \begin{macro}{\svncgyear}
% \begin{macro}{\svncgmonth}
% \begin{macro}{\svncgday}
% \begin{macro}{\svncghour}
% \begin{macro}{\svncgminute}
% \begin{macro}{\svncgsecond}
% \begin{macro}{\svncgtimezonehour}
% \begin{macro}{\svncgtimezoneminute}
% \begin{macro}{\svncgurl}
% \begin{macro}{\svncgfname}
% Define all |\svncgXXX| macros by calling \cs{svncg@def} in a for loop.
%    \begin{macrocode}
\@for\@tempa:=%
  rev,author,date,year,month,day,hour,minute,second,%
  timezonehour,timezoneminute,url,fname%
\do{%
  \expandafter\svncg@def\expandafter{\@tempa}%
}
%    \end{macrocode}
% \end{macro}
% \end{macro}
% \end{macro}
% \end{macro}
% \end{macro}
% \end{macro}
% \end{macro}
% \end{macro}
% \end{macro}
% \end{macro}
% \end{macro}
% \end{macro}
% \end{macro}

% \begin{macro}{\thesvncg}
% Simply return the internal macro.
%    \begin{macrocode}
\def\thesvncg{\svn@cg}
%    \end{macrocode}
% \end{macro}

% \begin{macro}{\svng}[2]{group name}{keyword name}
% Simply returns |svng@#1@#2| if defined, '??' otherwise.
%    \begin{macrocode}
\def\svng#1#2{%
  \@ifundefined{svng@\svn@temp @#2}%
    {??}%
    {\csname svng@\svn@temp @#2\endcsname}%
}
%    \end{macrocode}
% \end{macro}

% \begin{macro}{\svn@addfiletogroup}[2]{file name}{group name}
% Adds the given file to the given group. If the group list doesn't exist yet
% it is initialised. A extra macro for each file is used to remember that the
% file is already in the group. This could be avoided using a list search.\par
% This is an internal macro so no `|*|' substitution for the group name.
%    \begin{macrocode}
\def\svn@addfiletogroup#1#2{%
  \expandafter
  \ifx\csname @svng@#2@files@#1\endcsname\relax%
    \expandafter\gdef\csname @svng@#2@files@#1\endcsname{1}%
    %
    \expandafter
    \ifx\csname @svng@#2@files\endcsname\relax%
      \expandafter\xdef\csname @svng@#2@files\endcsname{#1}%
    \else
      \expandafter\xdef\csname @svng@#2@files\endcsname{%
        \csname @svng@#2@files\endcsname,#1%
      }%
    \fi
  \fi
}
%    \end{macrocode}
% \end{macro}

% The input files are added to the list of the current group at their begin to
% have them before the included graphics and other external files.
% Special care is taken to not re-initialise the main file which could happen in
% some special cases (\eg |\lstinputlisting{\jobname .tex}|).
%    \begin{macrocode}
\svnmulti@atbegininputfile{%
  \begingroup
  \let\svn@temp=\fink@nextdir
  \svn@removedotslash\svn@temp
  \svn@ifempty{\svn@temp}%
    {\svn@ifequal{\fink@nextext}{\fnk@mainext}%
      {}%
      {\svn@initfile}%
    }%
    {\svn@initfile}%
  \svn@addfiletogroup{\svn@temp\fink@file\fink@nextbase\fink@nextext}{\svn@g}%
  \endgroup
}
%    \end{macrocode}

%    \begin{macrocode}
\fi
%    \end{macrocode}

% \subsection{Files as extra groups}
% Macros which allow single files to be declared as extra groups so that their
% keywords can be accessed in the whole document like with normal groups.
% This special groups are not added to the list of groups.

% A user-level switch is declared to enable or disable the automatic declaration
% of every file as own group. This causes \cs{svnfileasgroup} to be called for
% all input files.
% The if macro is defined outside the |\if@svnmulti@filesasgroups| because
% |\newif| inside |\if| is not a good idea.
%    \begin{macrocode}
\newif\ifsvnfilesasgroup
\svnfilesasgroupfalse
%    \end{macrocode}

%    \begin{macrocode}
\if@svnmulti@filesasgroups
\svnfilesasgrouptrue
%    \end{macrocode}

% \begin{macro}{\svnfileasgroup}
% User level and internal macro to declare the current file as extra group.
% It produces the current file path and calls \cs{svn@fileasgroup}.
%    \begin{macrocode}
\def\svnfileasgroup{%
  \begingroup
    \edef\svn@filename{\finkdir\fink@file\finkbase\finkext}%
    \svn@removedotslash\svn@filename
    \svn@fileasgroup{\svn@filename}%
  \endgroup
}
%    \end{macrocode}
% \end{macro}

% \begin{macro}{\svn@fileasgroup}[1]{file name}
% Macro to write a file as group to |.svn| file. After checking if the filename
% was not already written, the |.svn| file is checked if it is open and then the
% file keyword information is written.
%    \begin{macrocode}
\def\svn@fileasgroup#1{%
 \ifnum\svnfilerev>-1\relax
   \begingroup
     \expandafter\ifx\csname svn@g@#1\endcsname\relax%
       \expandafter\gdef\csname svn@g@#1\endcsname{1}%
       \svn@checkwrite
       \def\svn@writekw##1{%
         \noexpand\@namedef{@svng@#1@##1}{\csname svnfile##1\endcsname}^^J%
       }%
       \immediate\write\svn@write{^^J%
         \@percentchar\space File '#1'^^J%
         \svn@writekw{rev}%
         \svn@writekw{date}%
         \svn@writekw{author}%
         \svn@writekw{year}%
         \svn@writekw{month}%
         \svn@writekw{day}%
         \svn@writekw{hour}%
         \svn@writekw{minute}%
         \svn@writekw{second}%
         \svn@writekw{timezonehour}%
         \svn@writekw{timezoneminute}%
         \noexpand
         \svn@namegdefverb{@svng@#1@url}{\csname svnfileurl\endcsname}^^J%
         \noexpand
         \svn@namegdefverb{@svng@#1@fname}{\csname svnfilefname\endcsname}^^J%
       }%
     \fi
   \endgroup
 \fi
}
%    \end{macrocode}
% \end{macro}

% \begin{macro}{\svnignoreextensions}[1]{A comma separated list of file name
% extensions (without leading dot) to ignore for automatic \csd{svnfileasgroup}.}
% A special macro is defined for all extensions. The existents of this macro is
% then tested later to check if this extension should be ignored.
%    \begin{macrocode}
\def\svnignoreextensions#1{%
  \@for\svn@temp:=#1\do{%
    \expandafter\def\csname svn@ignore@ext@\svn@temp\endcsname{}%
  }%
}
%    \end{macrocode}
% \end{macro}

% \begin{macro}{\svnconsiderextensions}[1]{A comma separated list of file name
% extensions (without leading dot) to consider for automatic
% \csd{svnfileasgroup}.}
% The special macro defined by \csi{svnignoreextentions} is deleted, i.e. |\let|
% to |\relax|.
%    \begin{macrocode}
\def\svnconsiderextensions#1{%
  \@for\svn@temp:=#1\do{%
  \expandafter\let\csname svn@ignore@ext@\svn@temp\endcsname\relax%
  }%
}
%    \end{macrocode}
% \end{macro}

% The following extensions are ignored by default.
%    \begin{macrocode}
\svnignoreextensions{aux,toc,out,svn,svx,cls,sty,cfg,enc}
%    \end{macrocode}

% Check at the end of every input file if files should be extra groups and
% declare this file as group if its extension is not configured to be ignored.
%    \begin{macrocode}
\svnmulti@atendinputfile{%
  \if@svnmulti@filesasgroups
    \ifsvnfilesasgroup
      \expandafter\ifx\csname svn@ignore@ext@\finkext\endcsname\relax
      \svnfileasgroup
      \fi
    \fi
  \fi
}
%    \end{macrocode}

%    \begin{macrocode}
\AtBeginDocument{%
  \if@svnmulti@filesasgroups
    \svn@addfiletogroup{\jobname .\fnk@mainext}{\svn@g}%
    \ifsvnfilesasgroup
      \svnfileasgroup
    \fi
  \fi
}
%    \end{macrocode}

%    \begin{macrocode}
\fi
%    \end{macrocode}

% \subsection{External Files}
% Macros to declare external files and load the keywords from |.svx| files
% generated by \scr{svn-multi.pl}.
%    \begin{macrocode}
\if@svnmulti@external
%    \end{macrocode}

% \begin{macro}{\svnexternalgroup}[1]{group name}
% Defines the default group of external files. The default is to always use the
% current group.  An empty argument puts the external files in no group. A `|*|'
% switches back to always use the current group.
%    \begin{macrocode}
\if@svnmulti@groups
\def\svnexternalgroup#1{%
  \svn@ifequal{#1}{*}%
    {\def\svn@externalgroup{#1}}%
    {\def\svn@externalgroup{\svn@g}}%
}
\def\svn@externalgroup{\svn@g}
\else
\def\svn@externalgroup{}
\fi
%    \end{macrocode}
% \end{macro}

% \begin{macro}{\svnexternal}[2]{group name}{list of filenames in \{ \}}
% Writes the current input file path and its argument as arguments of
% \cs{@svnexternal} into the |.svn| file.
%    \begin{macrocode}
\newcommand*\svnexternal[2][\svn@externalgroup]{%
  \if@filesw
    \svn@checkwrite
    \begingroup
      \svn@ifequal{#1}{*}%
        {\def\svn@temp{\svn@g}}%
        {\def\svn@temp{#1}}%
      \svn@removedotslash\finkpath
      \immediate\write\svn@write{%
        \noexpand\@svnexternal[\svn@temp]{\finkpath}{#2}%
      }%
    \endgroup
  \fi
}
%    \end{macrocode}
% \end{macro}

% \begin{macro}{\svnexternalpath}[1]{list of paths in \{ \}}
% Writes its argument as argument of \cs{@svnexternalpath} into the |.svn| file.
%    \begin{macrocode}
\def\svnexternalpath#1{%
  \if@filesw
    \svn@checkwrite
    \immediate\write\svn@write{%
      \noexpand\@svnexternalpath{#1}%
    }%
  \fi
}
%    \end{macrocode}
% \end{macro}

% \begin{macro}{\@svnexternal}
% \begin{macro}{\@svnexternalpath}
% Discards the argument(s). These macros and their arguments are only used by
% the external \scr{svn-multi.pl} script.
%    \begin{macrocode}
\newcommand*\@svnexternal[3][]{}
\def\@svnexternalpath#1{}
%    \end{macrocode}
% \end{macro}
% \end{macro}


% \begin{macro}{\svnexternalfile}
% This macro is generated by \scr{svn-multi.pl} and should not be used by the
% user.  If files-as-group is enabled some special characters are disabled and
% the \cs{svn@externalfile} is called to read the file name. Otherwise the
% argument is simply removed.
%    \begin{macrocode}
\def\svnexternalfile{%
  \begingroup
    \catcode`\_=12
    \catcode`\&=12
    \catcode`\^=12
    \catcode`\$=12
    \catcode`\#=12
    \expandafter\svn@externalfile
}
%    \end{macrocode}
% \end{macro}

% \begin{macro}{\svn@externalfile}[1]{file name}
% Ends group which began in \cs{svnexternalfile} and calls the appropriate
% macros.
%    \begin{macrocode}
\def\svn@externalfile#1{%
  \endgroup
  \svn@addfiletogroup{#1}{\svn@g}%
  \if@svnmulti@filesasgroups
    \ifsvnfilesasgroup
      \svn@fileasgroup{#1}%
    \fi
  \fi
}
%    \end{macrocode}
% \end{macro}

% If \op{external} option is not enabled a placeholder macro is defined which
% simply ignores its argument.
%    \begin{macrocode}
\else
  \def\svnexternalfile#1{}%
\fi
%    \end{macrocode}


% \subsection{Auto loading of \texttt{.svx} files}
% Auto loading of |.svx| files at the begin of |\input| or |\include| files
% using the \cs{svnmulti@atbegininputfile} macro.
% The macros \cs{svn@addfiletogroup} and \cs{svnfileasgroup} are used to do the
% actual work.
%    \begin{macrocode}
\if@svnmulti@autoload

\svnmulti@atbegininputfile{%
  \svn@input{\fink@nextdir\fink@nextbase .svx}%
}
%    \end{macrocode}

% The main |.svx| is loaded at the end of the package.
%    \begin{macrocode}
\AtEndOfPackage{%
  \svn@input{\jobname .svx}%
}
%    \end{macrocode}

%    \begin{macrocode}
\fi
%    \end{macrocode}


% \subsection{Support for Graphic Packages}

% \subsubsection{Common Code}
%    \begin{macrocode}
\if@svnmulti@anygraphic
%    \end{macrocode}

% \begin{macro}{\svngraphicsgroup}[1]{graphic group name}
% Defines the default group of graphics files. The default is empty which means
% the current group.
%    \begin{macrocode}
\def\svngraphicsgroup#1{%
  \svn@ifequal{#1}{*}%
    {\def\svn@graphicsgroup{\svn@g}}%
    {\def\svn@graphicsgroup{#1}}%
}
\def\svn@graphicsgroup{\svn@externalgroup}
%    \end{macrocode}
% \end{macro}

% \begin{macro}{\svnignoregraphic}[1]{file name/path}
% Ignores the given graphic file by defining a special macro.
%    \begin{macrocode}
\def\svnignoregraphic#1{%
  \expandafter\def\csname svn@ignoregraphic@#1\endcsname{}%
}
%    \end{macrocode}
% \end{macro}

% \begin{macro}{\svnconsidergraphic}[1]{file name/path}
% Deletes the special ignore macro to consider the graphic again.
%    \begin{macrocode}
\def\svnconsidergraphic#1{%
  \expandafter\let\csname svn@ignoregraphic@#1\endcsname\relax%
}
%    \end{macrocode}
% \end{macro}

%    \begin{macrocode}
\fi
%    \end{macrocode}

% \subsubsection{Package \texttt{graphics}}
% Automatic declaration of all images included by |\includegraphics| from the
% \pkg{graphics} package as external files. We use the |\Gin@setfile| macro from
% that package which receives the image file name as third argument.
%    \begin{macrocode}
\if@svnmulti@graphics
\RequirePackage{graphics}[2006/02/20]
%    \end{macrocode}

% \begin{macro}{\Gin@setfile}[3]{??, not used}{??, not used}{graphic file
% name/path}
%    \begin{macrocode}
\message{Package svn-multi: patching macro '\string\Gin@setfile' from the
'graphics' package!}%
\let\svnmulti@Gin@setfile\Gin@setfile
\renewcommand*{\Gin@setfile}[3]{%
  \expandafter\ifx\csname svn@ignoregraphic@#3\endcsname\relax%
    \svnexternal[\svn@graphicsgroup]{{#3}}%
  \fi
  \svnmulti@Gin@setfile{#1}{#2}{#3}%
}
%    \end{macrocode}
% \end{macro}

%    \begin{macrocode}
\fi
%    \end{macrocode}

% \subsubsection{Package \texttt{pgf}}
% The \pkg{pgf} macro |\pgf@declareimage| which is called by the user macro
% |\pgfdeclareimage| is used.
%    \begin{macrocode}
\if@svnmulti@pgfimages
\RequirePackage{pgf}[2008/01/15]
%    \end{macrocode}

% \begin{macro}{\pgf@declareimage}[3]{??, not used}{image label}{??, not used}
%    \begin{macrocode}
\message{Package svn-multi: patching macro '\string\pgf@declareimage' and the
generated macro '\string\pgf@image@<name>!' from the 'pgf' package!}%
\let\svnmulti@pgf@declareimage\pgf@declareimage
\renewcommand*{\pgf@declareimage}[3][]{%
  \svnmulti@pgf@declareimage[#1]{#2}{#3}%
%    \end{macrocode}
% At this point the used image filename is defined by |\pgf@filename| and the
% image itself is defined by |\pgf@image@#2!| which is a |\let| copy of
% temporary |\pgf@image|.  An own copy of this is created and the old name
% |\pgf@image@#2!| is used to execute \cs{svnexternal} every time the image is
% included using |\pgfuseimage|.
% \begin{macrocode}
  \ifx\pgf@filename\empty\else
    \expandafter\ifx\csname svn@ignoregraphic@\pgf@filename\endcsname\relax%
      \expandafter\global\expandafter%
      \let\csname svnmulti@pgf@image@#2!\endcsname=\pgf@image%
      \expandafter\xdef\csname pgf@image@#2!\endcsname{%
        \noexpand\svnexternal[\noexpand\svn@graphicsgroup]{{\pgf@filename}}%
        \csname svnmulti@pgf@image@#2!\endcsname
      }%
    \fi
  \fi
}
%    \end{macrocode}
% \end{macro}
%    \begin{macrocode}
\fi
%    \end{macrocode}
%
% \subsection{Table of Revisions}
%
%    \begin{macrocode}
\if@svnmulti@table
\ifx\tableofcontents\relax\else
%    \end{macrocode}
%
% \begin{macro}{\svnrevisionsname}
% Simple definition for now. Language support over `babel's |\languagename|
% possible.
%    \begin{macrocode}
\def\svnrevisionsname{Table of Revisions}%
%    \end{macrocode}
% \end{macro}
%
% \begin{macro}{\svn@svt}
% File ending for table of revision auxiliary file. A macro is used to allow
% redefinition by the user if another package is uses the same ending.
%    \begin{macrocode}
\def\svn@svt{svt}
%    \end{macrocode}
% \end{macro}
%

% \begin{macro}{\tableofrevisions}
% The |\tableofcontents| macro from standard \LaTeX\ is adapted for this macro.
% Classes which provide chapters will get a different table then one which not.
%    \begin{macrocode}
\AtBeginDocument{%
\ifx\chapter\@undefined

%% Adapted from the \tableofcontents macro, LaTeX `article' class [2005/09/16]
\newcommand\tableofrevisions{%
  \section*{\svnrevisionsname
  \@mkboth{\MakeUppercase\svnrevisionsname}{\MakeUppercase\svnrevisionsname}}%
  \svn@input{\jobname .\svn@svt}%
}

\else

%% Adapted from the \tableofcontents macro, LaTeX `book' class [2005/09/16]
\newcommand\tableofrevisions{%
  \if@twocolumn
    \@restonecoltrue\onecolumn
  \else
    \@restonecolfalse
  \fi
  \chapter*{\svnrevisionsname
    \@mkboth{\MakeUppercase\svnrevisionsname}{\MakeUppercase\svnrevisionsname}}%
  \svn@input{\jobname .\svn@svt}%
  \if@restonecol\twocolumn\fi
}

\fi
}
%    \end{macrocode}
% \end{macro}
%
%    \begin{macrocode}
\fi % defined \tableofcontents
%    \end{macrocode}

% \begin{macro}{\svn@writerow}[2]{row type ('group', 'file', 'global', \ldots)}
% {row type specific argument}
% Writes a table row by using |\svn@tabcell| and |\svn@tabcellarg| defined by
% the |\svn@writeXXXrow| macro below.
%    \begin{macrocode}
\def\svn@writerow#1#2{%
  \immediate\write\svn@svtwrite{%
    \expandafter\noexpand\csname svn#1row\endcsname
    \expandafter\noexpand\csname svntab#1\endcsname{#2}\space
    \@ampersamchar\space
    \svn@tabcell{rev}\space\@ampersamchar\space
    \svn@tabcell{author}\space\@ampersamchar\space
    \noexpand\svntabdate%
    \svn@tabcellarg{year}%
    \svn@tabcellarg{month}%
    \svn@tabcellarg{day}%
    \svn@tabcellarg{hour}%
    \svn@tabcellarg{minute}%
    \svn@tabcellarg{second}%
    \svn@tabcellarg{timezonehour}%
    \svn@tabcellarg{timezoneminute}%
    \@backslashchar\@backslashchar
  }%
}
%    \end{macrocode}
% \end{macro}

% \begin{macro}{\svn@writegrouprow}[1]{current group}
%    \begin{macrocode}
\def\svn@writegrouprow#1{%
  \begingroup
    \def\svn@tabcellarg##1{{\csname @svng@#1@##1\endcsname}}%
    \def\svn@tabcell##1{\expandafter\noexpand\csname svntab##1\endcsname%
      \svn@tabcellarg{##1}%
    }%
    \svn@writerow{group}{#1}%
  \endgroup
}
%    \end{macrocode}
% \end{macro}

% \begin{macro}{\svn@writefilerow}[1]{file name}
%    \begin{macrocode}
\def\svn@writefilerow#1{%
  \expandafter
  \ifx\csname @svng@#1@rev\endcsname\relax\else
  \begingroup
    \def\svn@tabcellarg##1{{\csname @svng@#1@##1\endcsname}}%
    \def\svn@tabcell##1{\expandafter\noexpand\csname svntab##1\endcsname%
      \svn@tabcellarg{##1}%
    }%
    \svn@writerow{file}{#1}%
  \endgroup
  \fi
}
%    \end{macrocode}
% \end{macro}

% \begin{macro}{\svn@writeglobalrow}
%    \begin{macrocode}
\def\svn@writeglobalrow{%
  \begingroup
  \def\svn@tabcellarg##1{{\csname @svn@##1\endcsname}}%
  \def\svn@tabcell##1{\expandafter\noexpand\csname svntab##1\endcsname%
    \svn@tabcellarg{##1}%
  }%
  \svn@writerow{global}{}%
  \endgroup
}
%    \end{macrocode}
% \end{macro}

% \subsubsection{Table Format Macros}\label{sec:impl:table}
% Generic format macro used in the |.svt| file. Can be redefined by the user to
% change table format. % TODO: More documentation needed!
%    \begin{macrocode}
\def\svntable{%
  \begin{tabular}{lrll}%
    \hline
    \strut Name & Rev & Last Author & Last Changed At \\\hline
}
\def\endsvntable{\hline\end{tabular}}
\def\svnbeforetable{}
\def\svnaftertable{\clearpage}
\def\svnglobalrow{}
\def\svngrouprow{}
\def\svnfilerow{}
\def\svntabgroup#1{Group `#1'}
\def\svntabglobal{Whole Document}
\def\svntabfile{%
  \begingroup
  \catcode`\_=12
  \catcode`\&=12
  \catcode`\^=12
  \catcode`\$=12
  \catcode`\#=12
  \svn@tabfile
}
\def\svn@tabfile#1{\endgroup\hspace*{1em}File `\texttt{#1}'}
\def\svntabrev{}
\def\svntabauthor#1{\svnFullAuthor{#1}}
\def\svntabdate#1#2#3#4#5#6#7#8{%
    #1-#2-#3 #4:#5:#6 #7#8%
}
%    \end{macrocode}

%    \begin{macrocode}
\fi
%    \end{macrocode}
%

% \subsection{Other macros}
% This section contains macros which don't fit in any other section.
%
% \begin{macro}{\svn}
% \begin{macro}{\svn*}
% After *-testing, the intermediate macros |\svn@s| and |\svn@n| are called to
% strip the |{ }| from |\svn|[|*|]|{$...$}| and to remove the |*|. Then the
% actual macros are called to strip the dollars with or without the space
% before the last dollar.
% \changes{v1.2}{2007/06/22}{Added star version. Normal version was not
% changed to not break existing documents with user defined keywords without
% leading space.}
%    \begin{macrocode}
\newcommand{\svn}{\@ifnextchar{*}{\svn@s}{\svn@n}}
\def\svn@n#1{\@svn@n#1}
\def\svn@s*#1{\@svn@s#1}
\def\@svn@n$#1${#1}
\def\@svn@s$#1 ${#1}
%    \end{macrocode}
% \end{macro}
% \end{macro}

% \begin{macro}{\svnnolinkurl}[1]{URL}
% This code is taken from the \pkg{hyperref} package and is the definition of
% |\url| just without the part which creates the actual hyperlink. This needs
% of course the \pkg{hyperref} package. A warning is given if it isn't loaded.
% \changes{v1.2}{2007/06/22}{New macro}
%    \begin{macrocode}
%% Adapted from the \url macro of the `hyperref` package.
\DeclareRobustCommand*{\svnnolinkurl}{%
  \@ifundefined{hyper@normalise}%
    {\PackageWarning{svn-multi}{Package hyperref is needed for \noexpand
     \svnnolinkurl.}}%
    {\hyper@normalise\svnnolinkurl@}%
}%
\def\svnnolinkurl@#1{\Hurl{#1}}%
%    \end{macrocode}
% \end{macro}

% \begin{macro}{\svn@getfilename}[1]{URL}
% This macro expands the content using the temporary macro and sets it in front
% of the \csi{svn@getfilename} sub-macro together with |/{}| to make sure the
% macro does not break at values without directories. A |\relax| is used as
% end marker.
%    \begin{macrocode}
\def\svn@getfilename#1{%
  \begingroup
    \edef\svn@temp{#1}%
    \expandafter\@svn@getfilename\svn@temp/{}\relax
}%
%    \end{macrocode}
% \end{macro}

% \begin{macro}{\@svn@getfilename}[2]{URL part before first slash}{URL part after
% first slash}
% Splits the content at the first slash (|/|) and checks if the remainder is
% empty. If so the end marker got reached and the part before the slash is the
% filename which is returned. Otherwise the macro recursively calls itself to
% split the remainder.
%    \begin{macrocode}
\def\@svn@getfilename#1/#2\relax{%
    \svn@ifempty{#2}%
      {\endgroup\gdef\svnfilefname{#1}}%
      {\@svn@getfilename#2\relax}%
}%
%    \end{macrocode}
% \end{macro}

% \subsection{Auxiliary file generation and read-back}
%
% Reread output from last compile run if it exists.
%    \begin{macrocode}
\@input{\jobname .svn}
%    \end{macrocode}

% \begin{macro}{\svn@checkwrite}
% Checks if .svn file is already open and if not opens it. This makes sure that
% the file is only created if really needed. The macro is only needed once, so
% it's redefines itself to |\relax| at the end.
%    \begin{macrocode}
\def\svn@checkwrite{%
 \@ifundefined{svn@write}{%
   \newwrite\svn@write
   \immediate\openout\svn@write=\jobname.svn\relax%
   \immediate\write\svn@write{\@percentchar\space SVN Keyword cache}%
   %\immediate\write\svn@write{\noexpand\makeatletter}%
 }{}%
 \let\svn@checkwrite=\relax
}
%    \end{macrocode}
% \end{macro}
%
% \begin{macro}{\svn@writeaux}
% This macro writes the |.svn| auxiliary file and is called from a
% |\AtEndDocument| macro later on.
%    \begin{macrocode}
{\catcode`\&=12
\gdef\@ampersamchar{&}
}
\def\svn@writeaux{%
%    \end{macrocode}
% We first check if we have something to save. Revision, date and author must be
% non-empty. This suppresses the auxiliary file if the user doesn't use the
% appropriate macros but other provided by this package.
%    \begin{macrocode}
  \if@filesw \ifx\@svn@rev\empty\else \ifnum\@svn@rev=0\else
  \ifx\@svn@date\empty\else \ifx\@svn@author\empty\else
%    \end{macrocode}
% Remove all files which do not have a revision number from list:
%    \begin{macrocode}
    \if@svnmulti@groups
      \ifx\@svng@@files\empty\else
      \begingroup
        \def\svn@tmplist{}%
        \@for\svn@temp:=\@svng@@files\do{%
        \expandafter\ifx\csname @svng@\svn@temp @rev\endcsname\relax\else
          \edef\svn@tmplist{\svn@tmplist,\svn@temp}%
        \fi
        }%
        \xdef\@svng@@files{\expandafter\@gobble\svn@tmplist\empty}%
      \endgroup
      \fi
    \fi
%    \end{macrocode}
% Write document global values:
%    \begin{macrocode}
    \svn@checkwrite
    \immediate\write\svn@write{^^J%
      \@percentchar\space Global values:^^J%
      \noexpand\def\noexpand\svnrev{\@svn@rev}^^J%
      \noexpand\def\noexpand\svndate{\@svn@date}^^J%
      \noexpand\def\noexpand\svnauthor{\@svn@author}^^J%
      \noexpand\def\noexpand\svnyear{\@svn@year}^^J%
      \noexpand\def\noexpand\svnmonth{\@svn@month}^^J%
      \noexpand\def\noexpand\svnday{\@svn@day}^^J%
      \noexpand\def\noexpand\svnhour{\@svn@hour}^^J%
      \noexpand\def\noexpand\svnminute{\@svn@minute}^^J%
      \noexpand\def\noexpand\svnsecond{\@svn@second}^^J%
      \noexpand\def\noexpand\svntimezonehour{\@svn@timezonehour}^^J%
      \noexpand\def\noexpand\svntimezoneminute{\@svn@timezoneminute}^^J%
      \noexpand\svn@gdefverb\noexpand\svnurl{\@svn@url}^^J%
      \noexpand\svn@gdefverb\noexpand\svnfname{\@svn@fname}^^J%
    }%
    \if@svnmulti@groups
      \immediate\write\svn@write{%
        \noexpand\def\noexpand\svng@@files{\@svng@@files}^^J%
      }%
    \else
      \immediate\write\svn@write{^^J}%
    \fi
%    \end{macrocode}
% Write keyword group values if groups were specified:
%    \begin{macrocode}
    \if@svnmulti@groups
      \ifx\svn@glist\empty\else
        \begingroup
%    \end{macrocode}
% Write table of revisions file if \op{table} option was used.
%    \begin{macrocode}
          \if@svnmulti@table
            \newwrite\svn@svtwrite
            \immediate\openout\svn@svtwrite=\jobname.\svn@svt\relax
            \immediate\write\svn@svtwrite{%
              \noexpand\svnbeforetable^^J%
              \noexpand\svntable%
            }%
            \begingroup
              \svn@writeglobalrow{}%
              \ifx\@svng@@files\@undefined\else
                \ifx\@svng@@files\relax\else
                  \@for\svn@file:=\@svng@@files\do{%
                    \svn@writefilerow{\svn@file}%
                  }%
                \fi
              \fi
            \endgroup
          \fi
          \immediate\write\svn@write{^^J%
            \@percentchar\space SVN File Groups: \svn@glist
          }%
%    \end{macrocode}
% For every activated keyword group check if any keywords got recorded
% and write to the output file:
%    \begin{macrocode}
          \@for\svn@g:=\svn@glist\do{%
            \expandafter\ifx\csname @svng@\svn@g @rev\endcsname\relax\else
            \expandafter\ifnum\csname @svng@\svn@g @rev\endcsname>-1\relax
            \def\svn@writekw##1{%
              \noexpand\@namedef{svng@\svn@g @##1}%
              {\csname @svng@\svn@g @##1\endcsname}^^J%
            }%
            \immediate\write\svn@write{%
              \@percentchar\space\svn@g^^J%
              \svn@writekw{rev}%
              \svn@writekw{date}%
              \svn@writekw{author}%
              \svn@writekw{year}%
              \svn@writekw{month}%
              \svn@writekw{day}%
              \svn@writekw{hour}%
              \svn@writekw{minute}%
              \svn@writekw{second}%
              \svn@writekw{timezonehour}%
              \svn@writekw{timezoneminute}%
              \svn@writekw{files}%
              \noexpand\svn@namegdefverb{svng@\svn@g @url}%
                  {\csname @svng@\svn@g @url\endcsname}^^J%
              \noexpand\svn@namegdefverb{svng@\svn@g @fname}%
                  {\csname @svng@\svn@g @fname\endcsname}^^J%
            }%
            \if@svnmulti@table
              \begingroup
              \svn@writegrouprow{\svn@g}%
              \expandafter\let\expandafter
                \svn@temp\csname @svng@\svn@g @files\endcsname
              \ifx\svn@temp\relax\else
                \@for\svn@file:=\svn@temp\do{%
                  \svn@writefilerow{\svn@file}%
                }%
              \fi
              \endgroup
            \fi
            \fi\fi
          }%
          \if@svnmulti@table
            \immediate\write\svn@svtwrite{%
              \noexpand\endsvntable^^J%
              \noexpand\svnaftertable
            }%
            \immediate\closeout\svn@svtwrite%
          \fi
        \endgroup
      \fi
    \fi
%    \end{macrocode}
% Finally close output file:
%    \begin{macrocode}
    \immediate\closeout\svn@write%
  \fi\fi\fi\fi\fi
}
%    \end{macrocode}
% \end{macro}
%
% At the end of document the values are written to the auxiliary file.
%    \begin{macrocode}
\AtEndDocument{%
  \svn@writeaux
}
%    \end{macrocode}
%
% \subsection{Backward compatibility wrapper \texttt{svnkw.sty}}
% For backward compatibility a wrapper file with the old package name |svnkw| is
% provided. Newer documents should use the name \svnmulti.
% \setcounter{CodelineNo}{0}
% \iffalse
%</package>
%<*wrapper>
% \fi
%    \begin{macrocode}
\NeedsTeXFormat{LaTeX2e}[1999/12/01]
\ProvidesPackage{svnkw}
[\filedate\space\fileversion\space Backward compatibility wrapper for svn-multi]
\PackageWarning{svnkw}{The package 'svnkw' got renamed to 'svn-multi'
and is now only a backward compatibility wrapper which loads 'svn-multi'.
Please adjust your document preamble to use the new name.}
\RequirePackage{svn-multi}[\filedate]
%    \end{macrocode}
% \iffalse
%</wrapper>
% \fi
%
% \Finale
\endinput
