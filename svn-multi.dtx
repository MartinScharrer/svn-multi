% \iffalse meta-comment
% Copyright (C) 2006-2009 by Martin Scharrer <martin@scharrer-online.de>
% http://www.scharrer-online.de/latex/
% -----------------------------------------------------------------
%
% This work may be distributed and/or modified under the
% conditions of the LaTeX Project Public License, either version 1.3
% of this license or (at your option) any later version.
% The latest version of this license is in
%   http://www.latex-project.org/lppl.txt
% and version 1.3 or later is part of all distributions of LaTeX
% version 2005/12/01 or later.
%
% This work has the LPPL maintenance status `maintained'.
%
% The Current Maintainer of this work is Martin Scharrer.
%
% This work consists of the files svn-multi.dtx and svn-multi.ins
% and the derived files svn-multi.sty and svnkw.sty.
% $Id$
% \fi
% \iffalse
%<*package|driver|wrapper>
\def\filedate$#1: #2 #3 #4-#5-#6 #7 ${%
 \year#4\month#5\day#6\relax
 \def\filedate{#4/#5/#6}%
}
\filedate$Id$
\def\fileversion{v2.0}
%</package|driver|wrapper>
%<*driver>
\ProvidesFile{svn-multi.dtx}
 [\filedate\space\fileversion\space SVN Keywords for multi-file LaTeX documents]

\documentclass{ltxdoc}
\usepackage{svn-multi}
\usepackage{ifpdf}
\ifpdf
    % use hypdoc if you have it, hyperref else
    %\usepackage[hyperfootnotes=false]{hyperref}
    \usepackage{hypdoc}
\else\let\url=\texttt\fi
\usepackage{xspace}
\newcommand{\ie}{i.e.\@\xspace}
\newcommand{\eg}{e.g.\@\xspace}

\iftrue % crossreference of macros in documentation
\begingroup
\catcode`\*=0
\catcode`\\=12
*gdef*removeslash\#1 #2END{#1#2}
*endgroup

\makeatletter
% link \cs to macro definitions
\let\origmacro\macro
\let\origendmacro\endmacro
\let\origStopEventually\StopEventually
\let\origPrintDescribeMacro\PrintDescribeMacro
\def\DescribeMacro#1{
    \leavevmode \@bsphack
    \marginpar{%
    \raggedleft\texttt{\strut\textbackslash #1}}%
    \raisebox{11pt}[11pt]{\hypertarget{Desc@#1}{}}%
    \SpecialUsageIndex {#1}\@esphack \ignorespaces
}
\def\DescribeOption#1{
    \leavevmode \@bsphack
    \marginpar{%
    \raggedleft\texttt{\strut #1}}%
    \raisebox{11pt}[11pt]{\hypertarget{option@#1}{}}%
    \@esphack \ignorespaces
}
\let\DescribeScript\DescribeOption

\renewenvironment{macro}[1]{%
\edef\macroname{\expandafter\removeslash\detokenize{#1}END}
\hypertarget{\macroname}{}%
\origmacro{#1}}%
{\origendmacro}

\DeclareRobustCommand{\cs}[1]{\hyperlink{Desc@#1}{\texttt{\textbackslash#1}}}
\DeclareRobustCommand{\css}[1]{\texttt{\textbackslash#1}}
\DeclareRobustCommand{\op}[1]{\hyperlink{option@#1}{\texttt{#1}}}

\def\StopEventually#1{\origStopEventually{#1}%
\DeclareRobustCommand{\cs}[1]{\hyperlink{##1}{\texttt{\textbackslash##1}}}%
}

\fi

\EnableCrossrefs
%\DisableCrossrefs
\CodelineIndex
%\PageIndex
\RecordChanges
%\OnlyDescription
\begin{document}
  \DocInput{svn-multi.dtx}%
  \PrintChanges
  \clearpage
  \PrintIndex
\end{document}
%</driver>
%<*package>
% \fi
%
% \CheckSum{0}
%
% {\makeatother
% \CharacterTable
%  {Upper-case    \A\B\C\D\E\F\G\H\I\J\K\L\M\N\O\P\Q\R\S\T\U\V\W\X\Y\Z
%   Lower-case    \a\b\c\d\e\f\g\h\i\j\k\l\m\n\o\p\q\r\s\t\u\v\w\x\y\z
%   Digits        \0\1\2\3\4\5\6\7\8\9
%   Exclamation   \!     Double quote  \"     Hash (number) \#
%   Dollar        \$     Percent       \%     Ampersand     \&
%   Acute accent  \'     Left paren    \(     Right paren   \)
%   Asterisk      \*     Plus          \+     Comma         \,
%   Minus         \-     Point         \.     Solidus       \/
%   Colon         \:     Semicolon     \;     Less than     \<
%   Equals        \=     Greater than  \>     Question mark \?
%   Commercial at \@     Left bracket  \[     Backslash     \\
%   Right bracket \]     Circumflex    \^     Underscore    \_
%   Grave accent  \`     Left brace    \{     Vertical bar  \|
%   Right brace   \}     Tilde         \~}
% }
% \changes{v1.0}{2006/05/27}{Initial version}
% \changes{v1.1}{2006/06/08}{Added macros to extract and typeset date/time
% informations. Added macros to set and typeset main URL or filename.}
% \changes{v1.2}{2007/06/22}{Renamed packet from \texttt{svnkw} to
% \texttt{svn-multi} to match CTAN directory. Wrapper file
% \texttt{svnkw.sty} is provided for backward compatibility.}
% \changes{v1.3}{2007/07/01}{Added verbatim support. Keywords can now contain
% special character like \texttt{\_ \^{} \$ \% \& \textbackslash}. Rewrote
% keyword check macros to work with verbatim code. \texttt{\textbackslash
% nofiles} is now obeyed.}
% \changes{v1.3a}{2007/07/10}{Fixed issue with unwanted spaces generated by
% \cs{svnid}, \cs{svnidlong} and \cs{svnkwsave}, \eg when used in a file which
% is included with \css{input}}
% \changes{v1.3b}{2008/12/03}{Changed the way catcodes are modified to be
% compatible with the french option of the babel package or other packages which
% modify the list of special characters.}
% \changes{v1.4}{2009/02/27}{Added support for timezones with non-zero minute
% part, \eg +0530.}
% \changes{v1.5}{2009/02/28}{Added \css{today}-style macros \cs{svntoday} and
% \cs{svnfiletoday}.}
% \changes{v2.0}{2009/02/29}{File groups for subversion keywords got
% introduced.}
%
% \GetFileInfo{svn-multi.sty}
%
% \DoNotIndex{\newcommand,\newenvironment,\AtBeginDocument,\AtEndDocument}
% \DoNotIndex{\def,\let,\edef,\xdef,\item,\space,\write,\jobname,\relax,\!}
% \DoNotIndex{\closeout,\csname,\DeclareRobustCommand,\else,\empty,\newwrite}
% \DoNotIndex{\endcsname,\expandafter,\fi,\Hurl,\hyper@normalise,\@ifnextchar}
% \DoNotIndex{\ifnum,\@ifundefined,\ifx,\immediate,\InputIfFileExists,\ }
% \DoNotIndex{\newcount,\noexpand,\openout,\PackageWarning,\@percentchar}
% \DoNotIndex{\@sanitize,\@makeother,\@iwsvn,\%,\_,\&,\^,\$,\#,\ ,\\,\if@filesw}
% \DoNotIndex{\gdef,\begingroup,\endgroup,\catcode}
%
%
% \title{The \textsf{svn-multi} package\thanks{This document
%   corresponds to \textsf{svn-multi}~\fileversion, dated \filedate.}
%   \\also known as \textsf{svnkw}}
% \author{Martin Scharrer \\ \url{martin@scharrer-online.de} \\
% \url{http://www.scharrer-online.de/latex/svn-multi}}
%
% \ifpdf
% \hypersetup{
%	pdfauthor={Martin Scharrer <martin@scharrer-online.de>},
%	pdftitle={The svn-multi package also known as svnkw},
%	pdfsubject={Documentation of LaTeX package svn-multi which allows the
%	typesetting of Subversion keywords in multi-file LaTeX documents},
%	pdfkeywords={svn-multi, svnkw, LaTeX, Subversion, keywords, Version
%	Control, Id}
% }
% \fi
% \maketitle
%
% \section{Package naming}
% The authors first choice for this package was |svnkw| but the CTAN
% maintainer suggested a more descriptive name and put the package in the
% |svn-multi|\footnote{CTAN:
% \url{http://www.ctan.org/tex-archive/macros/latex/contrib/svn-multi/}}
% directory. Therefore the style file got renamed to |svn-multi|, but a
% |svnkw| dummy style file which loads the new package is still provided
% for backward compatibility.
%
% \section{Introduction}
% This package lets you typeset keywords of the version control system
% Subversion\footnote {Subversion homepage: \url{http://subversion.tigris.org/}}
% (svn), which is the successor of the popular CVS, inside your \LaTeX{} files
% anywhere you like.  Unlike the package |svn|\footnote{CTAN:
% \url{http://www.ctan.org/tex-archive/macros/latex/contrib/svn/}} the usage of
% multiple files for one \LaTeX{} document is well supported. The package
% acquires the keywords of the last changed file and provides them to the user
% through macros. The package has to read all keywords of all files first and
% writes the most recent values in an auxiliary file with an `|.svn|' extension.
% This file is read back at the next \LaTeX{} run which introduces a delay like
% by the table of contents. The standard \LaTeX{} switch |\nofiles| can be used
% to suppress the file generation. Macros to typeset the keywords of the current
% |\include|-d or |\input|-ed \LaTeX{} file are also provided.
%
% \subsection{Scope of Keywords}
% This package provides the Subversion keyword data in several different scopes:
% document-global, file-local and, new with v2.0, by group.
%
% \subsubsection*{Document global}
% The normal macros, \eg \cs{svnrev}, return the latest version control
% informations (keyword data) for the whole multi-file document, \ie the
% information of the latest changed file of the document. To collect, sort and
% provide this informations is the main functionality of this package.
%
% \subsubsection*{Local to current file}
% There are also other macros, \eg \cs{svnfilerev}, which return the version
% control informations of the current file, \ie the file they are used in. It is
% assumed here that every file using this macros calls first either a \cs{svnid}
% or \cs{svnidlong} macro or both.  See section~\ref{sec:usage:id} for more
% details about the id macros.  Please note that the file-local macros
% technically actually return the \emph{last registered} informations from the
% last \cs{svnid} or \cs{svnidlong} (or \cs{svnkwsave}) even when these were in
% a different, previous file.  This will cause wrong results if they are used in
% a file before or without any id macros.
%
% \subsubsection*{Groups}
% Version 2.0 introduces the concept of groups. Several files\footnote{actually
% several id macros} of a multi-file \LaTeX\ document can be grouped together
% and the latest version control informations of all files of a group is
% provided by macros. This works in the same way as the global macros mentioned
% above but only with the files in the group. It can also be seen from the other
% side: the macros are local like the file-local macros mentioned above but for
% all files of the group, not only the current one.
%
% This groups could also be called \textit{file groups}, \textit{keyword groups}
% or, like in programming languages, \textit{namespaces}. In this manual they
% will be reference as simple \textit{groups} most the time. In places where
% they could be confused with \TeX\ groups (|{ }|, |\begingroup| |\endgroup|),
% \eg ``in the current group'' or ``group local'', they will be called
% \textit{keyword groups}.
%
% There is no limitation (besides internal \LaTeX\ resource limits) for the
% number of different groups. The files of one group do not have to be included
% in a row but can be included everywhere in the document. The version control
% informations of the current group can be typeset with macros like
% \cs{svncgrev}.  Also, a general but less robust macro \cs{svng} is provided to
% access others groups by name everywhere in the document. To avoid some macro
% robustness problems the current group can be changed locally for the output
% macros using \cs{svnsetcg}.
%
% See section~\ref{sec:group} for further details and usage instructions on
% group macros.
%
% \section{Basic Usage} The version control informations are provided by
% Subversion keywords which first need to be read in by dedicated macros and can
% then be typeset using different macros.
%
% \subsection{Package Options} Since v2.0 this package provides some options to
% allow to activate only a certain subset of its functionality, \eg to avoid
% problems with other packages or save \TeX\ memory. For backwards compatibility
% to pre-2.0 package versions all old features are active by default and all new
% features are deactivated to save a little of \TeX\ memory.
%
% The available options are:\par
%
% \DescribeOption{old}
% Activate only pre-v2.0 features.\medskip
%
% \DescribeOption{all}
% Activate all features of the package.\medskip
%
% \DescribeOption{verbatim}
% \DescribeOption{noverbatim}
% Activate (default) or deactivate the verbatim mode of the keyword parser
% macros. Normally verbatim mode is very much wanted to support strange
% filenames of URLs, but This options gives the user a possibility to disable
% verbatim, \eg for trouble shooting.  Please note that verbatim mode is needed
% in order to make |svn-multi| work with some packages, like |babel| with the
% |french| option.
% \medskip
%
% \DescribeOption{external}
% \DescribeOption{noexternal}
% Activate or deactivate (default) the support for keywords from external files
% described in section~\ref{sec:external}. An old |.svn| file should be removed
% when this feature is deactivated to avoid \textit{undefined macro} errors
% caused by macros placed there by this feature.  Please note that the
% \op{noexternal} option causes \op{nographics} and \op{nopgfimages}.
% \medskip
%
% \DescribeOption{groups}
% \DescribeOption{nogroups}
% Activate or deactivate (default) the keyword groups feature described in
% section~\ref{sec:group}.\\[\baselineskip]
%
% \DescribeOption{filesasgroups}
% \DescribeOption{nofilesasgroups}
% Activate or deactivate (default) the declaration of all input files as
% independent groups so that there keyword informations can be typeset inside
% other files. The group name is the file path (`|subdir/file.tex|'). The files
% are still added to the current group defined by \cs{svngroup}.\par This can be
% disabled and re-enabled using \cs{svnfilesasgroupfalse} and
% \cs{svnfilesasgrouptrue} inside the document.
%
% \DescribeOption{graphics}
% \DescribeOption{nographics}
% This option (default: off) allows to automatically declare all images included
% using |\includegraphics| from the |graphics|/|graphicx| package as external
% files (see section~\ref{sec:external}). The options \op{external} and
% \op{autoload} are activated by the |graphics| option so that the produced
% |.svx| files are loaded automatically.  An \op{noautoload} option after
% |graphics| will deactivate this, but then an \cs{svnexternal} macro must be
% included in all \LaTeX\ files which should take the image revisions into
% account.\par The |graphics| package is loaded if this option is active. If
% this package is needed with some special options it should be loaded by the
% \LaTeX\ document before |svn-multi|.  Please note that this feature needs to
% tie itself into the |graphics| package and might fail if the internal
% structure of this package changes in future versions.
% \medskip
%
% \DescribeOption{pgfimages}
% \DescribeOption{nopgfimages}
% Same like the \op{graphics} option but for the |pgf| package (2008/01/15 or
% later) with the |pgfuseimage| and |pgfimage| macros. Please also see the notes
% about package loading and ties mentioned above.
% \medskip
%
% \DescribeOption{autoload}
% \DescribeOption{noautoload}
% Activate or deactivate (default) automatic loading of corresponding |.svx|
% files at the begin of files included using |\input| or |\import|. This avoids
% the need of putting an \cs{svnexternal} macro in every file just to load the
% |.svx| files created automatically by the \op{graphics} option. The option
% \op{external} is activated by \op{autoload}.
% \medskip
%
% \DescribeOption{table}
% \DescribeOption{notable}
% Activate or deactivate (default) a \cs{tableofrevisions} macro which shows the
% revisions of all files and groups. This needs \op{groups} to work which is
% activated with \op{table}.
%
% \subsection{Including of the Subversion keywords}\label{sec:usage:id}
% To include your Subversion Id keywords use \cs{svnid} or \cs{svnidlong}.
% These macros should be written very early in each file, \ie in the preamble of
% the main document soon after |\documentclass| and |\usepackage{svn-multi}| and
% as first in \emph{every} |\include|d subfile before the |\chapter| macro. They
% do not create any output.  See section~\ref{sec:kwaccess} to learn how to
% typeset the keyword values.
%
% \DescribeMacro{svnid}
% Macro for the svn Id keyword.  Write the macro as |\svnid{$||Id$}| into your
% \LaTeX{} files. A trailing colon with spaces after the |Id| is also valid but
% \textbf{everything else} except a valid Subversion string will cause a \TeX{}
% parse error.  Don't forget to set the subversion property |svn:keywords| of
% the files to at least `|Id|'. Subversion will expand it at the next commit.
% Please note that because the value is read verbatim the macro should exactly
% be written like above. Spaces, newlines or comments between |\svnid| and the
% \{ will lead to \TeX{} parse errors.
%
% \DescribeMacro{svnidlong}
% Macro for a ``long Id''.  Saves similar values like in `|Id|' but from the
% keywords |HeadURL|, |LastChangedDate|, |LastChangedRevision| and
% |LastChangedBy|.  The usage of \cs{svnid} or |\svnidlong| is a matter of
% taste. The second is more readable inside the code and results in a nicer date
% and a full URL, not only the filename. Both can also be used together.
%
% Write this macro like this (order of arguments not meaningfull)\\[2ex]
% |\svnidlong|\\
% |{$||HeadURL$}|\\
% |{$||LastChangedDate$}|\\
% |{$||LastChangedRevision$}|\\
% |{$||LastChangedBy$}|\\[2ex]
% in your files and set the subversion property |svn:keywords| of them
% to\\`|HeadURL LastChangedDate LastChangedRevision LastChangedBy|'.
%
% Please note that the arguments are read verbatim. Special precaution are taken
% to allow spaces, newlines and comments direct after the |\svnidlong| and after
% each of the four arguments, just in case someone need this.  In fact
% everything not inside braces \{ \} is ignored.
%
% The two macros above are sufficient for the operation of |svn-multi| but the
% following macros are provided for additional or alternative keyword handling.
%
% \DescribeMacro{svn}
% \DescribeMacro{svn*}
% This macro let you typeset svn keywords directly. The only argument is the svn
% keyword in the usual dollars. The dollars will be stripped and the rest is
% typeset as normal text. The star version strips also the space before the last
% dollar.  This macro alone was the very first version of |svnkw| and is still
% included for fast and simple keyword typesetting.
%
% \DescribeMacro{svnkwsave}
% This macro lets you include and save any keyword you like. The syntax is
% |\svnkwsave{$|\meta{keyword}|$}|. The keyword can be already expanded or not
% (no value and only ``|:|'' or nothing after the key name). This macro is also
% used internally and does not create any output.  Please note that the argument
% is read verbatim and that there should be no space between the macro and the
% argument's left brace.
%
% \subsection{Typesetting the keyword values}\label{sec:kwaccess}
% The following macros can be used to typeset the keyword values anywhere in the
% document. Please note that note all \LaTeX{} fonts have all special
% characters, \eg `\_' is not provided in the standard roman font. To proper
% typeset filename and URL containing these letters you can use either texttyper
% font (|\texttt|) or use |{\urlstyle{rm}\svnnolinkurl{...}}| which needs the
% |hyperref| package.
%
% Like already mentioned |svn-multi| knows two groups of keywords. The first
% group contains of the keywords for the whole document which hold the values of
% the most recent committed file, the second contains of the \emph{current} or
% \emph{file local} keywords, \eg the keywords of the current file.
%
% \DescribeMacro{svnrev}
% \DescribeMacro{svndate}
% \DescribeMacro{svnauthor}
% These macros hold the keyword values of the whole document, \ie of the most
% recent revision. They can be used everywhere in every file of the \LaTeX{}
% document, after |\usepackage{svn}| of course. Please see
% section~\ref{sec:date} how to typeset parts of the date.
%
% \DescribeMacro{svnfilerev}
% \DescribeMacro{svnfiledate}
% \DescribeMacro{svnfileauthor}
% These macros hold the keyword values of the current \LaTeX{} file, but only if
% it contains a \cs{svnid} or \cs{svnidlong} macro. Otherwise the macros hold
% still the values of the last file. Please see section~\ref{sec:date} how to
% typeset parts of the date. See \cs{svnkw} below for all other keywords.
%
% \DescribeMacro{svnmainurl}
% \DescribeMacro{svnmainfilename}
% The macro \cs{svnmainurl} and \cs{svnmainfilename} hold the URL and the
% filename of the main \LaTeX{file} as long the keywords |HeadURL| or |Id| were
% used in it, respectively.  These can be used to typeset this information
% anywhere in the document which might be more descriptive as the name of the
% current file (which can be typeset with \cs{svnkw}|{HeadURL}| or
% \cs{svnkw}|{Filename}| after \cs{svnid} or \cs{svnidlong}, respectively).
%
% \DescribeMacro{svnsetmainfile}
% This will declare the current file as the main LaTeX file by defining the
% above macros. It will automatically be called at the end of the preamble so
% the user normally doesn't have to use it by him- or herself as long it isn't
% needed in the preamble.\par Please note that this macro changes the
% definition of \cs{svnmainurl} and \cs{svnmainfilename} directly without going
% over the auxiliary file. Calling it in several files will make this two macros
% inconsistent.
%
% \DescribeMacro{svnkw}
% All keywords saved with \cs{svnid}, \cs{svnidlong} or \cs{svnkwsave} can be
% typeset by this macro which is a holdover from a very early version of this
% package when multiple files where not supported.  It takes one argument which
% must be a subversion keyword name. It then returns the current value of this
% keyword or nothing (|\relax|) when the keyword was not set yet.
% Examples:\\
% \indent\indent |\textsl{Revision: \svnkw{Revision}}|\\
% \indent\indent |URL: \url{\svnkw{HeadURL}}|\\
% In the second example |\url| (|hyperref| package) is used to add a hyperlink
% and to avoid problems with underscores (|_|) inside the URL.  |svn-multi| is
% also providing a macro \cs{svnnolinkurl} which works like |\url| but doesn't
% adds an hyperlink. See the description of this macro for more details.
%
% If the given keyword doesn't exists a package warning is given to allow
% spelling errors to be tracked down. This doesn't work well when \cs{svnkw} is
% used inside |\url|. In this case the warning code will be typeset(!) verbatim
% into the document by |\url|.
%
% \DescribeMacro{svnkwdef}
% This macro is used to define the keyword values, \ie
% \cs{svnkwdef}\marg{keyword}\marg{value}. This is normally only called
% internally but could be used by the user to override single keywords.  The
% values can then be typeset by \cs{svnkw}.  Note that this macro has no
% influence on the calculation of the latest revision.
%
% Note that for \cs{svnkw} and \cs{svnkwdef} all different names for one keyword
% are valid and result in the access of the same variable. So \eg subversion
% treats |Rev|, |Revision| and |LastChangedRev| the same way and so does this
% macros. You can \eg say |\svnkwdef{Rev}{123}| and then typeset it with
% |\svnkw{Revision}| or |\svnkw{LastChangedRev}| if you like.
%
% \subsubsection{Accessing Date Values}\label{sec:date}
% \DescribeMacro{svnfileyear}
% \DescribeMacro{svnfilemonth}
% \DescribeMacro{svnfileday}
% \DescribeMacro{svnfilehour}
% Whenever the date information is read, \ie by
% \cs{svnkwsave}|{LastChangedDate}| \cs{svnkwsave}|{Date}|, \cs{svnidlong} or
% \cs{svnid}, the following macros are set to the appropriate date parts for the
% current file (the |\svnfile...| versions) and for the whole document.
%
% \iffalse Put here and not above to avoid 'to many floats' error. \fi
% \DescribeMacro{svnfileminute}
% \DescribeMacro{svnfilesecond}
% \DescribeMacro{svnfiletimezone}
% \DescribeMacro{svnfiletimezonehour}
% \DescribeMacro{svnfiletimezoneminute}
% Please note that the hour and timezone are dependend on the keyword which
% defines the date informations. The hour will be in UTC aka Zulu-time, \ie
% timezone +0000, when the date comes from the |Id| keyword.
% \iffalse Put here and not above to avoid 'to many floats' error. \fi
% \DescribeMacro{svnyear}
% \DescribeMacro{svnmonth}
% \DescribeMacro{svnday}
% \DescribeMacro{svnhour}
% Otherwise the hour and timezone will be in local time.
% To avoid confusion the |Id| and |Date|/|LastChangedDate| keywords, \eg
% \cs{svnid} and \cs{svnidlong}, should not be intermixed and/or the timezone
% should always be typeset together with the time.
%
% \iffalse Put here and not above to avoid 'to many floats' error. \fi
% \DescribeMacro{svnminute}
% \DescribeMacro{svnsecond}
% \DescribeMacro{svntimezone}
% \DescribeMacro{svntimezonehour}
% \DescribeMacro{svntimezoneminute}
% Starting with v1.4 of |svn-multi| the timezone macros return the full
% timezone, \ie sign, hour and minute part, \eg |+0100|, not only the sign and
% hour. The new macros % \cs{svntimezonehour}/\cs{svnfiletimezonehour} and
% \cs{svntimezoneminute}/\linebreak[3]\cs{svnfiletimezoneminute} can be used to
% access only the hour including sign or the minute part, respectively.
%
% Older versions of this manual assumed the minute part as always |00| and
% suggested to add it manually if needed: |\svnfiletimezone00| or
% |\svntimezone00|.  In order not to ``break'' documents which followed this
% suggestion this two macros now remove a trailing |00| if present.  However,
% this can be a problem when they are used inside an argument of another macro.
% One solution for this is to redefine them without the |00| removal part:\\
% \begingroup\small
% |renewcommand{\svntimezone}{\svntimezonehour\svntimezoneminute}|\\
% |renewcommand{\svnfiletimezone}{\svnfiletimezonehour\svnfiletimezoneminute}|
% \endgroup\par
% To revert to the old (pre-v1.4) definition use:\\
% \begingroup\small
% |\renewcommand{\svntimezone}{\svntimezonehour}|\\
% |\renewcommand{\svnfiletimezone}{\svnfiletimezonehour}|
% \endgroup
% \vspace{1ex}
%
% \DescribeMacro{svntime}
% \DescribeMacro{svnfiletime}
% \DescribeMacro{svncgtime}
% This macros return the time part of the date only and simply return the
% corresponding hour, minute and second macros with a colon as separator.
% \vspace{2\baselineskip}
%
% \DescribeMacro{svnpdfdate}
% Returns the last changed date of the whole document in a format needed for
% |\pdfinfo|. Can be used like this:\\
% \hbox{}\hfill|\pdfinfo{ /CreationDate (D:\svnpdfdate) }|\hfill\hbox{}\\
% to set the PDF creation date to the last changed date if you use |pdflatex| to
% compile your \LaTeX{} document.
%
% \DescribeMacro{svntoday}
% \DescribeMacro{svnfiletoday}
% \DescribeMacro{svncgtoday}
% These macros typeset the document-global, current-file or current-group date,
% respectively, using the format of |\today| which depends on the used language.
% To adjust the language of your document use the |babel| package.
%
% \subsection{Using full author names}
% If you like to have the full author\footnote{This means subversion authors,
% \eg the persons who commit changes into the svn repository.} names, not only
% the usernames, in your document you can use the following macros. First you
% have to register all authors of the document with \cs{svnRegisterAuthor} and
% then you can write \eg |\svnFullAuthor{\svnauthor}| or
% |\svnFullAuthor{\svnfileauthor}|.
%
% \DescribeMacro{svnRegisterAuthor}
% The usage is \cs{svnRegisterAuthor}\marg{username}\marg{full name} which
% registers \meta{full name} as full name for \meta{username}.
%
% \DescribeMacro{svnFullAuthor}
% \DescribeMacro{svnFullAuthor*}
% Takes the username as argument and returns the full name if it was registered
% first with \cs{svnRegisterAuthor}, otherwise it returns the given username.
% The star version returns the username in parentheses after the full name.
%
% \subsection{Using full revision names}
% Like the author's also revision names/tags can be registered and used later.
% These macros were implemented on user request and have the drawback that you
% have to guess the next revision number of your document in order to get
% correct results when you like to tag the to-be-checked-in revision.  Please
% note that this has nothing to do with the normal subversion tagging.
%
% \DescribeMacro{svnRegisterRevision}
% The usage is \cs{svnRegisterRevision}\marg{revision number}\marg{tag name}
% which registers \meta{tag name} as tag name for \meta{revision number}.
%
% \DescribeMacro{svnFullRevision}
% \DescribeMacro{svnFullRevision*}
% Takes a \marg{revision number} (\ie \cs{svnrev}, \cs{svnfilerev} or a number)
% as argument and returns the full name if it was registered first with
% \cs{svnRegisterRevision}, otherwise it returns ``Revision \meta{revision
% number}''.  The star version returns also the revision number leaded by `r' in
% parentheses after the tag name, \eg |Name (r123)|.
%
% \subsection{Verbatim URLs with and without hyperlinks}
% \DescribeMacro{svnnolinkurl}
% This macro allows you to write |\svnnolinkurl{\svnkw{HeadURL}}| and get the
% Head URL typeset verbatim. However |\url{|\cs{svnkw}|{HeadURL}}| (|hyperref|
% package) gives you the same result with hyperlinked. Both macros require the
% |hyperref| package which is not automatically loaded by |svn-multi|.  Please
% load it manually when you like to use \cs{svnnolinkurl}.
%
% Since v1.3 all keywords are read and typeset verbatim so this macro isn't this
% important anymore. However together with |hyperref|s |urlstyle| macro it can
% be used to have keyword values with special characters in roman font, which
% normaly doesn't hold letters like `\_'.
%
% Please note that you can't use |hyperref|s |\nolinkurl| because it won't
% expand \cs{svnkw}.
%
% \subsection{File Groups}\label{sec:group}
% Starting with v2.0 files can be grouped together and the keyword values of the
% latest revision of a group can be accessed. Use the |groups| option to
% activate these macros.
%
% \noindent
% \DescribeMacro{svngroup}\hspace{-.75em}\marg{group name}\\
% This macro declares all following files (actually only following \cs{svnid},
% \cs{svnidlong} and \cs{svnkwsave} macros) until the next \cs{svngroup} as part
% of the given keyword group. It can be placed inside the main file before some
% |\include|/|\input| macros or inside sub-files before the id macros, \ie
% direct at the start of the file. Please note that the group will then swap
% over to the following file. However, the group can be closed manually using an
% empty |\svngroup{}|, \eg at the end of a file.\par The changes done by this
% macro are \TeX\ global, \ie there can't be caught using \TeX\ groups (|{ }|).
%
% \smallskip
% \DescribeMacro{thesvngroup}
% Returns the name of the current keyword group.
%
% \smallskip
% \DescribeMacro{svncgrev}
% \DescribeMacro{svncgauthor}
% \DescribeMacro{svncgdate}
% \DescribeMacro{svncgyear}
% \DescribeMacro{svncgmonth}
% \DescribeMacro{svncgday}
% \DescribeMacro{svncghour}
% \DescribeMacro{svncgminute}
% \DescribeMacro{svncgsecond}
% These macros return keyword values of the currently selected keyword group,
% like the \cs{svnrev}, \cs{svnfilerev}, etc., macros described in
% sections~\ref{sec:kwaccess} do for the whole document and the current file,
% respectively.
% In order to hold them robust, which is important to use them in macros like
% \cs{svnFullAuthor} they do not provide any arguments to select other groups
% than the current one. To access keyword values of other groups use the general
% macro \cs{svng}\marg{group name}\marg{key} or change the locally selected
% keyword group using the \cs{svnsetcg}.\par
% \vspace{2\baselineskip}
% \DescribeMacro{svncgtimezone}
% \DescribeMacro{svncgtimezonehour}
% \DescribeMacro{svncgtimezoneminute}
% \vspace{3\baselineskip}
%
% \smallskip\noindent
% \DescribeMacro{svnsetcg}\hspace{-.75em}\marg{group name}\\
% Normally the |\svncg|\meta{XXX} macros mentioned above use the last keyword
% group defined by \cs{svngroup}, but this can be changed using the
% \cs{svnsetcg} macro.\par The idea is that the currently selected group can be
% changed locally to the current \TeX\ group for the keyword output macros
% |\svncg|\meta{XXX} only while the group for the keyword input macros like
% \cs{svnid} is unaffected.\par
% To reset the used group to the last one defined by \cs{svngroup} simply
% use \cs{svnsetcg} with an empty argument.\par
% \paragraph*{Example 1:} |{\svnsetcg{abc}\svnFullAuthor{\svncgauthor}}|\\ would
% output the full author's name of group \textit{abc}.\par
% \paragraph*{Example 2:} To typeset the three keyword values of group
% \textit{abc} somewhere outside this group use:\\
% |{\svnsetcg{abc}Rev: \svncgrev\\||Date: \svncgdate\\|\\{}
% |Author: \svncgauthor\\}|
% \paragraph*{Example 3:} To typeset the date of group \textit{abc} outside of
% this group in the format of |\today| use: |{\svnsetcg{abc}\svncgtoday}|
% \smallskip
%
% \DescribeMacro{thesvncg}
% Returns the name of the current group selected by \cs{svnsetcg}.
%
% \smallskip\noindent
% \DescribeMacro{svng}\hspace{-.75em}\marg{group name}\marg{key}\\
% This macro is a general form of the |\svncg|\meta{XXX} macro mentioned above.
% The first argument is the requested keyword group, the second one the
% requested keyword in the form of |rev|, |date|, |author|, |year|, etc.. Please
% note that this macro can not be used inside macros like \cs{svnFullAuthor}.
%
%
% \section{Including Keywords of External Files}\label{sec:external}
% \DescribeMacro{svnexternal}\hspace{-.75em}||\meta{filea}|}{|\meta{fileb}|}|%
% \ldots|{|\meta{filex}|}|\\
% \indent
% Subversion keywords of external files (\eg non-\LaTeX\ files like images or
% even directories) can be included using this macro which awaits a list of
% files, each with the full path relative to the main \LaTeX\ file and each
% enclosed by |{ }|. The files must be under version control by Subversion, of
% course. Use \cs{svnexternalpath} to specify paths to be scanned for this
% files.  The requested filenames are written into the |.svn| auxiliary file and
% then processed by the external script |svn-multi.pl| which must be executed
% like described below. The appropriate keywords are then written in
% \meta{source file}|.svx| files (|x| like eXternal) which are read in by the
% \cs{svnexternal} macro at the next \LaTeX\ run. Therefore the keywords are
% processed at the location \eg in the same keyword group as this macro. The
% file local macros like \cs{svnfilerev} which appear in a source file after
% \cs{svnexternal} are affected, \ie updated if one of the external revision is
% higher than the one of the source file. This makes sense if the included
% graphics are taken as logical part of a source file.\medskip
%
% \noindent
% \DescribeMacro{svnexternalpath}\hspace{-.75em}|{|\meta{patha}|}{|\meta{pathb}%
% |}|\ldots|{|\meta{pathx}|}|\\
% \indent
% This macro can be used in the document preamble to declare a set of paths to
% be scanned for files specified with \cs{svnexternal}. This avoids the need to
% provide the path again and again for every file.
% The paths need to be enclosed in |{ }| and must be in Unix style, \ie with
% `|/|' as directory separator and should end with a `|/|'. Windows users should
% just replace all `|\|' with `|/|', \eg `|C:\My dir|' gets `|{C:/My dir/}|'.
%
% \subsection*{Script \texttt{svn-multi.pl}}
% The file |svn-multi.pl| which comes with the |svn-multi| package is an
% external Perl script which has to be run in the command line or by a \LaTeX\
% development environment/editor like other tools like |Bib|\TeX\ or
% |Makeindex|. A Perl interpreter and a Subversion command line client (|svn|)
% must be installed to execute this script. Both are available for free for all
% major operating systems.\par
% The script should be run inside the document folder in the following order:
% \begin{enumerate}
%  \item Compile \LaTeX\ document, |.svn| file is generated.
%  \item Run |svn-multi.pl| script, |.svx| files are generated.
%  \item Compile \LaTeX\ document, |.svx| files are read in.
% \end{enumerate}
%
% The script can be used with three different sets of arguments and with
% any combination of them. Please note that the word |jobname| stands for
% the main \LaTeX\ file name without the |.tex| extension.\par\medskip
% \noindent
% \DescribeScript{svn-multi.pl}\hspace{-.5em}\meta{jobname}\\
% \indent
% As already mentioned in the \cs{svnexternal} description above this script
% reads the requested external filename from the \meta{jobname}|.svn| file. The
% Subversion command line client |svn| is then used to fetch the needed keywords
% which are placed in a \cs{svnidlong} macro inside a \meta{source file}|.svx|
% file. Every single source file which uses \cs{svnexternal} will become its own
% |.svx| file which allows to attach specific external files to one (or more)
% specific source files.\par\medskip
% \noindent
% \DescribeScript{svn-multi.pl}\hspace{-.5em}\meta{jobname}
% |[--group| \meta{file % group}|]| \meta{file(s)} \ldots\\
% The second way to use |svn-multi.pl| is to call it with a list of external
% files. A keyword group can be specified using the |--group| \meta{group name}
% option which can placed any number of times between the file names. The group
% is used for all external files listed after the option until the next group is
% specified. All keywords of these files are written in the \meta{jobname}|.svx|
% file and read in by the main \LaTeX\ file if a, possible empty,
% \cs{svnexternal} macro is included. This allows for easy including of many
% external files without specifying them all inside the source file. For example
% |svn-multi.pl */*.jpg| (under Linux/Unix) will include the keywords of all JPG
% files in all subdirectories.\par
% It is also possible to do this with a sub-(\LaTeX)-file by calling the script
% on it: |svn-multi.pl|~\mbox{\meta{sub file} \meta{external files for sub
% file}}, which will create/overwrite the \meta{sub file}|.svx| file.  However
% the files given by \cs{svnexternal} in this sub-file will not be honoured in
% this case.\par\medskip
% \noindent
% \DescribeScript{svn-multi.pl}\hspace{-.5em}\meta{jobname}
% |[--group| \meta{group name}|] --fls|\\
% Instead of providing a list of all non-\LaTeX/external files the |--fls|
% option can be used to read this list from the \meta{jobname}|.fls| file. This
% file is produced by the \LaTeX\ compiler when run with the |--recorder| option
% and contains a list of all input and output files. Only input files with a
% relative path are used. A corresponding keyword group can also be specified.
%
% \section{Further Reading}
% The \textsf{svn-multi} package (in version 1.3) and its usage got discussed in
% the following articles:
%
% \begin{itemize}
%  \item[{[1]}] Martin Scharrer, ``Version Control of LaTeX Documents with
%  svn-multi'', The Prac\TeX\ Journal, (3), 2007.
%  URL: \url{http://www.tug.org/pracjourn/2007-3/scharrer/}
%  \item[{[2]}] Mark Eli Kalderon, ``LaTeX and Subversion'',
%  The Prac\TeX\ Journal, (3), 2007.
%  URL: \url{http://www.tug.org/pracjourn/2007-3/kalderon-svnmulti/}
%  \item[{[3]}] Uwe Ziegenhagen , ``LaTeX Document Management with Subversion'',
%  The Prac\TeX\ Journal, (3), 2007.
%  URL: \url{http://www.tug.org/pracjourn/2007-3/ziegenhagen/}
% \end{itemize}
%
% \StopEventually{}
% %%%%%%%%%%%%%%%%%%%%%%%%%%%%%%%%%%%%%%%%%%%%%%%%%%%%%%%%%%%%%%%%%%%%%%%%%%%%
% \section{Implementation}
% \subsection{Package Header}
% \subsubsection*{Package Identification}
%    \begin{macrocode}
\NeedsTeXFormat{LaTeX2e}[1999/12/01]
\ProvidesPackage{svn-multi}
 [\filedate\space\fileversion\space SVN Keywords for multi-file LaTeX documents]
%    \end{macrocode}

% \subsubsection*{Preliminary Declarations}
% Declaration of feature switches.
%    \begin{macrocode}
\newif\if@svnmulti@verbatim
\newif\if@svnmulti@groups
\newif\if@svnmulti@external
\newif\if@svnmulti@graphics
\newif\if@svnmulti@pgfimages
\newif\if@svnmulti@autoload
\newif\if@svnmulti@table
\newif\if@svnmulti@needinputfilename
\newif\if@svnmulti@anygraphic
\newif\if@svnmulti@filesasgroups
%    \end{macrocode}
%
% \subsubsection*{Options}
%    \begin{macrocode}
\DeclareOption{old}{%
  \@svnmulti@verbatimtrue
  \@svnmulti@groupsfalse
  \@svnmulti@externalfalse
  \@svnmulti@graphicsfalse
  \@svnmulti@pgfimagesfalse
  \@svnmulti@autoloadfalse
  \@svnmulti@tablefalse
  \@svnmulti@needinputfilenamefalse
  \@svnmulti@filesasgroupsfalse
}
\DeclareOption{all}{%
  \@svnmulti@verbatimtrue
  \@svnmulti@groupstrue
  \@svnmulti@externaltrue
  \@svnmulti@graphicstrue
  \@svnmulti@pgfimagestrue
  \@svnmulti@autoloadtrue
  \@svnmulti@tabletrue
  \@svnmulti@needinputfilenametrue
  \@svnmulti@filesasgroupstrue
}
\DeclareOption{verbatim}{\@svnmulti@verbatimtrue}
\DeclareOption{noverbatim}{\@svnmulti@verbatimfalse}
\DeclareOption{groups}{%
    \@svnmulti@groupstrue
    \@svnmulti@needinputfilenametrue
}
\DeclareOption{nogroups}{\@svnmulti@groupsfalse}
\DeclareOption{external}{%
    \@svnmulti@externaltrue
    \@svnmulti@needinputfilenametrue
}
\DeclareOption{filesasgroups}{%
    \@svnmulti@groupstrue
    \@svnmulti@filesasgroupstrue
    \@svnmulti@needinputfilenametrue
}
\DeclareOption{nofileasgroups}{\@svnmulti@filesasgroupsfalse}
\DeclareOption{external}{%
    \@svnmulti@externaltrue
    \@svnmulti@needinputfilenametrue
}
\DeclareOption{noexternal}{%
    \@svnmulti@externalfalse
    \@svnmulti@pgfimagesfalse
    \@svnmulti@graphicsfalse
    \@svnmulti@autoloadfalse
    \@svnmulti@needinputfilenamefalse
}
\DeclareOption{graphics}{%
    \@svnmulti@graphicstrue
    \@svnmulti@externaltrue
    \@svnmulti@autoloadtrue
    \@svnmulti@needinputfilenametrue
}
\DeclareOption{nographics}{\@svnmulti@graphicsfalse}
\DeclareOption{pgfimages}{%
    \@svnmulti@pgfimagestrue
    \@svnmulti@externaltrue
    \@svnmulti@autoloadtrue
    \@svnmulti@needinputfilenametrue
}
\DeclareOption{nopgfimages}{\@svnmulti@pgfimagesfalse}
\DeclareOption{autoload}{%
    \@svnmulti@autoloadtrue
    \@svnmulti@externaltrue
    \@svnmulti@needinputfilenametrue
}
\DeclareOption{noautoload}{\@svnmulti@autoloadfalse}
\DeclareOption{table}{%
    \@svnmulti@tabletrue
    \@svnmulti@groupstrue
    \@svnmulti@needinputfilenametrue
}
\DeclareOption{notable}{\@svnmulti@tablefalse}

\ExecuteOptions{old}
\ProcessOptions\relax
%    \end{macrocode}

% General switch if any graphic option is enabled:
%    \begin{macrocode}
\if@svnmulti@graphics
  \@svnmulti@anygraphictrue
\fi
\if@svnmulti@pgfimages
  \@svnmulti@anygraphictrue
\fi
%    \end{macrocode}
%

% \subsection{Definition of init values}
%    \begin{macrocode}
% Init values
\def\svnrev{-1}             \def\@svn@rev{-1}
\def\svndate{}              \def\@svn@date{}
\def\svnauthor{}            \def\@svn@author{}
\def\svnyear{0}             \def\@svn@year{0}
\def\svnmonth{0}            \def\@svn@month{0}
\def\svnday{0}              \def\@svn@day{0}
\def\svnhour{0}             \def\@svn@hour{0}
\def\svnminute{0}           \def\@svn@minute{0}
\def\svnsecond{0}           \def\@svn@second{0}
\def\svntimezonehour{+00}   \def\@svn@timezonehour{+00}
\def\svntimezoneminute{00}  \def\@svn@timezoneminute{00}
\def\svnmainurl{NOT SET}    \def\svnmainfilename{NOT SET}
\def\svn@g{} \def\svn@cg{\svn@g} \def\svn@temp{}

\def\svn@initfile{%
  \gdef\svnfilerev{-1}%
  \gdef\svnfiledate{}%
  \gdef\svnfileauthor{}%
  \gdef\svnfileyear{0}%
  \gdef\svnfilemonth{0}%
  \gdef\svnfileday{0}%
  \gdef\svnfilehour{0}%
  \gdef\svnfileminute{0}%
  \gdef\svnfilesecond{0}%
  \gdef\svnfiletimezonehour{+00}%
  \gdef\svnfiletimezoneminute{00}%
}
\svn@initfile
%    \end{macrocode}
%
% \subsection{Timezone macros}
% \begin{macro}{\svntimezone}
% \begin{macro}{\svnfiletimezone}
% \begin{macro}{\svncgtimezone}
% These macros return the global, file-local and current group time zones,
% respectively. Since v1.4 the minute part is returned as well and the macro
% removes manually added |00| after it to support older documents.
% \changes{v1.4}{2009/02/27}{Return now full timezone (hour + minute part).
% Manually added 00 minutes are removed.}
%    \begin{macrocode}
\def\svntimezone{\svntimezonehour\svntimezoneminute\svn@gobblezeros}
\def\svnfiletimezone{\svnfiletimezonehour\svnfiletimezoneminute\svn@gobblezeros}
\def\svncgtimezone{\svncgtimezonehour\svncgtimezoneminute}
%    \end{macrocode}
% \end{macro}
% \end{macro}
% \end{macro}

% \begin{macro}{\svn@gobblezeros}
% \begin{macro}{\svn@gobblezeros@}
% This two cascaded macros remove a trailing |00| and are used by
% \cs{svnfiletimezone} and \cs{svntimezone}.
%    \begin{macrocode}
\def\svn@gobblezeros{%
\futurelet\svn@nextchar\svn@gobblezeros@
}
\def\svn@gobblezeros@{%
\let\@tempa=\relax
\def\@tempb{0}%
\ifx0\svn@nextchar
\let\@tempa=\@gobbletwo
\fi
\@tempa
}
%    \end{macrocode}
% \end{macro}
% \end{macro}

% \begin{macro}{\svntime}
% \begin{macro}{\svnfiletime}
% \begin{macro}{\svncgtime}
% This macros simple use the hour, minute and second macros.
%    \begin{macrocode}
\def\svntime{\svnhour:\svnminute:\svnsecond}
\def\svnfiletime{\svnfilehour:\svnfileminute:\svnfilesecond}
\def\svncgtime{\svncghour:\svncgminute:\svncgsecond}
%    \end{macrocode}
% \end{macro}
% \end{macro}
% \end{macro}

% \subsection{\textit{Today} macros}
% These macros use the |\today| macro to typeset the current date using the
% local language settings. Thanks and credit goes to Manuel P\'egouri\'e-Gonnard
% for suggesting this feature and for providing the code.
% \begin{macro}{\svntoday}
%    \begin{macrocode}
\newcommand*{\svntoday}{%
\begingroup
\year\svnyear \month\svnmonth \day\svnday
\relax \today
\endgroup}
%    \end{macrocode}
% \end{macro}
%
% \begin{macro}{\svnfiletoday}
%    \begin{macrocode}
\newcommand*{\svnfiletoday}{%
\begingroup
\year\svnfileyear \month\svnfilemonth \day\svnfileday
\relax \today
\endgroup}
%    \end{macrocode}
% \end{macro}
%
% \begin{macro}{\svncgtoday}
%    \begin{macrocode}
\newcommand*{\svncgtoday}{%
\@ifundefined{svng@\svn@cg @year}{??}{%
\begingroup
\year\svncgyear \month\svncgmonth \day\svncgday
\relax \today
\endgroup
}%
}%
%    \end{macrocode}
% \end{macro}

% \subsection{Id macros}
% \subsubsection{Normal Id}
% \begin{macro}{\svnid}
% Calls \cs{svnkwsave} with |\@svnidswtrue| so that the Id keyword will be
% parsed at the end of \cs{svnkwsave}.
%    \begin{macrocode}
\newcommand*{\svnid}{%
\@svnidswtrue
\svnkwsave
}
\newif\if@svnidsw
\@svnidswfalse
%    \end{macrocode}
% \end{macro}
%

% \begin{macro}{\svn@scanId}
% Scans svn Id (after it got parsed by \cs{svnkwsave}).  Awaits only Id value
% without leading `|Id:|' and a trailing |\relax| as end marker.  It calls
% \cs{@svn@scandate} to extract the date informations and \cs{@svn@updateid} to
% update global Id values and also sets the appropriate keywords.
%    \begin{macrocode}
\def\svn@scanId#1 #2 #3 #4 #5\relax{%
% #1 is filename, #2 is revision, #3 is date (JJJJ-MM-DD),
% #4 is time (HH:MM:SST), #5 is author (username)
\@svn@scandate{#3 #4}%
\@svn@updateid{#2}{#3 #4}{#5}%
\svnkwdef{Filename}{#1}%
\svnkwdef{Date}{#3 #4}%
\svnkwdef{Revision}{#2}%
\svnkwdef{Author}{#5}%
}
%    \end{macrocode}
% \end{macro}
%

% \begin{macro}{\@svn@updateid}
% We first define the expanded arguments to variables for the user.  The
% expansion is needed because the arguments content is mostly generic like
% |\svn@value| which can change very soon after this macro.
%    \begin{macrocode}
\def\@svn@updateid#1#2#3{% #1 = rev, #2 date, #3 author (username)
\xdef\svnfilerev{#1}%
\xdef\svnfiledate{#2}%
\xdef\svnfileauthor{#3}%
%    \end{macrocode}
% Then we check if the revision is non-empty (not yet expanded by subversion?)
% and larger then the current maximum value |\@svn@rev|.  If yes we save all
% value to save them in the .svn-file later.
%    \begin{macrocode}
\ifx\svnfilerev\empty\else
\ifnum\@svn@rev<\svnfilerev
\xdef\@svn@rev{\svnfilerev}%
\xdef\@svn@date{\svnfiledate}%
\xdef\@svn@author{\svnfileauthor}%
\xdef\@svn@year{\svnfileyear}%
\xdef\@svn@month{\svnfilemonth}%
\xdef\@svn@day{\svnfileday}%
\xdef\@svn@hour{\svnfilehour}%
\xdef\@svn@minute{\svnfileminute}%
\xdef\@svn@second{\svnfilesecond}%
\xdef\@svn@timezonehour{\svnfiletimezonehour}%
\xdef\@svn@timezoneminute{\svnfiletimezoneminute}%
\else\fi

\ifx\svn@g\empty\else
\expandafter\ifnum\csname @svng@\svn@g @rev\endcsname<\svnfilerev
\@svncg@save{rev}{\svnfilerev}%
\@svncg@save{date}{\svnfiledate}%
\@svncg@save{author}{\svnfileauthor}%
\@svncg@save{year}{\svnfileyear}%
\@svncg@save{month}{\svnfilemonth}%
\@svncg@save{day}{\svnfileday}%
\@svncg@save{hour}{\svnfilehour}%
\@svncg@save{minute}{\svnfileminute}%
\@svncg@save{second}{\svnfilesecond}%
\@svncg@save{timezonehour}{\svnfiletimezonehour}%
\@svncg@save{timezoneminute}{\svnfiletimezoneminute}%
\fi
\fi

\fi
}

\def\@svncg@save#1#2{%
\expandafter\xdef\csname @svng@\svn@g @#1\endcsname{#2}%
}

%    \end{macrocode}
% \end{macro}
%

% \subsubsection{Long Id}
% \begin{macro}{\svnidlong}
% We clear the keyword value first to reduce the risk though bad user input.
%    \begin{macrocode}
\newcommand{\svnidlong}{%
\svnkwdef{HeadURL}{}%
\svnkwdef{LastChangedDate}{}%
\svnkwdef{LastChangedRevision}{0}%
\svnkwdef{LastChangedBy}{}%
%    \end{macrocode}
% The catcodes are changed by \cs{svn@catcodes} to allow \TeX-special characters
% inside the keywords.  The braces \{ \} are changed to allow comments between
% the arguments.  \cs{svnidlong@readargsfull} is called to read the arguments.
%    \begin{macrocode}
\begingroup
\if@svnmulti@verbatim
\svn@catcodes
\catcode`\{=12
\catcode`\}=12
\else
\def\svnidlong@readargsfull{\svnidlong@readargs}%
\fi
\svnidlong@readargsfull
}
%    \end{macrocode}
% \end{macro}

% \begin{macro}{\svn@catcodes}
% Changes all \TeX-special character to category ``other''. The newline aka
% return is changed to category ``ignore'' so line breaks are not taken as part
% of the verbatim arguments.
%    \begin{macrocode}
\if@svnmulti@verbatim
\def\svn@catcodes{%
\let\do\@makeother \dospecials
\catcode`\^^M9 \catcode`\ 10
\catcode`\{1 \catcode`\}2
}
\else
\def\svn@catcodes{}
\fi
%    \end{macrocode}
% \end{macro}

% \begin{macro}{\svnidlong@readargsfull}
% Reads all four arguments of \cs{svnidlong} and passes them to
% \cs{svnidlong@readargs}. The normal argument braces are changed to category
% ``other'' and put into the macros parameter text to remove all code between
% them. This is done to avoid problems with comments direct after one of the
% arguments. Because the braces are now non-special the parentheses are made to
% a local replacement.
%
%    \begin{macrocode}
\if@svnmulti@verbatim
\begingroup
\catcode`\{=12\catcode`\}=12
\catcode`\(=1\catcode`\)=2
\gdef\svnidlong@readargsfull#1{#2}#3{#4}#5{#6}#7{#8}(%
 \svnidlong@readargs(#2)(#4)(#6)(#8)%
)
\endgroup
\fi
%    \end{macrocode}
% \end{macro}

% \begin{macro}{\svnidlong@readargs}
% Calls sub macro for all four arguments and ends the catcode changes made
% by \cs{svnidlong}.
%    \begin{macrocode}
\def\svnidlong@readargs#1#2#3#4{%
\svnkwsave@read #1\relax
\svnkwsave@read #2\relax
\svnkwsave@read #3\relax
\svnkwsave@read #4\relax
\endgroup
%    \end{macrocode}
% Now the update macros for date and id are called.
%    \begin{macrocode}
\ifx\svnkwLastChangedDate\empty
\else\@svn@scanlongdate{\svnkwLastChangedDate}\fi
\@svn@updateid{\svnkw{LastChangedRevision}}{\svnkw{LastChangedDate}}%
{\svnkw{LastChangedBy}}%
\ignorespaces
}%
%    \end{macrocode}
% \end{macro}

% \subsection{KeyWord Macros}
% \begin{macro}{\svnkwsave}
% Enabled verbatim mode and uses a sub macro to read the arguments afterwards.
%    \begin{macrocode}
\def\svnkwsave{%
\begingroup
\svn@catcodes
\svnkwsave@readargs
}
%    \end{macrocode}
% \end{macro}

% \begin{macro}{\svnkwsave@readargs}
% Reads full argument, calls parse submacro and ends catcode changes.
% If \cs{svnkwsave} was called by \cs{svnid} scans the id keyword by calling the
% scan macro.
%    \begin{macrocode}
\gdef\svnkwsave@readargs#1{%
\svnkwsave@read#1\relax
\endgroup
\if@svnidsw
  \ifx\svnkwId\empty
  \else
    \expandafter
    \svn@scanId\svnkwId\relax
    \@svnidswfalse
  \fi
\fi
\ignorespaces
}
%    \end{macrocode}
% \end{macro}

% \begin{macro}{\svnkwsave@read}
% Reads the full keyword and strips the dollars.
%    \begin{macrocode}
\begingroup
\if@svnmulti@verbatim
\catcode`\$=12
\fi
\gdef\svnkwsave@read $#1$\relax{%
    \svn@checkcolon#1:\relax
}
\endgroup
%    \end{macrocode}
% \end{macro}

% \begin{macro}{\svnkwsave@parse}
% Parse the keyword and save it away.
%    \begin{macrocode}
\begingroup
\catcode`\$=11
\gdef\svnkwsave@parse$#1:#2${%
\expandafter\xdef\csname svnkw#1\endcsname{#2}}%
\endgroup
%    \end{macrocode}
% \end{macro}

% \begin{macro}{\svnkwdef}
% First we check if there is a `setter'-macro for the keyword called
% \cs{svnkwdef@}\meta{keyword}.
%    \begin{macrocode}
\newcommand{\svnkwdef}[2]{%
\@ifundefined{svnkwdef@#1}%
%    \end{macrocode}
% If not we call the general macro \cs{svnkwdef@}.
%    \begin{macrocode}
{\svnkwdef@{#1}{#2}}%
%    \end{macrocode}
% If yes we just call it with the value as argument.
%    \begin{macrocode}
{\csname svnkwdef@#1\endcsname{#2}}%
}
%    \end{macrocode}
% \end{macro}

% \begin{macro}{\svnkwdef@}
% This macro defines the second argument under \cs{svnkw}\meta{1st argument}.
% The |\xdef| is used to expand the content first (needed for internal use) and
% make the definition globally.
%    \begin{macrocode}
\newcommand{\svnkwdef@}[2]%
{\expandafter\xdef\csname svnkw#1\endcsname{#2}}
%    \end{macrocode}
% Example: |\svnkwdef{Revision}{23}| will define |\svnkwRevision| as 23.
% \end{macro}

% \begin{macro}{\svnkwdef@Rev}
% \begin{macro}{\svnkwdef@Author}
% \begin{macro}{\svnkwdef@Date}
% `Setter'-macros for single keywords, used by \cs{svnkwdef}.\\ These are needed
% to have have a common value for all alternative keyword names ala |Rev|,
% |Revision|, |LastChangedRevision|.
%
% The keywords |Author| and |Date| are just calling \cs{svnkwdef@} with a fixed
% first argument.  For the revision the value is checked if empty and then a 0
% is substituted.
%    \begin{macrocode}
\def\svnkwdef@Rev#1{%
    \begingroup
    \def\svn@temp{#1}%
    \ifx\svn@temp\empty
      \endgroup
      \svnkwdef@{Rev}{0}%
    \else
      \endgroup
      \svnkwdef@{Rev}{#1}%
    \fi
}
\def\svnkwdef@Author#1{\svnkwdef@{Author}{#1}}
\def\svnkwdef@Date#1{\svnkwdef@{Date}{#1}}
%    \end{macrocode}
% The long keywords are defined then as aliases of the short,\\
% first for writing
%    \begin{macrocode}
\let\svnkwdef@Revision=\svnkwdef@Rev
\let\svnkwdef@LastChangedRevision=\svnkwdef@Rev
\let\svnkwdef@LastChangedBy=\svnkwdef@Author
\let\svnkwdef@LastChangedAt=\svnkwdef@Date
%    \end{macrocode}
% and then for reading.
%    \begin{macrocode}
\def\svnkwRevision{\svnkwRev}
\def\svnkwLastChangedRevision{\svnkwRev}
\def\svnkwLastChangedBy{\svnkwAuthor}
\def\svnkwLastChangedAt{\svnkwDate}
%    \end{macrocode}
% So \eg |\svnkw{LastChangedRevision}| is always be the
% same as |\svnkw{Rev}|.
% \end{macro}
% \end{macro}
% \end{macro}

% We define default values for normal keywords. Keyword |Filename| is the name
% given by |Id| and not a real keyword.
%    \begin{macrocode}
\svnkwdef{Rev}{0}
\svnkwdef{Date}{}
\svnkwdef{Author}{}
\svnkwdef{Filename}{}
\svnkwdef{HeadURL}{}
%    \end{macrocode}

% \begin{macro}{\svnkw}
% Macro to get keyword value. Just calls \cs{svnkw}\meta{ARGUMENT} where
% the argument interpreted as text. So \eg |\svnkw{Date}| is the same as
% |svnkwDate| but this could be changed later so always use this interface
% to get the keyword values.
%
% \changes{v1.2}{2007/06/22}{Added warning when a wrong, maybe
% misspelled, keyword is given.}
%    \begin{macrocode}
\newcommand{\svnkw}[1]{%
\@ifundefined{svnkw#1}%
{\PackageWarning{svn-multi}{SVN keyword '#1' not defined (typo?)}}%
{\csname svnkw#1\endcsname}}%
%    \end{macrocode}
% \end{macro}
%

% \subsection{Keyword check and strip macros}
% The following macros are used to test whether the given keywords are fully
% expanded or not.
% Subversion supports unexpanded keywords as input with or without colon and
% with or without trailing space(s), \ie a:~|$KW$|, b:~|$KW:$| or c:~|$KW: $|.
% To avoid \LaTeX{} syntax errors in this pre-commit state the keyword is
% checked by the following macros. Unexpanded keywords result in an empty value.
% Also leading and trailing spaces are removed.
%
% \begin{macro}{\svn@checkcolon}
% Checks if the keyword contains a colon. It is called by \cs{svnkwsave@read}
% with a trailing |:\relax| so that \#2 will be empty if there is no earlier
% colon or will hold the value with this trailing colon otherwise.
% The first case means that the keyword is unexpanded without colon (case a)
% which leads to an empty value. In the second case \cs{svn@stripcolon} is
% called to strip the colon and surrounding spaces. The final value is
% returned by |\svn@value|.
%    \begin{macrocode}
\def\svn@checkcolon#1:#2\relax{%
    \def\svn@temp{#2}%
    \ifx\svn@temp\empty
    \svnkwdef{#1}{}%
    \else
      \svn@stripcolon#2\relax
      \svnkwdef{#1}{\svn@value}%
    \fi
}
%    \end{macrocode}
% \end{macro}

% \begin{macro}{\svn@stripcolon}
% Strips the previous added colon (for \cs{svn@checkcolon}).
% The remaining argument is checked if it's empty (case b) or only a space
% (case c). Otherwise the keyword is expanded and \cs{svn@stripspace} is
% called to strip the spaces.
%    \begin{macrocode}
\def\svn@stripcolon#1:\relax{%
    \def\svn@temp{#1}%
    \ifx\svn@temp\empty
      \gdef\svn@value{}%
    \else
    \ifx\svn@temp\svn@spaceonly
      \gdef\svn@value{}%
    \else
     \svn@stripspace#1\relax\relax
    \fi\fi
}

\def\svn@spaceonly{ }
%    \end{macrocode}
% \end{macro}

% \begin{macro}{\svn@stripspace}
% Strips leading space if present and calls \cs{svn@striptrailingspace} to
% strip the trailing space.
%    \begin{macrocode}
\def\svn@stripspace#1#2\relax{%
    \def\svn@temp{#1}%
    \ifx\svn@temp\svn@spaceonly
      \gdef\svn@value{#2}%
    \else
      \svn@striptrailingspace#1#2\relax
    \fi
}
%    \end{macrocode}
% \end{macro}

% \begin{macro}{\svn@striptrailingspace}
% Strips trailing space using the macros parameter text. Must be called with
% |\relax| as end marker.
%    \begin{macrocode}
\def\svn@striptrailingspace#1 \relax{%
    \gdef\svn@value{#1}%
}
%    \end{macrocode}
% \end{macro}

% \subsection{Date Macros}
% \begin{macro}{\@svn@scandate}
% Scans data informations in Id keyword and saves them in macros.
%    \begin{macrocode}
\def\@svn@scandate#1{\@svn@scandate@#1\relax}

\def\@svn@scandate@#1-#2-#3 #4:#5:#6#7#8\relax{%
\gdef\svnfileyear{#1}%
\gdef\svnfilemonth{#2}%
\gdef\svnfileday{#3}%
\gdef\svnfilehour{#4}%
\gdef\svnfileminute{#5}%
\gdef\svnfilesecond{#6#7}%
\gdef\svnfiletimezonehour{+00}%
\gdef\svnfiletimezoneminute{00}% #8 always 'Z' for Zulu-time (UTC)
}
%    \end{macrocode}
% \end{macro}

% \begin{macro}{\@svn@scanlongdate}
% Scans data informations in Date keyword and saves them in macros.
%    \begin{macrocode}
\def\@svn@scanlongdate#1{\expandafter\@svn@scanlongdate@#1\relax}
%
\def\@svn@scanlongdate@#1-#2-#3 #4:#5:#6 #7 #8\relax{%
\gdef\svnfileyear{#1}%
\gdef\svnfilemonth{#2}%
\gdef\svnfileday{#3}%
\gdef\svnfilehour{#4}%
\gdef\svnfileminute{#5}%
\gdef\svnfilesecond{#6}%
\@svn@parsetimezone#7\relax%
}
%    \end{macrocode}
% \end{macro}

% \begin{macro}{\@svn@parsetimezone}
% Scans timezone and splits hour and minute part.
%    \begin{macrocode}
\def\@svn@parsetimezone#1#2#3#4#5\relax{%
\gdef\svnfiletimezonehour{#1#2#3}%
\gdef\svnfiletimezoneminute{#4#5}%
}
%    \end{macrocode}
% \end{macro}

% \begin{macro}{\svnpdfdate}
% Returns date in a format needed for |\pdfinfo|.
%    \begin{macrocode}
\def\svnpdfdate{\svnyear\svnmonth\svnday
\svnhour\svnminute\svnsecond\svntimezonehour'\svntimezoneminute'}
%    \end{macrocode}
% \end{macro}

% \subsection{Mainfile Makros}
% \begin{macro}{\svnsetmainfile}
% Saves the current |HeadURL| and |Filename| keywords to macros.
% Will be called automatically in the preamble.
% \changes{v1.2}{2007/06/22}{New macro}
%    \begin{macrocode}
\newcommand{\svnsetmainfile}%
{\edef\svnmainurl{\svnkw{HeadURL}}%
\edef\svnmainfilename{\svnkw{Filename}}}
\AtBeginDocument{\svnsetmainfile}
%    \end{macrocode}
% \end{macro}

% \subsection{Register and FullName Macros}
% \begin{macro}{\svnRegisterAuthor}
% Saves the author's name by defining
% |svn@author@|\meta{username} to it.
%    \begin{macrocode}
\newcommand{\svnRegisterAuthor}[2]{%
\expandafter\def\csname svn@author@#1\endcsname{#2}%
}
%    \end{macrocode}
% \end{macro}

% \begin{macro}{\svnFullAuthor}
% \begin{macro}{\svnFullAuthor*}
% We test if the starred or the normal version is used and call the
% appropriate submacro |svnFullAuthor@star| or |svnFullAuthor@normal|.
% \changes{v1.2}{2007/06/22}{Macro now returns the username if the full name
% was not registered.}
%    \begin{macrocode}
\newcommand{\svnFullAuthor}{%
\@ifnextchar{*}%
{\svnFullAuthor@star}%
{\svnFullAuthor@normal}%
}%
%    \end{macrocode}
% Both submacros are calling |svnFullAuthor@| but with different arguments.
% The star macro also removes the star of course.
%    \begin{macrocode}
\def\svnFullAuthor@star*#1{%
\edef\svn@temp{#1}%
\svnFullAuthor@{\svn@temp}{~(\svn@temp)}}%
\def\svnFullAuthor@normal#1{%
\edef\svn@temp{#1}%
\svnFullAuthor@{\svn@temp}{}}%
%    \end{macrocode}
% |svnFullAuthor@| now sets the author's full name. Note that |#2| is empty
% when the normal version is called.
%    \begin{macrocode}
\def\svnFullAuthor@#1#2{%
\@ifundefined{svn@author@#1}{#1}%
{\csname svn@author@#1\endcsname #2}%
}
%    \end{macrocode}
% \end{macro}
% \end{macro}

% \begin{macro}{\svnRegisterRevision}
% Saves the revision's name or tag by defining
% |svn@revision@|\meta{revisionnumber} to it.
% \changes{v1.2}{2007/06/22}{New macro}
%    \begin{macrocode}
\newcommand{\svnRegisterRevision}[2]{%
\expandafter\def\csname svn@revision@#1\endcsname{#2}%
}
%    \end{macrocode}
% \end{macro}

% \begin{macro}{\svnFullRevision}
% \begin{macro}{\svnFullRevision*}
% We test if the starred or the normal version is used and call the
% appropriate submacro |svnFullRevision@star| or |svnFullRevision@normal|.
% \changes{v1.2}{2007/06/22}{New macro}
%    \begin{macrocode}
\newcommand{\svnFullRevision}{%
\@ifnextchar{*}%
{\svnFullRevision@star}%
{\svnFullRevision@normal}%
}
%    \end{macrocode}
% Both submacros are calling |svnFullRevision@| but with different arguments.
% The star macro also removes the star of course.
%    \begin{macrocode}
\def\svnFullRevision@star*#1{%
\edef\svn@temp{#1}%
\svnFullRevision@{\svn@temp}{~(r\svn@temp)}%
}
\def\svnFullRevision@normal#1{%
\edef\svn@temp{#1}%
\svnFullRevision@{\svn@temp}{}%
}
%    \end{macrocode}
% |svnFullRevision@| now sets the revision name. Note that |#2| is empty
% when the normal version is called.
%    \begin{macrocode}
\def\svnFullRevision@#1#2{%
\@ifundefined{svn@revision@#1}%
{Revision #1}%
{\csname svn@revision@#1\endcsname #2}%
}
%    \end{macrocode}
% \end{macro}
% \end{macro}

% \subsection{Input File Name}
% The FiNK package is used to get the input file names. AtBegin/AtEnd hooks are
% installed which will be used later.
%    \begin{macrocode}
\if@svnmulti@needinputfilename
%    \end{macrocode}

% Load |fink| package and check if all needed macros are provided.
%    \begin{macrocode}
\RequirePackage{fink}[2008/02/27]
\begingroup
\def\svn@finkerror{%
\PackageError{svn-multi}{Your installed version of the 'fink' package does not
provide the needed macros. It is either too old or too new.
Try a different version, e.g. v2.1.1 from 2008/02/27}{}%
\let\svn@finkerror\relax
}
\@ifundefined{finkpath}{\svn@finkerror}{}%
\@ifundefined{finkdir}{\svn@finkerror}{}%
\@ifundefined{finkbase}{\svn@finkerror}{}%
\@ifundefined{fink@prepare}{\svn@finkerror}{}%
\@ifundefined{fink@restore}{\svn@finkerror}{}%
\endgroup
%    \end{macrocode}

% \begin{macro}{\svnmulti@begininputfilehook}
% This hook is installed in the |\fink@prepare| macro from the |fink| package
% which will be executed at the begin of a input file. The file name and path
% are not yet in |\finkpath| etc. but in |\fink@nextpath|.
%    \begin{macrocode}
\def\svnmulti@begininputfilehook{}
\let\orig@fink@prepare\fink@prepare
\renewcommand*{\fink@prepare}[1]{%
  \orig@fink@prepare{#1}%
  \svnmulti@begininputfilehook
}
%    \end{macrocode}
% \end{macro}

% \begin{macro}{\svnmulti@endinputfilehook}
% This hook is installed in the |\fink@restore| macro from the |fink| package
% which will be executed at the end of a input file. The file path |\finkpath|
% etc. is still valid.
%    \begin{macrocode}
\def\svnmulti@endinputfilehook{}
\let\orig@fink@restore\fink@restore
\def\fink@restore{%
    \svnmulti@endinputfilehook
    \orig@fink@restore
}
%    \end{macrocode}
% \end{macro}

% \begin{macro}{\svnmulti@atbegininputfile}
% This macro adds the argument to the \cs{svnmulti@begininputfilehook}.
%    \begin{macrocode}
\def\svnmulti@atbegininputfile{%
    \g@addto@macro\svnmulti@begininputfilehook
}
%    \end{macrocode}
% \end{macro}

% \begin{macro}{\svnmulti@atendinputfile}
% This macro adds the argument to the \cs{svnmulti@endinputfilehook}.
%    \begin{macrocode}
\def\svnmulti@atendinputfile{%
    \g@addto@macro\svnmulti@endinputfilehook
}
%    \end{macrocode}
% \end{macro}

% \begin{macro}{\svn@removedotslash}
% Removes leading './' from given macro (holding a directory path). Awaits a
% macro as argument which is redefined inside the current group!
%    \begin{macrocode}
\def\svn@removedotslash#1{%
    \def\svn@removedotslash@##1##2##3\relax{%
        \begingroup
        \def\@tempa{./}%
        \def\@tempb{##1##2}%
        \ifx\@tempa\@tempb%
            \endgroup
            \edef#1{##3}%
        \else
            \endgroup
            \edef#1{##1##2##3}%
        \fi
    }%
    \expandafter\svn@removedotslash@#1\empty\empty\empty\relax
}
%    \end{macrocode}
% \end{macro}

%    \begin{macrocode}
\fi
%    \end{macrocode}

% \subsection{Keyword Group Macros}
% These macros implement the user interface for the keyword group functionality
% introduced with v2.0.
%
% The list of keyword groups |svn@glist| is initial set empty and will be filled
% by \cs{svngroup}.
%    \begin{macrocode}
\if@svnmulti@groups
\let\svn@glist=\empty
%    \end{macrocode}

% \begin{macro}{\svngroup}
% Saves the group to |svn@g| and initiates |svn@g@|\meta{group name}|@rev|
% and |@svn@g@|\meta{group name}|@rev| if this is the first time the group
% got used.
%    \begin{macrocode}
\def\svngroup#1{%
\xdef\svn@g{#1}%
\ifx\svn@g\empty\else%
%    \end{macrocode}
% Only initialise the group at first usage:
%    \begin{macrocode}
\expandafter\ifx\csname svn@g@#1\endcsname\relax%
\expandafter\gdef\csname svn@g@#1\endcsname{1}%
%    \end{macrocode}
% If first use, init revision numbers to avoid not-a-number errors:
%    \begin{macrocode}
\expandafter\gdef\csname @svng@#1@rev\endcsname{-1}%
\expandafter\ifx\csname svng@#1@rev\endcsname\relax
\expandafter\gdef\csname svng@#1@rev\endcsname{-1}%
\fi
%    \end{macrocode}
% Now save new group to list. The list is checked if its empty to avoid an
% unwanted leading comma.
%    \begin{macrocode}
\ifx\svn@glist\empty
\xdef\svn@glist{#1}%
\else
\xdef\svn@glist{\svn@glist,#1}%
\fi
\fi
\fi
}
%    \end{macrocode}
% \end{macro}

% \begin{macro}{\thesvngroup}
% Returns the current group name to the user.
%    \begin{macrocode}
\def\thesvngroup{\svn@g}
%    \end{macrocode}
% \end{macro}

% \begin{macro}{\svnsetcg}
% Defines |svn@cg| to the given argument or to |svn@g| if the argument was
% empty.
%    \begin{macrocode}
\def\svnsetcg#1{%
\def\svn@temp{#1}%
\ifx\svn@temp\empty
\def\svn@cg{\svn@g}%
\else
\def\svn@cg{#1}%
\fi
}
%    \end{macrocode}
% \end{macro}

% \begin{macro}{\svncg@def}
% Defines a |\svncg|\meta{XXX} macro, \eg |svncgrev|, which returns the
% requested keyword values of the current keyword group.
%    \begin{macrocode}
\def\svncg@def#1{%
\expandafter\def\csname svncg#1\endcsname{%
\@ifundefined{svng@\svn@cg @#1}{??}{%
\csname svng@\svn@cg @#1\endcsname}%
}}
%    \end{macrocode}
% \end{macro}

% \begin{macro}{\svncgXXX}
% Define all |\svncg|\meta{XXX} macros by calling \cs{svncg@def} in a for loop.
%    \begin{macrocode}
\@for\@tempa:=rev,author,date,year,month,day,hour,minute,second,%
timezonehour,timezoneminute\do{\expandafter\svncg@def\expandafter{\@tempa}}
%    \end{macrocode}
% \end{macro}

% \begin{macro}{\thesvncg}
% Simply return the internal macro.
%    \begin{macrocode}
\def\thesvncg{\svn@cg}
%    \end{macrocode}
% \end{macro}

% \begin{macro}{\svng}
% Simply returns |svng@#1@#2| if defined, '??' otherwise.
%    \begin{macrocode}
\def\svng#1#2{%
\@ifundefined{svng@#1@#2}{??}{%
\csname svng@#1@#2\endcsname}%
}
%    \end{macrocode}
% \end{macro}

% \begin{macro}{\svn@addfiletogroup}
% Adds the given file to the given group. If the group list doesn't exist yet
% it is initialised. A extra macro for each file is used to remember that the
% file is already in the group. This could be avoided using a list search.
%    \begin{macrocode}
\def\svn@addfiletogroup#1#2{%
    \expandafter\ifx\csname @svng@#2@files@#1\endcsname\relax%
      \expandafter\gdef\csname @svng@#2@files@#1\endcsname{1}%
      %
      \expandafter\ifx\csname @svng@#2@files\endcsname\relax%
        \expandafter\xdef\csname @svng@#2@files\endcsname{#1}%
      \else
        \expandafter\xdef\csname @svng@#2@files\endcsname{%
            \csname @svng@#2@files\endcsname,#1}%
      \fi
    \fi
}
%    \end{macrocode}
% \end{macro}

% The input files are added to the list of the current group at their begin to
% have them before the included graphics and other external files.
%    \begin{macrocode}
\svnmulti@atbegininputfile{%
    \svn@initfile
    \let\svn@temp=\fink@nextdir
    \svn@removedotslash\svn@temp
    \svn@addfiletogroup{\svn@temp\fink@file\fink@nextbase\fink@nextext}{\svn@g}%
}
%    \end{macrocode}

%    \begin{macrocode}
\fi
%    \end{macrocode}

% \subsection{Files as extra groups}
% Macros which allow single files to be declared as extra groups so that their
% keywords can be accessed in the whole document like with normal groups.
% This special groups are not added to the list of groups.

% A user-level switch is declared to enable or disable the automatic declaration
% of every file as own group. This causes \cs{svnfileasgroup} to be called for
% all input files.
% The if macro is defined outside the |\if@svnmulti@filesasgroups| because
% |\newif| inside |\if| is not a good idea.
%    \begin{macrocode}
\newif\ifsvnfilesasgroup
\svnfilesasgroupfalse
%    \end{macrocode}

%    \begin{macrocode}
\if@svnmulti@filesasgroups
\svnfilesasgrouptrue
%    \end{macrocode}

% \begin{macro}{\svnfileasgroup}
% User level and internal macro to declare the current file as extra group.
% It produces the current file path and calls \cs{svn@fileasgroup}.
%    \begin{macrocode}
\def\svnfileasgroup{%
    \begingroup
    \edef\svn@filename{\finkdir\fink@file\finkbase\finkext}%
    \svn@removedotslash\svn@filename
    \svn@fileasgroup{\svn@filename}
    \endgroup
}
%    \end{macrocode}
% \end{macro}

% \begin{macro}{\svn@fileasgroup}
% Macro to write a file as group to |.svn| file. After checking if the filename
% was not already written, the |.svn| file is checked if it is open and then the
% file keyword information is written.
%    \begin{macrocode}
\def\svn@fileasgroup#1{%
 \ifnum\svnfilerev>-1\relax
   \begingroup
   \expandafter\ifx\csname svn@g@#1\endcsname\relax%
   \expandafter\gdef\csname svn@g@#1\endcsname{1}%
   \svn@checkwrite
   \def\svn@writekw##1{%
     \noexpand\@namedef{@svng@#1@##1}{\csname svnfile##1\endcsname}^^J%
   }
   \immediate\write\svn@write{^^J%
     \@percentchar\space File '#1'^^J%
     \svn@writekw{rev}%
     \svn@writekw{date}%
     \svn@writekw{author}%
     \svn@writekw{year}%
     \svn@writekw{month}%
     \svn@writekw{day}%
     \svn@writekw{hour}%
     \svn@writekw{minute}%
     \svn@writekw{second}%
     \svn@writekw{timezonehour}%
     \svn@writekw{timezoneminute}%
   }%
   \fi
   \endgroup
 \fi
}
%    \end{macrocode}
% \end{macro}

% \begin{macro}{\svnignoreextensions}
% A comma separated list of file name extensions (without leading dot) to ignore
% for automatic |\svnfileasgroup|.
%% TODO: DescribeMacro
%    \begin{macrocode}
\def\svnignoreextensions#1{%
  \@for\svn@temp:=#1\do{%
    \expandafter\def\csname svn@ignore@ext@\svn@temp\endcsname{}%
  }%
}
%    \end{macrocode}
% \end{macro}

% \begin{macro}{\svnconsiderextensions}
% A comma separated list of file name extensions (without leading dot) to
% consider for automatic |\svnfileasgroup|.
%% TODO: DescribeMacro
%    \begin{macrocode}
\def\svnconsiderextensions#1{%
  \@for\svn@temp:=#1\do{%
    \expandafter\let\csname svn@ignore@ext@\svn@temp\endcsname\relax%
  }%
}
%    \end{macrocode}
% \end{macro}

% The following extensions are ignored by default.
%    \begin{macrocode}
\svnignoreextensions{aux,toc,out,svn,svx,cls,sty,cfg,enc}
%    \end{macrocode}

% Check at the end of every input file if files should be extra groups and
% declare this file as group if its extension is not configured to be ignored.
%    \begin{macrocode}
\svnmulti@atendinputfile{%
  \if@svnmulti@filesasgroups
  \ifsvnfilesasgroup
    \expandafter\ifx\csname svn@ignore@ext@\finkext\endcsname\relax
     \svnfileasgroup
    \fi
  \fi
  \fi
}
%    \end{macrocode}

%    \begin{macrocode}
\fi
%    \end{macrocode}

% \subsection{External Files}
% Macros to declare external files and load the keywords from |.svx| files
% generated by |svn-multi.pl|.
%    \begin{macrocode}
\if@svnmulti@external
%    \end{macrocode}

% \begin{macro}{\svnexternalgroup}
% Defines the default group of external files. The default is empty which means
% the current group.
%    \begin{macrocode}
\def\svnexternalgroup#1{%
    \def\svn@externalgroup{#1}%
}
\def\svn@externalgroup{}
%    \end{macrocode}
% \end{macro}

% \begin{macro}{\svnexternal}
% Writes the current input file path and its argument as arguments of
% \cs{@svnexternal} into the |.svn| file.
%    \begin{macrocode}
\newcommand*\svnexternal[2][\svn@externalgroup]{%
  \if@filesw
      \svn@checkwrite
      \begingroup
      \let\svn@temp\finkpath
      \svn@removedotslash\svn@temp
      \immediate\write\svn@write{%
          \noexpand\@svnexternal[#1]{\svn@temp}{#2}%
      }%
      \endgroup
  \fi
}
%    \end{macrocode}
% \end{macro}

% \begin{macro}{\svnexternalpath}
% Writes its argument as argument of \cs{@svnexternalpath} into the |.svn| file.
%    \begin{macrocode}
\def\svnexternalpath#1{%
  \if@filesw
      \svn@checkwrite
      \immediate\write\svn@write{%
          \noexpand\@svnexternalpath{#1}%
      }%
  \fi
}
%    \end{macrocode}
% \end{macro}

% \begin{macro}{\@svnexternal}
% \begin{macro}{\@svnexternalpath}
% Discards the argument(s). These macros and their arguments are only used by
% the external |svn-multi.pl| script.
%    \begin{macrocode}
\newcommand*\@svnexternal[3][]{}
\def\@svnexternalpath#1{}
%    \end{macrocode}
% \end{macro}
% \end{macro}


% \begin{macro}{\svnexternalfile}
% This macro is generated by |svn-multi.pl| and should not be used by the user.
% If files-as-group is enabled some special characters are disabled
% and the \cs{svn@externalfile} is called to read the file name. Otherwise the
% argument is simply removed.
%    \begin{macrocode}
\def\svnexternalfile{%
  \begingroup
  \catcode`\_=12
  \catcode`\&=12
  \catcode`\^=12
  \catcode`\$=12
  \catcode`\#=12
  \expandafter\svn@externalfile
}
%    \end{macrocode}
% \end{macro}

% \begin{macro}{\svn@externalfile}
% Ends group which began in \cs{svnexternalfile} and calls the appropriate
% macros.
%    \begin{macrocode}
\def\svn@externalfile#1{%
  \endgroup
  \svn@addfiletogroup{#1}{\svn@g}%
  \if@svnmulti@filesasgroups
  \ifsvnfilesasgroup
    \svn@fileasgroup{#1}%
  \fi
  \fi
}
%    \end{macrocode}
% \end{macro}

% \begin{macro}{\svn@inputsvx}
% Macro to load |.svx| files.  The current keyword group is saved away and
% restored after the |.svx| file is loaded. |\IfFileExists| with |\@@input| is
% used because |\InputIfFileExists| got redefined by the |fink| package and
% there is no need to use |fink| for the |.svx| files.
%    \begin{macrocode}
\def\svn@inputsvx#1{%
  \begingroup
  \let\svn@temp\svn@g
  \edef\svn@svxfile{#1}%
  \IfFileExists{#1}{\@@input#1\relax}{}{}%
  \global\let\svn@g\svn@temp
  \endgroup
}
%    \end{macrocode}
% \end{macro}

% If \op{external} option is not enabled a placeholder macro is defined which
% simply ignores its argument.
%    \begin{macrocode}
\else
\def\svnexternalfile#1{}%
\fi
%    \end{macrocode}


% \subsection{Auto loading of \texttt{.svx} files}
% Auto loading of |.svx| files at the begin of |\input| or |\include| files
% using the \cs{svnmulti@atbegininputfile} macro.
% The macros \cs{svn@addfiletogroup} and \cs{svnfileasgroup} are used to do the
% actual work.
%    \begin{macrocode}
\if@svnmulti@autoload

\svnmulti@atbegininputfile{%
  \svn@inputsvx{\fink@nextdir\fink@nextbase.svx}%
}
%    \end{macrocode}

% The main |.svx| is loaded at the end of the package.
%    \begin{macrocode}
\AtEndOfPackage{%
  \svn@inputsvx{\jobname.svx}%
}
%    \end{macrocode}

%    \begin{macrocode}
\fi
%    \end{macrocode}


% \subsection{Support for Graphic Packages}

% \subsubsection{Common Code}
%    \begin{macrocode}
\if@svnmulti@anygraphic
%    \end{macrocode}

% \begin{macro}{\svngraphicsgroup}
% Defines the default group of graphics files. The default is empty which means
% the current group.
%    \begin{macrocode}
\def\svngraphicsgroup#1{%
    \def\svn@graphicsgroup{#1}%
}
\def\svn@graphicsgroup{}
%    \end{macrocode}
% \end{macro}

% \begin{macro}{\svnignoregraphic}
% Ignores the given graphic file by defining a special macro.
%    \begin{macrocode}
\def\svnignoregraphic#1{%
  \expandafter\def\csname svn@ignoregraphic@#1\endcsname{}%
}
%    \end{macrocode}
% \end{macro}

% \begin{macro}{\svnconsidergraphic}
% Deletes the special ignore macro to consider the graphic again.
%    \begin{macrocode}
\def\svnconsidergraphic#1{%
  \expandafter\let\csname svn@ignoregraphic@#1\endcsname\relax%
}
%    \end{macrocode}
% \end{macro}

%    \begin{macrocode}
\fi
%    \end{macrocode}

% \subsubsection{Package \texttt{graphics}}
% Automatic declaration of all images included by |\includegraphics| from the
% |graphics| package as external files. We use the |\Gin@setfile| macro from
% that package which receives the image file name as third argument.
%    \begin{macrocode}
\if@svnmulti@graphics
\RequirePackage{graphics}[2006/02/20]

\let\svnmulti@Gin@setfile\Gin@setfile
\renewcommand*{\Gin@setfile}[3]{%
    \expandafter\ifx\csname svn@ignoregraphic@#3\endcsname\relax%
      \svnexternal[\svn@graphicsgroup]{{#3}}%
    \fi
    \svnmulti@Gin@setfile{#1}{#2}{#3}%
}
\fi
%    \end{macrocode}

% \subsubsection{Package \texttt{pgf}}
% The |pgf| macro |\pgf@declareimage| which is called by the user macro
% |\pgfdeclareimage| is used.
%    \begin{macrocode}
\if@svnmulti@pgfimages
\RequirePackage{pgf}[2008/01/15]

\let\svnmulti@pgf@declareimage\pgf@declareimage
\renewcommand*{\pgf@declareimage}[3][]{%
  \svnmulti@pgf@declareimage[#1]{#2}{#3}%
%    \end{macrocode}
% At this point the used image filename is defined by |\pgf@filename| and the
% image itself is defined by |\pgf@image@#2!| which is a |\let| copy of
% temporary |\pgf@image|.  An own copy of this is created and the old name
% |\pgf@image@#2!| is used to execute \cs{svnexternal} every time the image is
% included using |\pgfuseimage|.
% \begin{macrocode}
  \ifx\pgf@filename\empty\else
    \expandafter\ifx\csname svn@ignoregraphic@\pgf@filename\endcsname\relax%
      \expandafter\global\expandafter%
        \let\csname svnmulti@pgf@image@#2!\endcsname=\pgf@image%
      \expandafter\xdef\csname pgf@image@#2!\endcsname{%
        \noexpand\svnexternal[\noexpand\svn@graphicsgroup]{{\pgf@filename}}%
        \csname svnmulti@pgf@image@#2!\endcsname
      }%
    \fi
  \fi
}
\fi
%    \end{macrocode}
%
% \subsection{Table of Revisions}
%
%    \begin{macrocode}
\if@svnmulti@table
\ifx\tableofcontents\relax\else
%    \end{macrocode}
%
% \begin{macro}{\l@svngroup}
% Macro used by |\contentsline{svngroup}{..}|.
%    \begin{macrocode}
\def\l@svngroup#1#2{\l@@svngroup{#1}#2}
\def\l@@svngroup#1#2#3#4{%
    \@dottedtocline {1}{1.5em}{2.3em}{#1}{#2}%
}
%    \end{macrocode}
% \end{macro}
%
% \begin{macro}{\svnrevisionsname}
% Simple definition for now. Language support over `babel's |\languagename|
% possible.
%    \begin{macrocode}
\def\svnrevisionsname{Table of Revisions}%
%    \end{macrocode}
% \end{macro}
%
% \begin{macro}{\svn@svt}
% File ending for table of revision auxiliary file. A macro is used to allow
% redefinition by the user if another package is uses the same ending.
%    \begin{macrocode}
\def\svn@svt{svt}
%    \end{macrocode}
% \end{macro}
%
% \begin{macro}{\svn@svt}
% File ending for table of revision auxiliary file. A macro is used to allow
% redefinition by the user if another package is uses the same ending.
%    \begin{macrocode}
\def\svn@svtwrite{\csname ft@\svn@svt\endcsname}
%    \end{macrocode}
% \end{macro}

% TODO: write own macros
% \begin{macro}{\tableofrevisions}
% The |\tableofcontents| macro from standard \LaTeX\ is adapted for this macro.
% Classes which provide chapters will get a different table then one which not.
%    \begin{macrocode}
\AtBeginDocument{%
\ifx\chapter\@undefined

%% Adapted from the \tableofcontents macro, LaTeX `article' class [2005/09/16 v1.4f]
\newcommand\tableofrevisions{%
  \section*{\svnrevisionsname
    \@mkboth{\MakeUppercase\svnrevisionsname}{\MakeUppercase\svnrevisionsname}}%
  \@starttoc{\svn@svt}%
}

\else

%% Adapted from the \tableofcontents macro, LaTeX `book' class [2005/09/16 v1.4f]
\newcommand\tableofrevisions{%
  \if@twocolumn
    \@restonecoltrue\onecolumn
  \else
    \@restonecolfalse
  \fi
  \chapter*{\svnrevisionsname
    \@mkboth{\MakeUppercase\svnrevisionsname}{\MakeUppercase\svnrevisionsname}}%
  \@starttoc{\svn@svt}%
  \if@restonecol\twocolumn\fi
}

\fi
}
%    \end{macrocode}
% \end{macro}
%
%    \begin{macrocode}
\fi % defined \tableofcontents
%    \end{macrocode}

% \begin{macro}{\svn@writerow}
% Writes a table row by using |\svn@tabcell| and |\svn@tabcellarg| defined by
% the |\svn@writeXXXrow| macro below. First argument is row type ('group',
% 'file', 'global') and second one is a type specific argument provided to
% |\svntab|\meta{type}.
%    \begin{macrocode}
\def\svn@writerow#1#2{%
  \immediate\write\svn@svtwrite{%
    \expandafter\noexpand\csname svn#1row\endcsname
    \expandafter\noexpand\csname svntab#1\endcsname{#2}\space
    \@ampersamchar\space
    \svn@tabcell{rev}\space\@ampersamchar\space
    \svn@tabcell{author}\space\@ampersamchar\space
    \noexpand\svntabdate%
    \svn@tabcellarg{year}%
    \svn@tabcellarg{month}%
    \svn@tabcellarg{day}%
    \svn@tabcellarg{hour}%
    \svn@tabcellarg{minute}%
    \svn@tabcellarg{second}%
    \svn@tabcellarg{timezonehour}%
    \svn@tabcellarg{timezoneminute}%
    \@backslashchar\@backslashchar
  }
}
%    \end{macrocode}
% \end{macro}

% \begin{macro}{\svn@writegrouprow}
%    \begin{macrocode}
\def\svn@writegrouprow#1{% #1 is current group
  \begingroup
  \def\svn@tabcellarg##1{{\csname @svng@#1@##1\endcsname}}
  \def\svn@tabcell##1{\expandafter\noexpand\csname svntab##1\endcsname%
    \svn@tabcellarg{##1}%
  }%
  \svn@writerow{group}{#1}%
  \endgroup
}
%    \end{macrocode}
% \end{macro}

% \begin{macro}{\svn@writefilerow}
%    \begin{macrocode}
\def\svn@writefilerow#1{% #1 is file
  \expandafter
  \ifx\csname @svng@#1@rev\endcsname\relax\else
  \begingroup
  \def\svn@tabcellarg##1{{\csname @svng@#1@##1\endcsname}}
  \def\svn@tabcell##1{\expandafter\noexpand\csname svntab##1\endcsname%
    \svn@tabcellarg{##1}%
  }%
  \svn@writerow{file}{#1}%
  \endgroup
  \fi
}
%    \end{macrocode}
% \end{macro}

% \begin{macro}{\svn@writeglobalrow}
%    \begin{macrocode}
\def\svn@writeglobalrow{% no arguments
  \begingroup
  \def\svn@tabcellarg##1{{\csname @svn@##1\endcsname}}
  \def\svn@tabcell##1{\expandafter\noexpand\csname svntab##1\endcsname%
    \svn@tabcellarg{##1}%
  }%
  \svn@writerow{global}{}%
  \endgroup
}
%    \end{macrocode}
% \end{macro}

% \subsubsection{Table Format Macros}
% Generic format macro used in the |.svt| file. Can be redefined by the user to
% change table format. % TODO: User documentation needed!
%    \begin{macrocode}
\def\svntabledef{lrll}
\def\svntable{%
  \begin{tabular}{\svntabledef}%
    \hline
    \strut Name & Rev & Last Author & Last Changed At \\\hline
}
\def\endsvntable{\hline\end{tabular}}
\def\svnglobalrow{}
\def\svngrouprow{}
\def\svnfilerow{}
\def\svntabgroup#1{Group `#1'}
\def\svntabglobal{Whole Document}
\def\svntabfile{%
  \begingroup
  \catcode`\_=12
  \catcode`\&=12
  \catcode`\^=12
  \catcode`\$=12
  \catcode`\#=12
  \svn@tabfile
}
\def\svn@tabfile#1{\endgroup\hspace*{1em}File `\texttt{#1}'}
\def\svntabrev{}
\def\svntabauthor#1{\svnFullAuthor{#1}}
\def\svntabdate#1#2#3#4#5#6#7#8{%
      #1-#2-#3 #4:#5:#6 #7#8%
}
%    \end{macrocode}

%    \begin{macrocode}
\fi
%    \end{macrocode}
%

% \subsection{Other macros}
% \begin{macro}{\svn}
% \begin{macro}{\svn*}
% After *-testing, the intermediate macros |\svn@s| and |\svn@n| are called to
% strip the |{ }| from |\svn|[|*|]|{$...$}| and to remove the |*|. Then the
% actual macros are called to strip the dollars with or without the space
% before the last dollar.
% \changes{v1.2}{2007/06/22}{Added star version. Normal version was not
% changed to not break existing documents with user defined keywords without
% leading space.}
%    \begin{macrocode}
\newcommand{\svn}{\@ifnextchar{*}{\svn@s}{\svn@n}}
\def\svn@n#1{\@svn@n#1}
\def\svn@s*#1{\@svn@s#1}
\def\@svn@n$#1${#1}
\def\@svn@s$#1 ${#1}
%    \end{macrocode}
% \end{macro}
% \end{macro}

% \begin{macro}{\svnnolinkurl}
% This code is taken from the |hyperref| package and is the definition of
% |\url| just without the part which creates the actual hyperlink. This needs
% of course the |hyperref| package. A warning is given if it isn't loaded.
% \changes{v1.2}{2007/06/22}{New macro}
%    \begin{macrocode}
%% Adapted from the \url macro of the `hyperref` package.
\DeclareRobustCommand*{\svnnolinkurl}{%
    \@ifundefined{hyper@normalise}%
    {\PackageWarning{svn-multi}{Package hyperref is needed for \noexpand
    \svnnolinkurl.}}%
    {\hyper@normalise\svnnolinkurl@}}%
\def\svnnolinkurl@#1{\Hurl{#1}}
%    \end{macrocode}
% \end{macro}

% \subsection{Auxiliary file generation and read-back}
%
% Reread output from last compile run if it exists.
%    \begin{macrocode}
\IfFileExists{\jobname.svn}{\@@input\jobname.svn\relax}{}
%    \end{macrocode}

% \begin{macro}{\svn@checkwrite}
% Checks if .svn file is already open and if not opens it. This makes sure that
% the file is only created if really needed. The macro is only needed once, so
% it's redefines itself to |\relax| at the end.
%    \begin{macrocode}
\def\svn@checkwrite{%
 \@ifundefined{svn@write}{%
   \newwrite\svn@write
   \immediate\openout\svn@write=\jobname.svn\relax%
   \immediate\write\svn@write{\@percentchar\space SVN Keyword cache}%
   %\immediate\write\svn@write{\noexpand\makeatletter}%
 }{}%
 \let\svn@checkwrite=\relax
}
%    \end{macrocode}
% \end{macro}
%
% \begin{macro}{\svn@writeaux}
% This macro writes the |.svn| auxiliary file and is called from a
% |\AtEndDocument| macro later on.
%    \begin{macrocode}
{\catcode`\&=12
\gdef\@ampersamchar{&}
}
\def\svn@writeaux{%
%    \end{macrocode}
% We first check if we have something to save. Revision, date and author must be
% non-empty. This suppresses the auxiliary file if the user doesn't use the
% appropriate macros but other provided by this package.
%    \begin{macrocode}
  \if@filesw \ifx\@svn@rev\empty\else \ifnum\@svn@rev=0\else
  \ifx\@svn@date\empty\else \ifx\@svn@author\empty\else
%    \end{macrocode}
% Write document global values:
%    \begin{macrocode}
  \svn@checkwrite
  \immediate\write\svn@write{^^J%
    \@percentchar\space Global values:^^J%
    \noexpand\def\noexpand\svnrev{\@svn@rev}^^J%
    \noexpand\def\noexpand\svndate{\@svn@date}^^J%
    \noexpand\def\noexpand\svnauthor{\@svn@author}^^J%
    \noexpand\def\noexpand\svnyear{\@svn@year}^^J%
    \noexpand\def\noexpand\svnmonth{\@svn@month}^^J%
    \noexpand\def\noexpand\svnday{\@svn@day}^^J%
    \noexpand\def\noexpand\svnhour{\@svn@hour}^^J%
    \noexpand\def\noexpand\svnminute{\@svn@minute}^^J%
    \noexpand\def\noexpand\svnsecond{\@svn@second}^^J%
    \noexpand\def\noexpand\svntimezonehour{\@svn@timezonehour}^^J%
    \noexpand\def\noexpand\svntimezoneminute{\@svn@timezoneminute}^^J%
    \noexpand\def\noexpand\svng@@files{\csname @svng@\svn@g @files\endcsname}^^J%
  }%
%    \end{macrocode}
% Write keyword group values if groups were specified:
%    \begin{macrocode}
  \if@svnmulti@groups
  \ifx\svn@glist\empty\else
    \begingroup
    \if@svnmulti@table
      \newwrite\svn@svtwrite
      \immediate\openout\svn@svtwrite=\jobname.\svn@svt\relax
      \immediate\write\svn@svtwrite{\noexpand\svntable}%
      \begingroup
      \svn@writeglobalrow{}%
      \ifx\@svng@@files\@undefined\else
      \ifx\@svng@@files\relax\else
          \@for\svn@file:=\@svng@@files\do{%
             \svn@writefilerow{\svn@file}%
          }%
      \fi\fi
      \endgroup
    \fi
    \immediate\write\svn@write{^^J\@percentchar\space SVN File Groups: \svn@glist}%
%    \end{macrocode}
% For every activated keyword group check if any keywords got recorded
% and write to the output file:
%    \begin{macrocode}
    \@for\svn@g:=\svn@glist\do{%
      \expandafter\ifx\csname @svng@\svn@g @rev\endcsname\relax\else
      \expandafter\ifnum\csname @svng@\svn@g @rev\endcsname>-1\relax
        \def\svn@writekw##1{%
          \noexpand\@namedef{svng@\svn@g @##1}%
          {\csname @svng@\svn@g @##1\endcsname}^^J%
        }
        \immediate\write\svn@write{%
          \@percentchar\space\svn@g^^J%
          \svn@writekw{rev}%
          \svn@writekw{date}%
          \svn@writekw{author}%
          \svn@writekw{year}%
          \svn@writekw{month}%
          \svn@writekw{day}%
          \svn@writekw{hour}%
          \svn@writekw{minute}%
          \svn@writekw{second}%
          \svn@writekw{timezonehour}%
          \svn@writekw{timezoneminute}%
          \svn@writekw{files}%
        }%
        \if@svnmulti@table
            \begingroup
            \svn@writegrouprow{\svn@g}%
            \expandafter\let\expandafter\svn@temp\csname @svng@\svn@g @files\endcsname
            \ifx\svn@temp\relax\else
                \@for\svn@file:=\svn@temp\do{%
                    \svn@writefilerow{\svn@file}%
                }%
            \fi
            \endgroup
        \fi
      \fi\fi
    }%
    \if@svnmulti@table
      \immediate\write\svn@svtwrite{\noexpand\endsvntable}%
      \immediate\closeout\svn@svtwrite%
    \fi
    \endgroup
  \fi\fi
%    \end{macrocode}
% Finally close output file:
%    \begin{macrocode}
  \immediate\closeout\svn@write%
 \fi\fi\fi\fi\fi
}
%    \end{macrocode}
% \end{macro}
%
% At the end of document the values are written to the auxiliary file.
%    \begin{macrocode}
\AtEndDocument{%
\svn@writeaux
}
%    \end{macrocode}
%
% \subsection{Backward compatibility wrapper \texttt{svnkw.sty}}
% For backward compatibility a wrapper file with the old package name |svnkw| is
% provided. Newer documents should use the name |svn-multi|.
% \setcounter{CodelineNo}{0}
% \iffalse
%</package>
%<*wrapper>
% \fi
%    \begin{macrocode}
\NeedsTeXFormat{LaTeX2e}[1999/12/01]
\ProvidesPackage{svnkw}
[\filedate\space\fileversion\space Backward compatibility wrapper for svn-multi]
\PackageWarning{svnkw}{The package 'svnkw' got renamed to 'svn-multi'
and is now only a backward compatibility wrapper which loads 'svn-multi'.
Please adjust your document preamble to use the new name.}
\RequirePackage{svn-multi}[\filedate]
%    \end{macrocode}
% \iffalse
%</wrapper>
% \fi
%
% \Finale
\endinput
