\documentclass[a4paper,oneside]{scrbook}
\usepackage[groups]{svn-multi}[2009/03/06]
\svnidlong
{$HeadURL: svn://server/group_example.tex $}
{$LastChangedDate: 2000-01-01 01:00:01 +0000 (Main file) $}
{$LastChangedRevision: 98 $}
{$LastChangedBy: author3 $}

\let\ifvtex=\relax
\usepackage{hyperref}

% This is an example document to show the group feature of svn-multi 2.0.
% Please the notes below.

% Version control information table for each chapter
\newcommand{\chaptervctable}{%
\textbf{Version Control Information for this chapter}\\[\bigskipamount]%
\begin{tabular}{ll}
Last Changed Revision & \svnfilerev\\
Last Changed Author   & \svnfileauthor\\
Last Changed Date     & \svnfiledate\\
\end{tabular}
}

% This file generates some example sub files from this file to avoid a lot of small
% example files in the CTAN directory. In real life there would be separate
% files from the beginning.

\begin{filecontents}{group_example_part1a.tex}
\svnidlong
{$HeadURL: svn://server/group_example_part1a.tex $}
{$LastChangedDate: 2000-01-01 01:00:01 +0000 (Chapter 1a) $}
{$LastChangedRevision: 101 $}
{$LastChangedBy: author1 $}

\chapter{Subfile 1a}
\chaptervctable
\end{filecontents}

\begin{filecontents}{group_example_part1b.tex}
\svnidlong
{$HeadURL: svn://server/group_example_part1b.tex $}
{$LastChangedDate: 2001-01-01 00:00:01 +0000 (Subfile 1b) $}
{$LastChangedRevision: 102 $}
{$LastChangedBy: author2 $}

\chapter{Subfile 1b}
\chaptervctable
\end{filecontents}

\begin{filecontents}{group_example_part1c.tex}
\svnidlong
{$HeadURL: svn://server/group_example_part1c.tex $}
{$LastChangedDate: 2000-01-01 00:00:01 +0000 (Subfile 1c) $}
{$LastChangedRevision: 104 $}
{$LastChangedBy: author3 $}

\chapter{Subfile 1c}
\chaptervctable
\end{filecontents}

\begin{filecontents}{group_example_part2a.tex}
\svnidlong
{$HeadURL: svn://server/group_example_part2a.tex $}
{$LastChangedDate: 2000-01-01 00:00:01 +0000 (Subfile 2a) $}
{$LastChangedRevision: 100 $}
{$LastChangedBy: author1 $}

\chapter{Subfile 2a}
\chaptervctable
\end{filecontents}

\begin{filecontents}{group_example_part2b.tex}
\svnidlong
{$HeadURL: svn://server/group_example_part2b.tex $}
{$LastChangedDate: 2000-01-01 00:00:01 +0000 (Subfile 2b) $}
{$LastChangedRevision: 101 $}
{$LastChangedBy: author1 $}

\chapter{Subfile 2b}
\chaptervctable
\end{filecontents}

\begin{filecontents}{group_example_part3a.tex}
\svnidlong
{$HeadURL: svn://server/group_example_part3a.tex $}
{$LastChangedDate: 2000-01-01 00:00:01 +0000 (Subfile 3a) $}
{$LastChangedRevision: 104 $}
{$LastChangedBy: author3 $}

\chapter{Subfile 3a}
\chaptervctable
\end{filecontents}

\begin{filecontents}{group_example_part3b.tex}
\svnidlong
{$HeadURL: svn://server/group_example_part3b.tex $}
{$LastChangedDate: 2000-01-01 00:00:01 +0000 (Subfile 3b) $}
{$LastChangedRevision: 103 $}
{$LastChangedBy: author2 $}

\chapter{Subfile 3b}
\chaptervctable
\end{filecontents}

\begin{filecontents}{group_example_part4a.tex}
\svnidlong
{$HeadURL: svn://server/group_example_part4a.tex $}
{$LastChangedDate: 2000-01-01 00:00:01 +0000 (Subfile 4a) $}
{$LastChangedRevision: 99 $}
{$LastChangedBy: author3 $}

\chapter{Subfile 4a}
\chaptervctable
\end{filecontents}

\begin{filecontents}{group_example_part4b.tex}
\svnidlong
{$HeadURL: svn://server/group_example_part4b.tex $}
{$LastChangedDate: 2000-01-01 00:00:01 +0000 (Subfile 4b) $}
{$LastChangedRevision: 105 $}
{$LastChangedBy: author2 $}

\chapter{Subfile 4b}
\chaptervctable
\end{filecontents}

\begin{filecontents}{group_example_end.tex}
\svnidlong
{$HeadURL: svn://server/group_example_end.tex $}
{$LastChangedDate: 2000-01-01 00:00:01 +0000 (End) $}
{$LastChangedRevision: 100 $}
{$LastChangedBy: author1 $}

\chapter{End credits}
\chaptervctable
\end{filecontents}


\svnRegisterAuthor{author1}{Andy Author, I.}
\svnRegisterAuthor{author2}{A. Author, II.}
\svnRegisterAuthor{author3}{Anthony Author, III.}

% Have VC info on every part page
\let\origpart=\part
\def\part{%
\setpartpreamble{%
\vspace*{5cm}
\textbf{Version Control Information for this part:}\\[\bigskipamount]%
\begin{tabular}{ll}
Last Changed Revision & \svncgrev\\
Last Changed Author   & \svncgauthor\\
Last Changed Date     & \svncgdate\\
\end{tabular}
}\origpart
}

\begin{document}

% Custom titlepage:
\vspace*{4cm}
{\centering
\Huge\texttt{svn-multi} Keyword Groups Example Document\\
\Large Martin Scharrer\\
\large 2009/03/01\\
}%
\vspace{4cm}
This is an example and test document for the group feature of svn-multi 2.0.
Please note that the revision keywords were generated manually for testing
purposes. The dates are not consistent with the revisions and contain a debug
name instead of the date text like `(Sat, January 1 2009)'. This doesn't
influence the correct functionality of the svn-multi package.

\bigskip
{\hbox{}\hfill Happy \TeX ing!}\\
\vfill%
{\large Version Control Information for this document:\\[\bigskipamount]}%
\begin{tabular}{ll}
Last Changed Revision & \svnrev\\
Last Changed Author   & \svnauthor\\
Last Changed Date     & \svndate\\
\end{tabular}
\thispagestyle{empty}
\clearpage

\chapter*{Version Control Overview}

\begin{tabular}{lllll}
   \hline
   Group & Revision & Author & Date \\ \hline
   Global & \svnrev & \svnFullAuthor{\svnauthor} & \svntoday\ \svntime\ \svntimezone \\
   \svngroup{abc}abc &
   \svncgrev & \svnFullAuthor{\svncgauthor} & \svncgtoday\ \svncgtime\ \svntimezone \\
   \svngroup{def}def &
   \svncgrev & \svnFullAuthor{\svncgauthor} & \svncgtoday\ \svncgtime\ \svntimezone \\
   \svngroup{ghi}ghi &
   \svncgrev & \svnFullAuthor{\svncgauthor} & \svncgtoday\ \svncgtime\ \svntimezone \\
   \svngroup{jkl}jkl &
   \svncgrev & \svnFullAuthor{\svncgauthor} & \svncgtoday\ \svncgtime\ \svntimezone \\
   \hline
\end{tabular}
\svngroup{}
\clearpage

\svngroup{abc}
\part{Abc}
\include{group_example_part1a}
\include{group_example_part1b}
\include{group_example_part1c}

\svngroup{def}
\part{Def}
\include{group_example_part2a}
\include{group_example_part2b}

\svngroup{ghi}
\part{Ghi}
\include{group_example_part3a}
\include{group_example_part3b}

\svngroup{jkl}
\part{Jkl}
\include{group_example_part4a}
\include{group_example_part4b}

% The rest doesn't belong to a file group:
\svngroup{}
\include{group_example_end}

\end{document}

