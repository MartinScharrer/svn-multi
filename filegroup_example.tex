\documentclass[a4paper,oneside]{scrbook}
\usepackage{svn-multi}
\svnidlong
{$HeadURL: svn+ssh://www.scharrer-online.de/home/martin/svn/src/trunk/latex/svn-multi-2/filegroup_example.tex $}
{$LastChangedDate: 2009-02-28 00:26:17 +0000 (Sat, 28 Feb 2009) $}
{$LastChangedRevision: 226 $}
{$LastChangedBy: martin $}
\svnid{$Id: filegroup_example.tex 226 2009-02-28 00:26:17Z martin $}

\usepackage{hyperref}
\let\include=\url

\svnRegisterAuthor{martina}{Martin A.}
\svnRegisterAuthor{martinb}{Martin B.}
\svnRegisterAuthor{martinj}{Martin J.}

\title{\texttt{svn-multi} File groups example document}
\author{Martin Scharrer}
\date{Revision: \svnrev}

\begin{document}
\maketitle
\tableofcontents
Last changed by: \svncfg{author} | \svnFullAuthor{\svnfg{abc}{author}}

\part{Abc}
\svnfilegroup{abc}
Last changed by: \svncfg{author}
\svnid{$Id: filegroup_example.tex 226 2009-02-27 00:26:17Z martina $}
\include{filegroup_example_part1a}
\include{filegroup_example_part1b}
\svnid{$Id: filegroup_example.tex 228 2009-02-28 01:26:17Z martinb $}
\include{filegroup_example_part1c}

Current: \svncfg{rev}\\
Def: \svnfg{def}{rev}\\

\part{Def}
\svnfilegroup{def}
Last changed by: \svncfg{author} | \svnFullAuthor{\svncfg{author}}
\svnid{$Id: filegroup_example.tex 222 2009-02-29 02:26:17Z martinj $}
\include{filegroup_example_part2a}
\include{filegroup_example_part2b}
Current: \svncfg{rev}\\
Abc: \svnfg{abc}{rev} | \svnFullAuthor{\svnfg{abc}{author}} \\

\part{Ghi}
\svnfilegroup{ghi}
\svnidlong
{$HeadURL: svn+ssh://www.scharrer-online.de/home/martin/svn/src/trunk/latex/svn-multi-2/filegroup_example.tex $}
{$LastChangedDate: 2009-02-28 00:26:17 +0000 (Sat, 28 Feb 2009) $}
{$LastChangedRevision: 226 $}
{$LastChangedBy: martin $}
\include{filegroup_example_part3a}
\include{filegroup_example_part3b}

% Part 4 inside an include file
\include{filegroup_example_part4}
\include{filegroup_example_part4a}
\include{filegroup_example_part4b}
\include{filegroup_example_part4c}

\svnfilegroup{}
\svnid{$Id: filegroup_example.tex 152 2009-02-29 02:26:17Z martinj $}

\end{document}

% Vim setting modeline - please do not delete
% vim: tabstop=2 expandtab autoindent
